\documentclass[fontsize=16pt,a4paper,DIV=17,parskip=half]{scrartcl}

\usepackage[utf8]{inputenc}
\usepackage[dvipsnames]{xcolor}
\usepackage[hyperindex]{hyperref}
\usepackage{lastpage}
\usepackage{tikz}
\usepackage{tikzpagenodes}
\usepackage{pgfplots}
\usepackage{graphicx}
\usepackage{enumitem}
\usepackage{scrlayer-scrpage}
\usepackage{xspace}
\usepackage{float}
\usepackage{amsfonts,amsmath,amsthm}
\usepackage{thmtools}
\usepackage[english]{babel}
\usepackage[autostyle, style=english]{csquotes}
\usepackage{subfigure}
\usepackage{listings}
\usepackage{mleftright}
\usepackage{tikz-cd}
\usepackage[framemethod=TikZ]{mdframed}
\usepackage{soulutf8}
\usepackage{mathtools}
\usepackage{multicol}
\usepackage{nameref}
\usepackage{stmaryrd}
\usepackage{fvextra}
\usepackage{adjustbox}
\usepackage{fontawesome}
\usepackage{fontspec}
\usepackage{tikz-3dplot}
\usepackage{mathpartir}
\usepackage{etoolbox}
\usepackage{todonotes}
\usepackage{verbatim}
\usepackage{algorithm}
\usepackage[rightComments=false,beginComment=$\qquad\triangleright$~]{algpseudocodex}
\usepackage{ebproof}
\usepackage[makeroom, thicklines]{cancel}
\renewcommand{\CancelColor}{\color{nicered}}
\usepackage{setspace}
\usepackage{fourier-otf}
\setmathfont{Erewhon-Math.otf}[CharacterVariant={20}]
\usepackage{juliamono}
\usepackage[osf]{Alegreya}
\usepackage[osf]{AlegreyaSans}
\let\mathds\mathbb

\usepackage{juliamono}
\usepackage[notion,hyperref]{knowledge}

\knowledgedirective {problem}{notion,style=problem,intro style=intro problem}
\knowledgestyle{intro problem}{smallcaps, ensuretext, color=deepblue!70}
\knowledgestyle{problem}{smallcaps, ensuretext, color=deepblue}

\knowledgenewrobustcmd\blank{\cmdkl{\texttt{▢}}}
\knowledgenewrobustcmd\start{\cmdkl{\texttt{▷}}}
\knowledgenewrobustcmd\dtime[1]{\ensuremath{\cmdkl{\mathrm{DTIME}(}{#1}\cmdkl{)}}}
\knowledgenewrobustcmd\ntime[1]{\ensuremath{\cmdkl{\mathrm{NTIME}(}{#1}\cmdkl{)}}}
\knowledgenewrobustcmd\dspace[1]{\ensuremath{\cmdkl{\mathrm{DSPACE}(}{#1}\cmdkl{)}}}
\knowledgenewrobustcmd\nspace[1]{\ensuremath{\cmdkl{\mathrm{NSPACE}(}{#1}\cmdkl{)}}}
\knowledgenewrobustcmd\p{\ensuremath{\cmdkl{\mathsf{P}}}}
\knowledgenewrobustcmd\ip{\ensuremath{\cmdkl{\mathsf{IP}}}}
\knowledgenewrobustcmd\np{\ensuremath{\cmdkl{\mathsf{NP}}}}
\knowledgenewrobustcmd\ls{\ensuremath{\cmdkl{\mathsf{L}}}}
\knowledgenewrobustcmd\nls{\ensuremath{\cmdkl{\mathsf{NL}}}}
\knowledgenewrobustcmd\pspace{\ensuremath{\cmdkl{\mathsf{PSPACE}}}}
\knowledgenewrobustcmd\exptime{\ensuremath{\cmdkl{\mathsf{EXPTIME}}}}
\knowledgenewrobustcmd\nexptime{\ensuremath{\cmdkl{\mathsf{NEXPTIME}}}}
\knowledgenewrobustcmd\code[1]{\ensuremath{\cmdkl{\langle}#1\cmdkl{\rangle}}}
\knowledgenewrobustcmd\redP{\ensuremath{\mathrel{\cmdkl{\le_\mathrm{P}}}}}
\knowledgenewrobustcmd\redL{\ensuremath{\mathrel{\cmdkl{\le_\mathrm{L}}}}}

\knowledgenewrobustcmd\oracle[2]{\ensuremath{\cmdkl{{#1}^{#2}}}}
\knowledgenewrobustcmd\oraclem[2]{\ensuremath{\cmdkl{{#1}^{#2}}}}

\knowledgenewrobustcmd\advice[2]{\ensuremath{\cmdkl{{#1}/{#2}}}}

\knowledgenewrobustcmd\advicelog{\ensuremath{\cmdkl{\mathsf{log}}}}
\knowledgenewrobustcmd\advicepoly{\ensuremath{\cmdkl{\mathsf{poly}}}}


\renewcommand{\mathsf}[1]{\text{\normalfont\sffamily#1}}
\renewcommand{\texttt}[1]{\text{\normalfont\ttfamily#1}}

\ebproofset{right label template=$\inserttext$, left label template=\tiny$\inserttext$, center=false}

\RedeclareSectionCommand[beforeskip=0.10em, afterskip=0.10em]{section}
\RedeclareSectionCommand[beforeskip=0.05em, afterskip=0.001em plus 0em]{subsection}


\fvset{bgcolor=lightgray!10,backgroundcolorpadding=3pt}

\MakeOuterQuote{"}

\newcommand\redQuestionBox{
  \tikz[baseline]{
    \node[rectangle,fill=nicered,anchor=base,rounded corners=2pt] (A) {\color{white}\textsf{\textbf{?}}\ensuremath{\,}};
  }
}

\definecolor{blue}{HTML}{5aa9e6}
\colorlet{deeppurple}{DarkOrchid}
\colorlet{deepgreen}{ForestGreen!70!black}
\colorlet{deepblue}{NavyBlue!70!black}
\colorlet{deepred}{RawSienna!70!black}
\colorlet{nicered}{BrickRed!70!white}

\makeatletter
\g@addto@macro\bfseries{\boldmath}
\makeatother

\lstset{
  basicstyle=\small\ttfamily,
  captionpos=b,
  escapeinside={@}{@},
  mathescape=true,
  language=sql,
  frame=leftline,
  framesep=1em,
  backgroundcolor=\color{gray!10},
  morekeywords={with, over, rank, partition}
}

\hypersetup{
  colorlinks,
  citecolor=deepgreen,
  filecolor=nicered,
  linkcolor=deepblue,
  pdfencoding=auto,
  psdextra,
  urlcolor=deepred
}

\usetikzlibrary{positioning,shadings,arrows.meta,fit}

\clearpairofpagestyles

\newcommand\showpage{\itshape\hfill--~\thepage/\pageref*{LastPage}~--\hfill}
\cofoot[\showpage]{\showpage} \cefoot[\showpage]{\showpage}

\setlist[enumerate]{font={\bfseries\color{deepblue}}}
\AtBeginDocument{
  \renewcommand{\labelitemi}{\bfseries\color{deepblue}$\triangleright$}
  \renewcommand{\labelitemii}{\bfseries\color{deepblue}–}
  \renewcommand{\labelitemiii}{\bfseries\color{deepblue}•}
}

\newcommand\separatorBlock{
  \raisebox{-0.2em}{
    \tikz{ \draw[deepblue,ultra thick, line cap=round] (0,0) -- (0,1em); }
  }
}

\newcommand\vertical[1]{
  \rotatebox[origin=c]{270}{\ensuremath{#1}}
}

\mdfsetup{skipabove=1em,skipbelow=0em,linewidth=0pt,rightline=false, topline=false, bottomline=false}


\theoremstyle{definition}

\declaretheoremstyle[
  headfont=\bfseries\sffamily\color{deepgreen}, bodyfont=\normalfont,
]{thmgreenbox}

\declaretheoremstyle[
  headfont=\bfseries\sffamily\color{deepblue}, bodyfont=\normalfont\sffamily\itshape,
]{thmbluebox}

\declaretheoremstyle[
  headfont=\bfseries\sffamily\color{deepblue}, bodyfont=\normalfont,
  mdframed={
    linecolor=deepblue,
    linewidth=2pt,
  },
  numbered=no,
]{thmblueline}

\declaretheoremstyle[
  headfont=\bfseries\sffamily\color{deepred}, bodyfont=\normalfont,
]{thmredbox}

\declaretheoremstyle[
  headfont=\itshape\sffamily\color{deepred}, bodyfont=\normalfont,
  numbered=no,
  qed=\qedsymbol,
]{thmproofbox}

\newcommand\defineMarkerColor[2]{
  \AtBeginEnvironment{#1}{
    \setlist[enumerate]{font={\color{#2}}}
    \renewcommand{\labelitemi}{\color{#2}\small$\triangleright$}
    \renewcommand{\labelitemii}{\color{#2}–}
    \renewcommand{\labelitemiii}{\color{#2}•}
    \renewcommand\emph[1]{{\bfseries\em\color{#2}##1}}
  }
}


\AtBeginDocument{
  \setlist[enumerate]{font={\bfseries\color{deepblue}}}
  \renewcommand{\labelitemi}{\bfseries\color{deepblue}\small$\triangleright$}
  \renewcommand{\labelitemii}{\bfseries\color{deepblue}–}
  \renewcommand{\labelitemiii}{\bfseries\color{deepblue}•}
}


\setlist[enumerate]{font={\color{deepblue}}}
\renewcommand{\labelitemi}{\color{deepblue}\small$\triangleright$}
\renewcommand{\labelitemii}{\color{deepblue}–}
\renewcommand{\labelitemiii}{\color{deepblue}•}

\declaretheorem[style=thmgreenbox, name=Axiom, numbered=no]{axi} \defineMarkerColor{axi}{deepgreen}
\declaretheorem[style=thmgreenbox, name=Definition]{defn} \defineMarkerColor{defn}{deepgreen}
\declaretheorem[style=thmbluebox, name=Example]{exm}      \defineMarkerColor{exm}{deepblue}
\declaretheorem[style=thmbluebox, name=Exercise]{exo}     \defineMarkerColor{exo}{deepblue}
\declaretheorem[style=thmbluebox, name=Question, parent=subsection]{que}     \defineMarkerColor{que}{deepblue}
\declaretheorem[style=thmredbox, name=Proposition]{prop}  \defineMarkerColor{prop}{deepred}
\declaretheorem[style=thmredbox, name=Theorem]{thm}      \defineMarkerColor{thm}{deepred}
\declaretheorem[style=thmredbox, name=Lemma]{lem}         \defineMarkerColor{lem}{deepred}
\declaretheorem[style=thmredbox, name=Corollary]{crlr}   \defineMarkerColor{crlr}{deepred}
\declaretheorem[style=thmblueline, name=Remark]{rmk}    \defineMarkerColor{rmk}{deepblue}
\declaretheorem[style=thmblueline, name=Note]{note}       \defineMarkerColor{note}{deepblue}
\declaretheorem[style=thmproofbox, name=Proof]{replacementproof}
\newenvironment{prv}[1][\proofname]{\vspace{-12pt}%
\begin{replacementproof}}{\end{replacementproof}} \defineMarkerColor{prv}{deepred}
\declaretheorem[style=thmproofbox, name=Proof idea]{replacementideaproof}
\newenvironment{prvid}[1][\proofname]{\vspace{-12pt}%
\begin{replacementideaproof}}{\end{replacementideaproof}} \defineMarkerColor{prvid}{deepred}

\RequirePackage{caption}
\DeclareCaptionLabelFormat{labelformat}{\textbf{#1~#2}\separatorBlock}
\captionsetup{labelformat=labelformat,labelsep=none,textfont=sl}

\DeclareMathSizes{11}{9}{7}{5}

\title{Computational Complexity -- \textit{Homework 1}}
\author{Hugo \textsc{Salou}}

\let\emph\relax
\DeclareTextFontCommand{\emph}{\bfseries\em\color{deepblue}}

\renewcommand{\thefootnote}{\alph{footnote}}

\tikzcdset{arrow style=math font}
\tikzset{
  equiv/.style={-,preaction={draw,double equal sign distance}},
  >=Straight Barb,
}

\usepackage{colortbl}
\usepackage{nicematrix}

\NewDocumentCommand{\problem}{ommm}{
  \csgdef{intro#2}{
    \begin{figure}[H]
      \centering
      \arrayrulecolor{deepblue}
      \setlength\arrayrulewidth{1pt}
      \begin{NiceTabular}{r|p{10cm}}
        \Block{2-1}{\textsc{#1}} & \textbf{Input.} #3\\
                            & \textbf{Output.} #4
      \end{NiceTabular}
      \label{pb:#2}
    \end{figure}
  }

  \csgdef{pb#2}{\hyperref[pb:#2]{\textsc{#1}}}

  \csgdef{recall#2}{
    \begin{figure}[H]
      \centering
      \arrayrulecolor{deepblue}
      \setlength\arrayrulewidth{1pt}
      \begin{NiceTabular}{r|p{10cm}}
        \Block{2-1}{\hyperref[pb:#2]{\textsc{#1}}} & \textbf{Input.} #3\\
                            & \textbf{Output.} #4
      \end{NiceTabular}
    \end{figure}
  }

  \csgdef{show#2}{%
    \ifcsname introduced@#2\endcsname
      \csname recall#2\endcsname
    \else
      \csname intro#2\endcsname
      \expandafter\gdef\csname introduced@#2\endcsname{}%
    \fi
  }
}

\directlua{dofile("../turing.lua")}

\begin{document}
  \begin{center}
    \bfseries
    \sffamily

    {\large\itshape ---\hspace{1em}Homework I\hspace{1em}---}

    {\huge Computational Complexity}

    {\large \itshape Hugo SALOU}
  \end{center}

  \section{Palindromes.}

  \begin{que}
    Let $y, z \in \mathsf{PaddedPal}_k$ and assume for some $i,j \in \llbracket k+1, 2k\rrbracket$, $C_i(y) = C_j(z)$. Show that $y = z$.
  \end{que}

  As $y, z \in \mathsf{PaddedPal}_k$, we can write $y = u 0^k \bar{u}$ and $z = v 0^k \bar{v}$ with $u, v \in \Sigma^k$.
  Consider an accepting execution $r$ for $y$, and $s$ for $z$.

  \begin{figure}[H]
    \centering
    \vspace{-1em}
    \begin{adjustbox}{center}
      \includegraphics[width=0.85\textwidth]{crossing.pdf}
    \end{adjustbox}
    \vspace{-3em}
    \caption{Intersection of two crossing sequences}

    \vspace{1em}

    \footnotesize\textsf{\textbf{Note:} Here, it is shown for the initial tape, but we have\\ a similar situation throughout the execution.}
  \end{figure}

  We can assume that $r$ will accept while its head is before $i$, or $s$ after $j$ (if that is not the case, swap $y$ and $z$).

  We can create an accepting execution $t$ for $w := u 0^\ell \bar{v}$ where $\ell = (i-k)+(2k -j) = k+i-j$: start like $r$, and whenever you try to cross from cell $i$ to cell $i+1$, jump to cell $j+1$ and act like $s$ (we have the same state as the crossing sequences are equal), and whenever you try to cross from $j+1$ to $j$, jump to $i$ act like $s$ again.
  This execution is, indeed, accepting, as either $r$ will accept while its head is before $i$ (as, in this case, the execution $t$ behaves like $r$), either $t$ will accept while its head is after $j$ (as, in this case, $t$ behaves like $s$).
  Thus $w$ is a palindrome, and so $u = v$, and we can finally conclude $y = z$.

  \begin{que}
    Show that for any $y \in \mathsf{PaddedPal}_k$, there is an $i \in \llbracket k+1, 2k\rrbracket$ such that $|C_i(y)| \le T(3k) / k$.
  \end{que}

  As each element of some $C_i(y)$ is a step during the execution of $M$ on a $y$ whose length is $3k$ (and we don't have duplicates when considering different $i$s), we know that
   \[
  \sum_{i=k+1}^{2k} |C_i(y)| \le T(3k)
  ,\]
  and if we take the average, we know at least one element will be less than or equal to the average:
  there is some $i \in \llbracket k+1, 2k\rrbracket$ such that
  \[
  |C_i(y)| \le T(3k) / k
  .\] 

  \begin{que}
    Conclude by observing that for each $y \in \mathsf{PaddedPal}_k$, we can associate a distinct sequence of states of size at most $T(3k) / k$.
  \end{que}

  By question 2, we can associate some $i = i(y)$ such that $|C_{i(y)}(y)| \le T(3k) / k$ to every $y \in \mathsf{PaddedPal}_k$.
  Then, the mapping $y \mapsto C_{i(y)}(y)$ is injective by question 1, thus giving a distinct sequence of states of size at most $T(3k) / k$ to every $y \in \mathsf{PaddedPal}_k$.

  We know that $|\mathsf{PaddedPal}_k| = |\Sigma|^k$ and that the number of state sequences of length at most $\ell$ is less than $\sum_{i=0}^\ell |Q|^i \le |Q|^{\ell+1}$.
  Therefore, with the association made previously, we know that $|\Sigma|^k \le |Q|^{(T(3k) / k) + 1}$, and thus \[
  k \log |\Sigma| \le \left( \frac{T(3k)}{k} + 1 \right) \log |Q|, \quad\quad \text{ so }\quad\quad
  k^2 \underbrace{\left( \frac{\log |\Sigma|}{\log|Q|} - \frac{1}{k} \right)}_{9\alpha\text{\footnote{This is constant, except for the $\frac{1}{k}$ term, which is bounded by $1$, thus getting a constant $\alpha$.}}} \le T(3k)
  .\]
  
  Writing $n = 3k$, we can finally conclude that $T(n) \ge \alpha n^2$ for infinitely many~$n$.

  \section{\texttt{\#\#\#} The Padding Technique \texttt{\#\#\#}.}

  \subsection{Problem 1.}

  \begin{que}
    Show that if $\p = \np$ then  $\exptime = \nexptime$.\footnote{\AP $\intro*\exptime = \bigcup_{k \in \mathds{N}} \dtime{2^{n^k}}$ and $\intro*\nexptime = \bigcup_{k \in \mathds{N}} \ntime{2^{n^k}}$.}
  \end{que}

  Let us assume $\p = \np$.
  We already know $\exptime \subseteq \nexptime$, so let us show the converse.
  Consider $L \in \nexptime$, and $M$ a non-deterministic Turing machine recognizing $M$ and that runs in time $C \cdot 2^{n^k}$ for some integer $k$ and some constant $C$.
  Then, consider the following language on $\{\texttt{0}, \texttt{1}, \texttt{\#}\}$:\footnote{We can always encode such an alphabet with the binary alphabet using twice as many cells.}
  \[
    L' := \mleft\{\,w \texttt{\#}^{C \cdot 2^{{|w|}^k}} \;\middle|\; w \in L \,\mright\}.
  \]
  We use the \texttt{\#}s as fuel for $M$ to run in polynomial time in the length of
  \[
    \bar{w} := w \texttt{\#}^{C \cdot 2^{{|w|}^k}}
  .\] 

  Define $M'$ to be the non-deterministic machine that follows exactly $M$'s behaviour except that it consider \texttt{\#} as a blank character \blank\ (so it'll only read the~"$w$ part" of $\bar{w}$).
  Therefore, the language recognized by $M'$ is exactly $L'$, and its runtime is in polynomial time (as the "extra length" of $\bar{w}$ allows the execution of $M$ on $w$ to happen in polynomial time in $|\bar{w}|$).
  Thus,
  \[
  L' \in \np = \p
  .\] 

  As $L' \in \p$, there exists some deterministic poly-time Turing machine $N$ recognizing $L'$.
  Let us deduce a exponential-time deterministic Turing machine $N'$ recognizing $L$:
  start by writing $w$ on $N$'s input tape, compute $C \cdot 2^{n^k}$ (which is time-constructible by a construction similar to Tutorial 3, Exercise 2) in unary (using the \texttt{\#} character), adding that on the input tape after $w$, and finally start the execution of~$N$.

  The input tape of $N$ contains $\bar{w}$ and so we have that $N$ will test whether $\bar{w} \in L'$, \textit{i.e.}\ whether $w \in L$.
  Furthermore, $N$ runs in exponential time and is deterministic.
  Therefore, $L \in \exptime$.

  We can conclude that $\nexptime \subseteq \exptime$.

  \subsection{Problem 2.}

  \begin{que}
    Suppose $\nspace{n^4} \subseteq \nspace{n^3}$.
    Let $M$ be a Turing machine running in $\mathrm{O}(n^5)$ space.
    Show that there is a Turing machine $N$ accepting the language $L(M)$ and running in $\mathrm{O}(n^4)$ space.
    Consequently, $\nspace{n^5} \subseteq \nspace{n^4}$.
  \end{que}

  Let $M$ be a non-deterministic Turing machine that runs in $\mathrm{O}(n^5)$ space.
  We define the language
  \[
  A := \mleft\{\,w \texttt{\#}^{\lceil n^{5 / 4} \rceil - n} \;\middle|\; w \in L(M), n = |w|\,\mright\} 
  .\]
  In the following, we will write $\bar{w}$ the word associated with $w \in L(M)$ in language $A$, and $m = \mathrm{O}(n^{5 / 4})$ its length.

  We have that $A \in \nspace{m^4}$ as we can simply ignore the \texttt{\#} and run $M$ on the input tape.
  The extra \texttt{\#}s allows $M$ to use more space, and $m^4 = \mathrm{O}(n^5)$.

  Thus we have that $A \in \nspace{m^3}$ by the assumption that $\nspace{m^4} \subseteq \nspace{m^3}$.
  Also, the computation of $\lceil n^{5 / 4} \rceil - n$ can be done in space $\mathrm{O}(n)$ (non-output tapes only), and thus, so is the computation of $\bar{w}$.
  Therefore, considering $N$ the non-deterministic Turing machine that:
  \begin{enumerate}
    \item computes $\bar{w}$ in space $\mathrm{O}(n)$ ;
    \item simulates $P$ on $\bar{w}$ in space $m^3 = \mathrm{O}(n^{15 / 4}) = \mathrm{O}(n^4)$,
  \end{enumerate}
  where $P$ is a non-deterministic Turing machine that recognizes $A'$ in space $\mathrm{O}(m^3)$.
  We have that $N$ runs in space $\mathrm{O}(n^4)$ and its language is exactly $L(M)$.
  So, $L(M) \in \nspace{n^4}$.

  We can conclude that \[
    \nspace{n^5} \subseteq \nspace{n^4}
  .\]

  \begin{que}
    Using Savitch’s theorem and the previous result, show that $\nspace{n^3} \subsetneq \nspace{n^4}$ has to be true.
  \end{que}

  By induction on $k \in \mathds{N}$, and with a proof very similar to the one in the previous question, we can show that if $\nspace{n^4} \subseteq \nspace{n^3}$, then \[
    \nspace{n^{k+4}} \subseteq \nspace{n^{k+3}} \subseteq \cdots \subseteq \nspace{n^3}
  ,\]
  and the reverse inclusions are clearly true, we have the equality of all these classes of languages.
  Also, by Savitch's theorem (for the right inclusion), we have that 
  \[
    \nspace{n^3} \subseteq \dspace{n^6} \subseteq \nspace{n^6} = \nspace{n^3}
  ,\] 
  thus $\dspace{n^6} = \nspace{n^6}$.
  Similarly, we have that $\dspace{n^8} = \nspace{n^8}$ as
  \[
    \nspace{n^4} \subseteq \dspace{n^8} \subseteq \nspace{n^8} = \nspace{n^4}
  .\]
  Thus $\dspace{n^6} = \dspace{n^8}$ which is absurd as $n^6 = \mathrm{o}(n^8)$ (and $n^8$ is space-constructible) using the space hierarchy.

  \begin{que}
    Can you now show that $\nspace{n^s} \subsetneq \nspace{n^t}$ for all $0 \le s < t$, $s, t \in \mathds{N}$?
  \end{que}

  Assume, by contradiction, that $\nspace{n^s} = \nspace{n^t}$ for some integers $s < t$.
  As
  \[
    \nspace{n^t} = \nspace{n^s} \subseteq \nspace{n^{s+1}} \subseteq \cdots \subseteq \nspace{n^t}
  \]
  form a chain of equalities, we have that
  \[
    \nspace{n^{s+1}} = \nspace{n^s}
  .\]
  Very similarly to the previous questions, we can show that 
  \[
    A := \mleft\{\,w \texttt{\#}^{n^{(s + 2)/(s + 1)} - n} \;\middle|\; w \in L(M), n = |w|\,\mright\} 
  \] is in $\nspace{m^{s+1}}$ where $M$ is a non-deterministic Turing machine that uses at most $\mathrm{O}(n^{s+1})$ space.
  Thus, we can show that $L(M) \in \nspace{n^{s+1}}$ as we can simulate $M$ on $\bar{w}$ using
  \[
    \mathrm{O}(m^{s+1}) = \mathrm{O}\big((n^{(s+1) / (s+2)})^{s+1}\big) = \mathrm{O}(n^{s+1})
  \] 
  space.
  Doing this for $t = s+1, s+2, s+3, \ldots$, we obtain that 
  \begin{equation}
    \nspace{n^s} = \nspace{n^{s+1}} = \cdots = \nspace{n^{s + k}} = \cdots,
    \tag{$*$}
    \label{eq:pb2-contra}
  \end{equation}
  and thus $\dspace{n^{2s}} = \dspace{n^{2(s+1)}}$ which contradicts the space hierarchy ($n^{2(s+1)}$ is space-contractible and $n^{2s} = \mathrm{o}(n^{2(s+1)})$).

  We can conclude that
  \[
    \nspace{n^s} \subsetneq \nspace{n^t}
  ,\] 
  for all integers $s < t$.

  \begin{note}
    After (\ref{eq:pb2-contra}), we could conclude that
    \[
      \pspace = \bigcup_{k \in \mathds{N}} \dspace{n^k} \subseteq \bigcup_{k \in \mathds{N}} \nspace{n^k} = \nspace{n^s}
    ,\] 
    and thus we get the contradiction $\pspace = \dspace{n^{2s}}$.
  \end{note}
\end{document}
