\documentclass[./main]{subfiles}

\begin{document}
  \selectlanguage{french}
  \setquotestyle{french}

  \chapter{Ordres partiels et treillis.}

  \section{Ordres partiels.}
  \begin{defn}
    Un \textit{ordre partiel} (ou \textit{poset} en anglais) est une paire $(P,{\le})$ où $\le$ est une relation binaire sur $P$ telle que 
    \begin{itemize}
      \item (\textit{reflexivité}) $\forall  x \in P, x \le x$ ;
      \item (\textit{transitivité}) $\forall x, y \in P, x \le y \implies y \le z \implies x \le z$ ;
      \item (\textit{antisymétrie}) $\forall x, y \in P, x \le y \implies y \le x \implies x = y$.
    \end{itemize}

    Un préodre est une relation binaire reflexive et transitive.
  \end{defn}

  \begin{exm}
    On donne quelques exemples de poset :
    \begin{enumerate}
      \item $(\wp(X), \subseteq)$, l'inclusion dans les parties de $X$
      \item $(\Omega X, \subseteq)$, l'inclusion dans les ouverts de $X$
      \item $(\Sigma^\star, \subseteq)$, la relation préfixe dans les mots sur $\Sigma$
    \end{enumerate}
    Attention, dans les trois exemples, il existe deux éléments $u, v$ où
    \[
      u \not\le v
      \quad\text{et}\quad
      v \not\le u
    .\] 
  \end{exm}

  \begin{defn}[Dual]
    Soit $(P, \le)$ un poset.
    Le \textit{dual} de $P$ est $(P, \le)^\mathrm{op} := (P, \ge)$ où
    \[
    a \ge b \iff b \le a
    .\] 
  \end{defn}

  \begin{defn}[Fonction (anti)monotone]
    Soit $(P, \le_P)$ et $(L, \le_L)$ deux posets.
    Une fonction $f : P \to L$ est \textit{monotone} si pour tout $a, b \in P$ on a 
    \[
    a \le_P b \implies f(a) \le_L f(b)
    .\]

    On dit que $f : (P, \le) \to (L, \le)$ est \textit{antimonotone} si $f : {(P, \flip{\le_P})} = (P, \le_P)^\mathrm{op} \to (L, \le_L)$ est monotone, autrement dit pour tout $a, b \in P$ on a  
    \[
      a \le_P b \implies f(a) \ge_L f(b)
    .\]
  \end{defn}

  \section{Treillis complet.}

  \begin{defn}
    Soit $(A, {\le})$ un poset et $S \subseteq A$.
    \begin{itemize}
      \item Un \textit{upper bound} de $S$ est un élément $a \in A$ tel que $\forall s \in S$, $s \le a$.
      \item Un \textit{least upper bound} (\textit{lub}, \textit{join} ou \textit{sup}) de $S$ est un upper bound $a \in A$ de $S$ tel que, pour tout upper bound $b \in A$ de $S$, on a $a \le b$.
    \end{itemize}

    Par dualité, on a les définitions suivantes.
    \begin{itemize}
      \item Un élément $a \in A$ est un \textit{lower bound} de $S$ ssi $a$ est un upper bound de $S$ dans $A^\mathrm{op}$.
      \item Un élément $a \in A$ est un \textit{greatest lower bound} (\textit{glb}, \textit{meet}, \textit{inf}) de $S$ ssi $a$ est un least upper bound de $S$ dans $A^\mathrm{op}$.
    \end{itemize}

    On note $\bigvee S$ le least upper bound de $S$.
    On note $\bigwedge S$ le greatest lower bound de $S$.
  \end{defn}

  \begin{exm}
    Soit $S \subseteq \wp(X)$ alors le least upper bound de $S$ dans $(\wp(X), \subseteq)$ est $\bigcup S \in \wp(X)$.
    Le greatest lower bound de $S$ dans $(\wp(X), \subseteq)$ est $\bigcap S \in \wp(X)$.
  \end{exm}

  \begin{exm}
    Soit $S \subseteq \Omega X$ alors le least upper bound dans ${(\Omega X, \subseteq)}$ est $\bigcup S \in \Omega X$.
    Le greatest lower bound dans $(\Omega X, \subseteq)$ n'est pas évident. En effet,
    \[
    \{\mathsf{ext}(\mathsf{a}^n)  \mid n \in \mathds{N}\}  \subseteq \Omega \Sigma^\omega
    ,\]
    mais $\bigcap_{n \in \mathds{N}} \mathsf{ext}(\mathsf{a}^n) = \{\mathsf{a}^\omega\} \not\in \Omega\Sigma^\omega$.
  \end{exm}

  \begin{exm}
    Dans $(\Sigma^\star, \subseteq)$ (la relation "préfixe de"), une partie $S \subseteq \Sigma^\star$ n'a pas forcément de sup.
  \end{exm}

  \begin{defn}
    Un poset $(L, \le)$ est un \textit{treillis complet} si
    \begin{itemize}
      \item tout $S \subseteq L$ a un sup $\bigvee S \in L$ ;
      \item tout $S \subseteq L$ a un inf $\bigwedge S \in L$.
    \end{itemize}
  \end{defn}

  \begin{rmk}[Unicité du lub/glb]
    Par antisymétrie, si $a$ et $b$ sont deux least upper bound (ou greatest lower bound) alors $a = b$.
  \end{rmk}

  En conséquence on a que tout treillis complet a 
  \begin{itemize}
    \item un plus petit élément $\bot := \bigvee \emptyset \in L$ ;
    \item un plus grand élément $\top := \bigwedge \emptyset \in L$.
  \end{itemize}

  \begin{rmk}[Non-exemple]
    Le poset $(\Sigma^\star, \subseteq)$ (avec la relation "préfixe de") n'est \textit{\textbf{pas}} un treillis complet, car il n'a pas de plus grand élément $\top$.
  \end{rmk}

  \begin{exm}
    Le poset $(\wp(X), \subseteq)$ (avec la relation d'inclusion ensembliste) est un treillis complet.
  \end{exm}

  \begin{lem}
    Les conditions suivantes sont équivalentes pour un poset $(L, \le)$ :
    \begin{enumerate}
      \item $(L, \le)$ est un treillis complet ;
      \item tout $S \subseteq L$ a un sup $\bigvee S \in L$ ;
      \item tout $S \subseteq L$ a un inf $\bigwedge S \in L$ ;
    \end{enumerate}
  \end{lem}
  \begin{prv}
    Pour montrer l'implication "2. $\implies$ 3.", on peut définir 
    \[
    \forall S \subseteq L, \quad \bigwedge S := \bigvee \{b  \mid \forall s \in S, b \le s\}
    ,\]
    et montrer que c'est bien un inf.
  \end{prv}

  \begin{exm}
    En revenant sur $(\Omega X, \subseteq)$, c'est un treillis complet dont l'inf de $S \subseteq \Omega X$ est 
    \[
      \bigwedge S = \bigcup \{V \in \Omega X  \mid V \subseteq {\textstyle \bigcap S} \} 
    .\]
    Il s'agit de $\widering {\bigcap S}$ qui est l'\textit{intérieur} de $\bigcap S$.

    Par exemple, dans $(\Omega \Sigma^\omega, \subseteq)$, on a 
    \[
      \bigwedge \{\mathsf{ext}(\mathsf{a}^n)  \mid  n\in \mathds{N}\} = \widering{\{\mathsf{a}^\omega\} } = \emptyset
    .\] 
  \end{exm}

  \section{Opérateur de clôture.}

  \begin{defn}
    Soit $(A, \le)$ un poset.
    Un \textit{opérateur de clôture} suer $(A, \le)$ est une fonction
    \[
    c : A \to A
    \] 
    telle que
    \begin{itemize}
      \item $c$ est monotone ;
      \item $c$ est "\textit{expansive}" : pour tout $a \in A$, $a \le c(a)$ ;
      \item $c$ est \textit{idempotent} : $c(c(a)) = c(a)$ pour tout $a \in A$.
    \end{itemize}
  \end{defn}

  \begin{exm}
    Soit $(X, \Omega X)$ un espace topologique.
    Alors
    \[
    \wp(X) \ni A \mapsto \bar{A} \in \wp(X)
    \]
    est un opérateur de clôture sur $(\wp(X), \subseteq)$.
  \end{exm}

  \begin{lem}
    Soit $c$ un opérateur de clôture sur $(L, \le)$.
    On pose \[
      L^c := \{a \in L  \mid \underbrace{c(a) = a}_{a \in \operatorname{im} c}\} 
    .\]
    Si $(L, \le )$ est un treillis complet alors $(L^c, \le)$ est un treillis complet avec \[
    \forall S \subseteq L^c, \quad\quad \bigwedge^{L^c} S = \bigwedge^L S
    .\] 
  \end{lem}

  \begin{exm}
    Pour $\overline{(-)} : \wp(X) \to \wp(X)$ où $(X, \Omega X)$ est un espace topologique, on a
    \[
      \big(\wp(X)\big)^{\overline{(-)}} = \{F \in \wp(X)  \mid F \text{ fermé}\}
    .\]
    Dans ce treillis complet :
    \[
      \bigwedge \mathcal{F} = \bigcap \mathcal{F} \quad\quad \text{ et } \quad\quad \bigvee \mathcal{F} = \overline{\bigcup \mathcal{F}} 
    ,\] 
    où $\mathcal{F}$ est un ensemble de fermés.
  \end{exm}

  \section{Connexion de Galois.}

  \begin{defn}
    Considérons deux posets $(A, \le_A)$ et $(B, \le_B)$.
    Une \textit{connexion de Galois} $g \dashv f : A \to B$ est une paire $(f, g)$ de fonctions :
    \[
    f : B \to A \quad\quad \text{ et }\quad\quad g : A \to B
    \] 
    telle que \[
    g(a) \le_B b \iff a \le_A f(b)
    .\] 
  \end{defn}

  \begin{exm}
    Soit $f : X \to Y$ une fonction.
    On possède deux "lifts" de $f$ sur les powersets :
    \begin{itemize}
      \item le lift covariant $
        f_! : \begin{array}{rcl}
          \wp(X) & \longrightarrow & \wp(Y)\\
          A & \longmapsto & \{f(a)  \mid a \in A\}
        \end{array}$ ;
      \item le lift contravariant $
        f^\bullet : \begin{array}{rcl}
          \wp(Y) & \longrightarrow & \wp(X)\\
          B & \longmapsto & \{x \in X  \mid f(x) \in B\}
        \end{array}$.\footnote{On note habituellement $f^*$ et non $f^\bullet$, mais vu qu'on utilise souvent "$*$" dans le cours, on change de notation.}
    \end{itemize}

    On a que $f_! \dashv f^\bullet$.
    En effet, pour tout  $A \in \wp(X)$ et $B \in \wp(Y)$,
    \begin{align*}
      f_!(A) \subseteq B &\iff \forall x \in X, (x \in A \implies f(x) \in B)\\
                         &\iff A \subseteq f^\bullet(B)
    .\end{align*}
  \end{exm}

  \begin{exm}
    Soit $\Sigma$ un alphabet.
    On a, d'une part,
    \begin{align*}
      \mathrm{Pref}: \wp(\Sigma^\omega) &\longrightarrow \wp(\Sigma^\star) \\
      A &\longmapsto \underbrace{\{\hat{\sigma} \in \Sigma^\star  \mid \exists \sigma \in A, \hat{\sigma} \subseteq \sigma\}} _{\bigcup_{\sigma \in A} \mathrm{Pref}(\sigma) }
    .\end{align*}
    D'autre part, on a 
    \begin{align*}
      \mathrm{cl}: \wp(\Sigma^\star) &\longrightarrow \wp(\Sigma^\omega) \\
      W &\longmapsto \{\sigma \in \Sigma^\omega  \mid \mathrm{Pref}(\sigma) \subseteq W\} 
    .\end{align*}
    Attention, ce n'est pas le $\mathrm{cl}$ vu en TD.
    On a que
    \[
    \mathrm{Pref}(-) \dashv \mathrm{cl}(-)
    .\] 
  \end{exm}

  \begin{lem}
    \begin{itemize}
      \item Si $g \dashv f$ et $g' \dashv f$ alors $g = g'$.
      \item Si $g \dashv f$ et $g \dashv f'$ alors $f = f'$.
      \item Si $g \dashv f$ alors $g$ et $f$ sont monotones.
    \end{itemize}
  \end{lem}
  \begin{prv}
    Vu en TD.
  \end{prv}

  Dans $g \dashv f$, on dit que 
  \begin{itemize}
    \item $g$ est un \textit{adjoint à gauche} de $f$ ;
    \item $f$ est un \textit{adjoint à droite} de $g$.
  \end{itemize}

  \begin{lem}
    Si $g \dashv f : (A, \le_A) \to (B, \le_B)$ alors 
    \[
    f \circ g : A \overset g \longrightarrow B \overset f \longrightarrow A
    \] 
    est un opérateur de clôture sur $(A, \le_A)$.\footnote{Attention à ne pas se tromper sur le sens de la composition !}
  \end{lem}
  \begin{prv}
    Vu en TD.
  \end{prv}

  \begin{exm}
    Pour $\mathrm{Pref}(-) \dashv \mathrm{cl}(-) : \wp(\Sigma^\omega) \to \wp(\Sigma^\star)$,
    le lemme précédent nous donne l'opérateur de clôture 
    \begin{align*}
      \mathrm{cl} \circ \mathrm{Pref}: \wp(\Sigma^\omega) &\longrightarrow \wp(\Sigma^\omega) \\
      A &\longmapsto \{\sigma \in \Sigma^\omega  \mid \mathrm{Pref}(\sigma) \subseteq \mathrm{Pref}(A)\} 
    \end{align*}
    (c'est le $\mathrm{cl}(-)$ vu en TD)
    est la clôture topologique pour $(\Sigma^\omega, \Omega \Sigma^\omega)$.
  \end{exm}

  \begin{rmk}
    En particulier, $A \subseteq \Sigma^\omega$ est un fermé si et seulement s'il existe un arbre $T \subseteq \Sigma^\star$ tel que \[
    A = \{\pi \in \Sigma^\omega  \mid \pi \text{ chemin infini dans } T\} 
    .\]
    On a que $\mathrm{cl} \circ \mathrm{Pref}(A)$ qui est un arbre sur $\Sigma$.
  \end{rmk}

  \begin{crlr}
    \begin{itemize}
      \item Une propriété $P \subseteq (\mathbf{2}^\mathrm{AP})^\omega$ est de sûreté si et seulement si on a $P = \mathrm{cl}(\mathrm{Pref}(P))$.
      \item Une propriété $P \subseteq (\mathbf{2}^\mathrm{AP})^\omega$ est de vivacité si et seulement si on a $(\mathbf{2}^\mathrm{AP})^\omega = \mathrm{cl}(\mathrm{Pref}(P))$.
    \end{itemize} 
  \end{crlr}
  \begin{prv}
    (Déjà) vu en TD.
    Ceci correspond exactement au fait que 
    \begin{itemize}
      \item $P$ est de sûreté ssi $P$ est fermé dans $(\Sigma^\omega, \Omega \Sigma^\omega)$ ;
      \item $P$ est de vivacité ssi $P$ est dense dans $(\Sigma^\omega, \Omega \Sigma^\omega)$ ;
      \item $\mathrm{cl} \circ \mathrm{Pref}$ est exactement $\overline{(-)}$ dans $(\Sigma^\omega, \Omega \Sigma^\omega)$.
    \end{itemize}
  \end{prv}

  \begin{prop}
    Une propriété $P \subseteq (\mathbf{2}^\mathrm{AP})^\omega$ est de vivacité si et seulement si $\mathrm{Pref}(P) = (\mathbf{2}^\mathrm{AP})^\star$.
  \end{prop}
  \begin{prv}
    En effet, par adjonction (connexion de Galois),
    on a 
    \[
      (\mathbf{2}^\mathrm{AP})^\star = \mathrm{Pref}((\mathbf{2}^\mathrm{AP})^\omega) \subseteq \mathrm{Pref}(P)
      \iff (\mathbf{2}^\mathrm{AP})^\omega \subseteq \mathrm{cl}(\mathrm{Pref}(P))
    .\]
  \end{prv}

  \subsubsection{Quelques propriétés des connexions de Galois.}

  \begin{lem}
    Soit $g \dashv f : A \to B$ une connexion de Galois.
    \begin{enumerate}
      \item pour tout $S \subseteq A$ tel que $\bigvee S \in A$ alors $g(\bigvee S) = \bigvee g_!(S)$ ;
      \item pour tout $S \subseteq B$ tel que $\bigwedge S \in B$ alors $f(\bigwedge S) = \bigwedge f_!(S)$.
    \end{enumerate}
  \end{lem}

  \begin{rmk}
    Dans le lemme précédent, il est important de remarquer que l'on a une implication "cachée" :
    $\bigvee S$ existe dans $A$ implique $\bigvee g_!(S)$ existe dans  $B$ (et idem pour $\bigwedge$ et $f$).
  \end{rmk}

  \begin{lem}
    Soient $(A, \le_A)$ et $(B, \le_B)$ deux treillis complets.
    \begin{enumerate}
      \item Si $g : A \to B$ préserve les sups (\textit{i.e.} $g(\bigvee S) = \bigvee g_!(S)$) alors il existe une fonction $f : B \to A$ telle que $g \dashv f$.
        Cette fonction est :
        \[
        f(b) := \bigvee \{a \in A  \mid g(a) \le_B b\} 
        .\]
      \item Si $f : B \to A$ préserve les infs alors il existe une fonction $g : A \to B$ telle que $g \dashv f$.
        Cette fonction est :
        \[
        g(a) := \bigwedge \{b \in B  \mid a \le_A f(b)\} 
        .\] 
    \end{enumerate}
  \end{lem}

  \begin{exm}[Algèbres de Heyting complètes]
    Soit $(L, \le)$ un treillis complet.
    Soit $a \in L$.
    On a une fonction 
     \begin{align*}
      - \wedge a: L &\longrightarrow L \\
      b &\longmapsto b \wedge a = \bigwedge \{a, b\} 
    .\end{align*}

    On dit que $(L, \le)$ est une \textit{algèbre de Heyting complète} si, pour tout $a \in A$, la fonction $- \wedge a$ a un adjoint à gauche.
    Si cet adjoint existe, on le note $a \Rightarrow -$.
    Ceci nous donne que 
    \[
    \forall a, b, c \in L, \quad\quad b \wedge a \le c \iff b \le a \Rightarrow c
    .\] 

    On a l'équivalence entre :
    \begin{itemize}
      \item $(L, \le)$ est une algèbre de Heyting complète ;
      \item pour tout $a \in L$, $- \wedge a : L \to L$ préserve les sups, autrement dit pour tout $S \subseteq L$, 
        \[
          \big(\bigvee S\big) \wedge a = \bigvee \{s \wedge a  \mid s \in S\} 
        .\] 
        C'est une sorte de distributivité.
    \end{itemize}
    Dans ce cas, on a que 
    \[
    a \Rightarrow c = \bigvee \{b  \mid b \wedge a \le c\}
    .\] 
  \end{exm}

\end{document}
