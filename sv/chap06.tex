\documentclass[./main]{subfiles}

\begin{document}
  \selectlanguage{french}
  \setquotestyle{french}

  \chapter{Logique temporelle linéaire.}

  \section{La logique $\mathsf{LML}$.}

  \begin{rmk}[Idée]
    La signification de $\mathsf{LML}$ est \textit{linear-time modal logic}.
    L'idée est de définir une logique pour une propriété LT $P \subseteq (\mathbf{2}^\mathrm{AP})^\omega$ telle que, pour $\mathrm{AP}$ fini, les formules correspondent aux clopens.
  \end{rmk}

  Soit $\mathrm{AP}$ un ensemble de proposition atomiques.
  
  \begin{defn}
    Les formules de $\mathsf{LML}$ sont 
    \begin{align*}
      \phi, \psi ::={}&{} \mathsf{a} & \mathsf{a} \in \mathrm{AP}\\
      {}\mid{}&{} \mathsf{True} & \text{ (parfois notée $\top$)} \\
      {}\mid{}&{} \mathsf{False} & \text{ (parfois notée $\bot$)} \\
      {}\mid{}&{} \phi \land \psi \\
      {}\mid{}&{} \phi \lor \psi \\
      {}\mid{}&{} \lnot \phi \\
      {}\mid{}&{} \nxt \phi \\
    .\end{align*}
    La \textit{modalité} $\nxt$ est appelée \textit{later} ou \textit{next}.
  \end{defn}

  \begin{defn}
    L'\textit{interprétation} $\llbracket \phi\rrbracket \subseteq (\mathbf{2}^\mathrm{AP})^\omega$ est définie par :
    \begin{itemize}
      \item $\llbracket \mathsf{a}\rrbracket := \{\sigma  \mid \mathsf{a} \in \sigma(0)\}$ ;
      \item $\llbracket \mathsf{True}\rrbracket := (\mathbf{2}^\mathrm{AP})^\omega$ ;
      \item $\llbracket \phi \land \psi\rrbracket := \llbracket \varphi\rrbracket \cap \llbracket \psi\rrbracket$ ;
      \item $\llbracket \mathsf{False}\rrbracket := \emptyset$ ;
      \item $\llbracket \phi \lor \psi\rrbracket := \llbracket \varphi\rrbracket \cup \llbracket \psi\rrbracket$ ;
      \item $\llbracket \lnot \phi\rrbracket := (\mathbf{2}^\mathrm{AP})^\omega  \setminus\llbracket \phi\rrbracket$ ;
      \item $\llbracket \nxt \phi \rrbracket := \{\sigma  \mid \sigma \upharpoonright 1 \in \llbracket \phi\rrbracket \}$ où, pour $\sigma \in (\mathbf{2}^\mathrm{AP})^\omega$, on note
        \begin{align*}
          \sigma \upharpoonright i : \mathds{N} &\longrightarrow \mathbf{2}^\mathrm{AP} \\
          k &\longmapsto \sigma \upharpoonright \sigma(k+i)
        ,\end{align*}
        (c'est un décalage d'indices).
    \end{itemize}
  \end{defn}

  \begin{exm}
    Quelques exemples de mots tels que $\sigma \in \llbracket \mathsf{a} \lor \nxt \mathsf{b}\rrbracket$ :
    \begin{itemize}
      \item $\sigma = \{\mathsf{a}\} \emptyset^\omega$,
      \item $\sigma = \{\mathsf{b}\} \{\mathsf{a}, \mathsf{b}\}^\omega$,
      \item $\sigma = \emptyset \{\mathsf{a}, \mathsf{b}\} \emptyset^\omega$.
    \end{itemize}
  \end{exm}

  \begin{prop}
    Pour $\phi$ une formule de $\mathsf{LML}$, on a que $\llbracket\phi \rrbracket$ est clopen dans $(\mathbf{2}^\mathrm{AP})^\omega$.
  \end{prop}
  \begin{prv}
    Par induction sur $\phi$, on a les cas suivants.
    \begin{itemize}
      \item On a que $\llbracket \mathsf{a}\rrbracket = \bigcup_{\mathsf{a} \in A \subseteq \mathrm{AP}} \mathsf{ext}(A)$\footnote{On rappelle que $\mathsf{ext}(A) = \{\sigma  \mid \sigma(0) = A \subseteq \mathrm{AP}\}$.} est un ouvert.
        De plus, on a que $(\mathbf{2}^\mathrm{AP})^\omega \setminus \llbracket \mathsf{a}\rrbracket = \bigcup_{a \not\in B \subseteq \mathrm{AP}} \mathsf{ext}(B)$ est un ouvert.
      \item On a que
        \begin{itemize}
          \item $\llbracket \nxt \phi\rrbracket = \bigcup_{u \in W, A \subseteq \mathrm{AP}} \mathsf{ext}(Au)$
          \item $(\mathbf{2}^\mathrm{AP})^\omega \setminus \llbracket \nxt \phi\rrbracket = \bigcup_{v \in V, A \subseteq \mathrm{AP}} \mathsf{ext}(Av)$
        \end{itemize}
        où par hypothèse d'induction, il existe $V, W \subseteq \Sigma^\star$ tels que~$\llbracket \phi\rrbracket = \mathsf{ext}(W)$ et $(\mathbf{2}^\mathrm{AP})^\omega \setminus \llbracket \phi\rrbracket = \mathsf{ext}(V)$.
      \item De même pour les autres cas.
    \end{itemize}
  \end{prv}

  \begin{prop}
    Si $\mathrm{AP}$ est \textit{\textbf{fini}} et $P \subseteq (\mathbf{2}^\mathrm{AP})^\omega$ est clopen alors il existe $\phi$ une formule $\mathsf{LML}$ telle que $P = \llbracket \phi\rrbracket$.
  \end{prop}
  \begin{prv}
    On a que $P = \mathsf{ext}(W)$ où $W \subseteq (\mathbf{2}^\mathrm{AP})^\star$ est \textit{\textbf{fini}}.
    On a montrer par récurrence sur la taille du mot $u$ que :
    \[
    \forall u \in (\mathbf{2}^\mathrm{AP})^\star, \quad \exists \phi_u \text{ une formule } \mathsf{LML}, \quad
    \llbracket \phi_u\rrbracket  = \mathsf{ext}(u)
    .\]

    \begin{itemize}
      \item \textit{Cas de base.}
        Soit $A \subseteq \mathrm{AP}$, on peut prendre 
        \[
        \phi_A := \big(\bigwedge_{\mathsf{a} \in A} \mathsf{a}\big) \land \big(\bigwedge_{\mathsf{b} \not\in A} \lnot \mathsf{b}\big)
        .\]
      \item \textit{Récurrence.}
        Soit $u = A v$ où $A \subseteq \mathrm{AP}$ et $v \in (\mathbf{2}^\mathrm{AP})^\star$.
        On peut poser 
        \[
        \phi_{A v} := \phi_A \land \nxt \phi_v
        .\]
    \end{itemize}
    On peut aussi faire le cas de base pour $\varepsilon$, en posant $\phi_{\varepsilon} := \mathsf{True}$.
  \end{prv}

  \begin{crlr}
    Si $\mathrm{AP}$ est \textit{\textbf{fini}} et $P \subseteq (\mathbf{2}^\mathrm{AP})^\omega$ alors 
    \[
    P \text{ clopen} \iff \text{ il existe $\phi$ telle que $\llbracket \phi\rrbracket  = P$}
    .\]
    \qed
  \end{crlr}

  \subsection{Équivalences logiques.}

  \begin{defn}
    On note $\phi \equiv \psi$ lorsque $\llbracket \phi\rrbracket = \llbracket \psi\rrbracket$.
    On dit que $\phi$ et $\psi$ sont \textit{(logiquement) équivalentes}.
  \end{defn}

  On a les équivalences suivantes :
  \begin{itemize}
    \item $\phi \equiv \phi \land \phi$
    \item $\phi \equiv \mathsf{True} \land \phi$
    \item $\mathsf{True} \equiv \phi \lor \lnot \phi$
    \item $\mathsf{False} \equiv \phi \land \lnot \phi$
    \item $\phi \equiv \lnot \lnot \phi$
    \item $\nxt(\phi \lor \psi) \equiv \nxt \phi \lor \nxt \psi$
    \item $\nxt \mathsf{False} \equiv \mathsf{False}$
    \item $\nxt(\phi \land \psi) \equiv \nxt \phi \land \nxt \psi$
    \item $\nxt \mathsf{True} \equiv \mathsf{True}$
  \end{itemize}
  d'autres équivalences sont possibles (\textit{c.f.} figure 6 des notes de cours).

  \subsection{\textit{Homework} : Dualité de Stone.}

  L'idée est de motiver le DM, et de donner quelques bases sur ce que l'on va montrer.

  On se place dans le cas où $\mathrm{AP}$ est un ensemble fini.
  Considérons un mot $\sigma \in (\mathbf{2}^\mathrm{AP})^\omega$, on pose 
  \[
    \mathcal{F}_\sigma := \{[\phi]_\equiv \mid \sigma \in \llbracket \phi\rrbracket \} 
  ,\] 
  où $[\phi]_\equiv$ est la classe d'équivalence de $\equiv$.
  Par les résultats précédents, on a que 
  \[
  \mathcal{F}_\sigma \cong \{C  \mid \sigma \in C \text{ clopen}\} 
  .\]
  On a $\sigma \neq \beta$ implique $\mathcal{F}_\sigma \neq \mathcal{F}_\beta$.
  De plus, on a les propriétés suivantes :
  \begin{enumerate}
    \item si $C \in \mathcal{F}_\sigma$ et $C \subseteq D$ clopen alors $D \in \mathcal{F}_\sigma$ ;
    \item si $C, D \in \mathcal{F}_\sigma$ alors $C \cap D \in \mathcal{F}_\sigma$ ;
    \item $(\mathbf{2}^\mathrm{AP})^\omega \in \mathcal{F}_\sigma$ ;
    \item si $C, D$ sont clopen tels que $C \cup D \in \mathcal{F}_\sigma$ alors $C \in \mathcal{F}_\sigma$ ou $D \in \mathcal{F}_\sigma$ ;
    \item $\emptyset \in \mathcal{F}_\sigma$.
  \end{enumerate}

  Ces cinq propriétés caractérisent totalement les mots infinis, comme le montre le théorème suivant.

  \begin{thm}
    Si $\mathcal{F}$ est un ensemble de clopens dans $(\mathbf{2}^\mathrm{AP})^\omega$ vérifiant les cinq propriétés, alors il existe $\sigma \in (\mathbf{2}^\mathrm{AP})^\omega$ tel que $\mathcal{F} = \mathcal{F}_\sigma$.
  \end{thm}

  Ce théorème est une spécialisation de la \textit{dualité de Stone}.

  \begin{defn}
    On dit que $(X, \Omega X)$ est un \textit{espace de Stone} s'il est compact, Hausdorff et qu'il admet une base de clopens.
  \end{defn}

  \begin{thm}
    Si $(X, \Omega X)$ est un espace de Stone alors
    \[
      (X, \Omega X) \cong \mathbf{Sp}\big(\!\underbrace{\{C  \mid C \text{ clopen}\}, \subseteq}_{\text{algèbre de Boole}}\!\big)
    .\]
  \end{thm}

  On va définir le \textit{spectre} $\mathbf{Sp}(B)$ où $B$ est une algèbre de Boole, à l'aide des ultrafiltres sur $B$ et les filtres premiers (\textit{c.f.} les cinq propriétés ci-dessus).

  \begin{rmk}[Idée]
    Si $\mathcal{F} \subseteq B$ alors c'est une "théorie consistante et complète" où
    \begin{itemize}
      \item \textit{théorie} : stable par implication, \textit{c.f.} 1.--3.
      \item \textit{consistante} : sans contradiction, 5.
      \item \textit{complète} : 4., $\forall C \text{ clopen}$, si $C \not\in \mathcal{F}_\sigma$ alors $(\mathbf{2}^\mathrm{AP})^\omega \setminus C \in \mathcal{F}_\sigma$.
    \end{itemize}
  \end{rmk}

  \subsection{Extension de $\mathsf{LML}$ avec $\always$ et $\eventually$.}

  \renewcommand\qedsymbol{\textbf{QED}}

  \begin{rmk}
    Avec $\mathsf{LML}$, on ne définit que des clopens.
    Ainsi, les propriétés de sécurité ne sont que "finitaires" et il n'y a que $(\mathbf{2}^\mathrm{AP})^\omega$ comme propriété de vivacité.
  \end{rmk}

  \begin{defn}
    On ajoute à $\mathsf{LML}$ les modalités
    \begin{itemize}
      \item $\always \phi$ qui signifie "always" (notée parfois $\mathrm{A}\phi$) ;
      \item $\eventually \phi$ qui signifie "eventually" (notée parfois $\mathrm{E}\phi$) ;
    \end{itemize}
    où
    \begin{itemize}
      \item $\llbracket \always \phi\rrbracket := \mleft\{\,\sigma \in (\mathbf{2}^\mathrm{AP})^\omega \;\middle|\; \forall n \in \mathds{N}, \sigma \upharpoonright n \in \llbracket \phi\rrbracket \,\mright\}$ ;
      \item $\llbracket \eventually \phi \rrbracket := \mleft\{\,\sigma \;\middle|\; \exists n \in \mathds{N}, \sigma \upharpoonright n \in \llbracket \phi\rrbracket\,\mright\}$.
    \end{itemize}
  \end{defn}

  Dans la suite, les preuves se termineront par "\qedsymbol" au lieu du symbole usuel "$\square$", pour enlever l'ambigüité avec la modalité.

  \begin{nota}
    On note $\sigma \Vdash \phi$ pour $\sigma \in \llbracket \phi\rrbracket$.
  \end{nota}

  \begin{exm}
    On a 
    \begin{enumerate}
      \item $\sigma \Vdash \eventually \mathsf{a}$ ssi $\exists n \in \mathds{N}, \mathsf{a} \in \sigma(n)$,
        l'ensemble $\llbracket \eventually a\rrbracket $ est ouvert mais pas compact, donc pas un clopen car 
        \[
        \llbracket \eventually \mathsf{a} \rrbracket   = \bigcup_{u \in (\mathbf{2}^\mathrm{AP})^\star, \mathsf{a} \in A} \mathsf{ext}(u A)
        ;\] 
      \item $\sigma \Vdash \always \mathsf{a}$ ssi  $\forall n \in \mathds{N}, \mathsf{a} \in \sigma(n)$ (l'ensemble $\llbracket \always \mathsf{a} \rrbracket$ est un fermé non clopen) ;
      \item $\sigma \Vdash \always \eventually \mathsf{a}$ ssi $\exists^\infty n, \mathsf{a} \in \sigma(n)$ (c'est une propriété de vivacité) ;
      \item $\sigma \Vdash \eventually \always \mathsf{a}$ ssi $\forall^\infty n, \mathsf{a} \in \sigma(n)$ (c'est une propriété de vivacité).
    \end{enumerate}
  \end{exm}

  \begin{lem}
    On a que :
    \begin{enumerate}
      \item $\always \phi \equiv \lnot \eventually \lnot \phi$ ;
      \item $\eventually \phi \equiv \lnot \always \lnot \phi$ ;
      \item $\always \phi \equiv \phi \land \nxt \always \phi$ ;
      \item $\eventually \phi \equiv \phi \lor \nxt \eventually \phi$.
    \end{enumerate}
  \end{lem}

  Avec des $\land$ et des $\lor$ infinis, on aurait :
  \[
  \always \phi \equiv \phi \land \nxt \phi \land \nxt\nxt \phi \land \cdots \equiv \bigwedge_{n \in \mathds{N}}\nxt^n \phi
  \] et \[
  \eventually \phi \equiv \phi \lor \nxt \phi \lor \nxt\nxt \phi \lor \cdots \equiv \bigvee_{n \in \mathds{N}}\nxt^n \phi
  .\]

  \begin{defn}
    On étend ensuite $\mathsf{LML}$ par variables :
    \[
    \phi, \psi ::= X  \mid \cdots
    ,\] 
    où $X, Y, \ldots \in \mathcal{X}$.
    Une \textit{valuation} de $\mathcal{X}$ est une fonction \[
    \rho : \mathcal{X} \to \wp\big((\mathbf{2}^\mathrm{AP})^\omega\big)
    ,\]
    où la sémantique $\llbracket \phi\rrbracket_\rho$ est définie de manière très similaire à $\llbracket \phi\rrbracket  $ en ajoutant 
    \[
    \llbracket X\rrbracket_\rho := \rho(X) 
    .\]
  \end{defn}

  \begin{nota}
    Soient $\phi, \rho, X$, on pose 
    \begin{align*}
      \llbracket \phi\rrbracket_\rho(X) : \wp\big((\mathbf{2}^\mathrm{AP})^\omega\big) &\longrightarrow \wp\big((\mathbf{2}^\mathrm{AP})^\omega\big) \\
      A &\longmapsto \llbracket \phi\rrbracket_\rho[A / X]
    ,\end{align*}
    où $\rho[A / X](X) = A$ et  $\rho[A / X](Y) = \rho(Y)$ si  $Y \neq X$.
  \end{nota}

  \begin{lem}
    Soit $\phi$ et $X$ n'apparaissant pas dans $\phi$.
    On pose
    \[
      \phi_{\eventually} (X) := \phi \land \nxt X
      \quad\quad
      \phi_{\always} (X) := \phi \land \nxt X
    .\]
    Alors, pour tout $\rho$,
    \begin{itemize}
      \item $\llbracket \eventually \phi \rrbracket_\rho$ est le plus petit point fixe de $\llbracket \phi_{\eventually} \rrbracket_\rho(X)$ ;
      \item $\llbracket \always \phi \rrbracket_\rho$ est le plus grand point fixe de $\llbracket \phi_{\always} \rrbracket_\rho(X)$.
    \end{itemize}
  \end{lem}
  \begin{prv}
    Pour le premier point, on doit montrer les deux propriétés suivantes :
    \begin{enumerate}
      \item $\llbracket \eventually \phi\rrbracket_\rho = \llbracket \phi_{\eventually} \rrbracket (\llbracket \eventually \phi\rrbracket)$ ;
      \item $\forall P \subseteq (\mathbf{2}^\mathrm{AP})^\omega$, $P = \llbracket \phi_{\eventually} \rrbracket_\rho(P) \implies \llbracket \eventually \phi \rrbracket_\rho \subseteq P$.
    \end{enumerate}

    Pour la première propriété, c'est la loi d'expansion de $\eventually$.
    Pour la seconde, soit $P$ tel que $P = \llbracket \phi \lor \nxt X\rrbracket (P) $ et $\sigma \in \llbracket \eventually \phi\rrbracket$, et montrons que $\sigma \in P$.
    Soit $n \in \mathds{N}$ tel que $\sigma \upharpoonright n \in \llbracket \phi\rrbracket$.
    Alors, on a que $\sigma \upharpoonright n \in \llbracket \phi \lor \nxt X\rrbracket (P)$ et donc $\sigma \upharpoonright n \in P$.
    Donc $\sigma \upharpoonright (n-1) \in \llbracket \phi \lor \nxt \phi \rrbracket(P) = P$, donc pour tout $i \le n$,  $\sigma \upharpoonright i \in \llbracket \phi \lor \nxt X\rrbracket (P) = P $ et en particulier $\sigma = \sigma \upharpoonright 0 \in P$.
  \end{prv}

  \begin{defn}
    On dit qu'une variable $X$ est \textit{positive} dans $\phi$ si toutes les occurrences de $X$ dans $\phi$ sont sous un nombre pair de~$\lnot$.
    (On peut donner une définition inductive de "$X$ est positive dans $\phi$" et de "$\phi$ est négative dans $\phi$".)
  \end{defn}

  \begin{exm}
    Dans les exemples ci-dessous, on a que $X$ est positive dans $\phi$ :
    \[
    \mathsf{a} \land \nxt X \quad\quad \mathsf{a} \lor \nxt X \quad\quad \text{ et } \lnot (\lnot \mathsf{a} \land \nxt \lnot X)
    ,\] 
    mais, dans les exemples ci-dessous, $X$ n'est pas positive dans $\phi$ :
    \[
    \lnot X \quad\quad \text{ et }\quad\quad \lnot X \lor (\mathsf{a} \land \nxt X)
    .\]
  \end{exm}

  \begin{lem}
    Si $X$ est positive dans $\phi$ alors 
    \[
    \llbracket \phi\rrbracket(X) : \big(\wp\big((\mathbf{2}^\mathrm{AP})^\omega\big), \subseteq\big) \longrightarrow \big(\wp\big((\mathbf{2}^\mathrm{AP})^\omega\big), \subseteq\big)
    \]
    est monotone.
    \qed
  \end{lem}

  \subsection{Théorème de Knaster-Tarski.}

  \begin{defn}
    Soient $(L, \le)$ un poset et $f : L \to L$.
    \begin{enumerate}
      \item On dit que $a \in L$ est un \textit{pré-pointfixe} de $f$ si $f(a) \le a$.
      \item On dit que $a \in L$ est un \textit{post-pointfixe} de $f$ si $a \le f(a)$.
    \end{enumerate}
  \end{defn}

  \begin{thm}[Knaster-Tarski]
    Soit $f : L \to L$ une fonction monotone où $(L, \le)$ est un treillis complet.
    Alors 
    \[
    \mu(f) := \bigwedge \mleft\{\,a \in L \;\middle|\; f(a) \le a\,\mright\}
    \] 
    est le plus petit point fixe de $f$, et
    \[
    \nu(f) := \bigvee \mleft\{\,a \in L \;\middle|\; a \le f(a)\,\mright\}
    \] est le plus grand point fixe de $f$.
  \end{thm}
  \begin{prv}
    Vu en TD.
  \end{prv}

  \begin{lem}
    Soit $f : (\wp(X), \subseteq) \to (\wp(X), \subseteq)$ monotone. On pose 
    \begin{align*}
      g: \wp(X) &\longrightarrow \wp(X) \\
      A &\longmapsto g(A) = X \setminus f(X \setminus A)
    .\end{align*}
    On a alors que 
    \[
    \mu(f) = X \setminus \nu(g) \quad\quad \nu(f) = X \setminus \mu(g)
    .\] 
  \end{lem}
  \begin{prv}
    \begin{align*}
      X \setminus \nu(g)
      &= X \setminus \bigcup \mleft\{\,A  \;\middle|\; A \subseteq g(a)\,\mright\}  \\
      &= \bigcap \mleft\{\,X \setminus A \;\middle|\; A \subseteq g(A)\,\mright\} \\
      &= \bigcap \mleft\{\,X \setminus A \;\middle|\; A \subseteq X \setminus f(X \setminus A)\,\mright\} \\
      &= \bigcap \mleft\{\,X \setminus A \;\middle|\; f(X \setminus A) \subseteq X \setminus A\,\mright\} \\
      &= \bigcap \mleft\{\,B \;\middle|\; f(B) \subseteq B\,\mright\}  \\
      &= \nu(f)
    .\end{align*}
  \end{prv}

  \section{La logique $\mathsf{LTL}$.}

  La logique $\mathsf{LTL}$ est une extension de $\mathsf{LML}$ par certains points fixes.
  Soit $\theta(X)$ avec  $X$ positive dans $\theta$.
  On peut écrire 
  \[
  \theta(X) \equiv \psi \lor \Big( \bigwedge_{i \in I} \Big( \phi_i \land \bigwedge_{j \in J_i} \nxt^{n_{i,j}} X \Big) \Big)
  ,\] 
  où $X$ n'apparait pas dans $\psi$ ni dans $\phi_i$ pour tout $i \in I$.

  Dans $\mathsf{LTL}$, on va avoir les points fixes de $\theta$ si $n_{i,j} = 1$.
  Dans ce cas, on pourra écrire 
  \[
  \theta(X) \equiv \psi \lor (\phi \land \nxt X)
  ,\] où $X$ n'apparait pas dans $\psi$ ni dans $\phi$.

  \begin{defn}
    On définit la syntaxe de $\mathsf{LTL}$ par 
    \begin{align*}
      \phi, \psi ::= {} & \mathsf{a} & \mathsf{a} \in \mathrm{AP}\\
      \mid{} & \phi \land \psi  \\
      \mid{} & \phi \lor \psi  \\
      \mid{} & \mathsf{False} \\
      \mid{} & \mathsf{True} \\
      \mid{} & \lnot \phi \\
      \mid{} & \nxt \phi \\
      \mid{} & \phi \until \psi \\
    .\end{align*}
    La modalité $\phi \until \psi$ est notée $\phi \mathrel{\mathsf{U}} \psi$ dans les notes de cours.

    La sémantique pour $\mathsf{LTL}$ est définie de manière identique à la sémantique de $\mathsf{LML}$ en ajoutant 
    \[
    \textstyle \llbracket \phi \until \psi \rrbracket  
    = \mleft\{\,\sigma \in (\mathbf{2}^\mathrm{AP})^\omega \;\middle|\; \exists i \in \mathds{N}, 
    \begin{array}{r}
      \sigma \upharpoonright i \in \llbracket \psi\rrbracket \\
      \forall j < i, \sigma \upharpoonright j \in \llbracket \phi\rrbracket  
    \end{array}\,\mright\} 
    .\] 
  \end{defn}

  \subsection{Points fixes dans $\mathsf{LTL}$.}

  On étend, dans cette sous-section uniquement, $\mathsf{LTL}$ avec des variables (comme pour $\mathsf{LML}$).

  \begin{lem}
    Soit $X$ n'apparaissant pas dans $\phi$ ni $\psi$.
    Alors, on a que~$\llbracket \phi \until \psi \rrbracket$ est le plus petit point fixe de $\llbracket \theta\rrbracket(X)$.
  \end{lem}
  \begin{prv}
    On a 
    \[
    \phi \until \psi \equiv \psi \lor (\phi \land \nxt (\phi \until \psi))
    .\] 
    Soit $P$ tel que $P = \llbracket \theta\rrbracket(P)$.
    Soit $\sigma \in \llbracket \phi \until \psi\rrbracket$.
    Soit $i \in \mathds{N}$ tel que $\sigma \upharpoonright i \in \llbracket \psi\rrbracket$ et $\sigma \upharpoonright 0, \ldots, \sigma \upharpoonright (i - 1) \in \llbracket \phi\rrbracket$.
    On a $\sigma \upharpoonright i \in \llbracket \theta\rrbracket(P) = P$, et comme $\sigma \upharpoonright (i-1) \in \llbracket \phi\rrbracket $, on a que $\sigma \upharpoonright (i-1) \in \llbracket \theta \rrbracket (P) = P $.
    Ainsi, pour tout $k \in \{i-1, \ldots, 0\}$, on a $\sigma \upharpoonright k \in \llbracket \theta\rrbracket (P) = P $, donc $\sigma = \sigma \upharpoonright 0 \in P$.
  \end{prv}

  \begin{rmk}
    Si $X$ est (positive et) sous exactement un $\nxt$ dans  $\theta(X)$ alors c'est le cas aussi dans  $\lnot \theta[\lnot X / X]$.

    On a 
    \begin{align*}
      \lnot \theta[\lnot X / X] &= \lnot \big(\phi \lor (\psi \land \nxt (\lnot X)) \big)\\
      &\equiv \lnot \psi \land (\lnot \phi \lor \nxt X)  \\
      &\equiv(\lnot \psi \land \lnot \phi) \lor (\lnot \psi \land \nxt X) \\
    ,\end{align*}
    donc 
    \[
      \mu\llbracket \lnot \theta[\lnot X / X] \rrbracket (X) = \llbracket \lnot \psi \until (\lnot \psi \land \lnot \phi\rrbracket  
    .\]
  \end{rmk}

  \begin{lem}
    Soit $X$ n'apparaissant pas dans $\phi, \psi$.
    Alors
    \[
    \llbracket \lnot (\lnot \psi \until (\lnot \psi \land \lnot \phi)) \rrbracket  
    \] qui est un plus petit point fixe de $\llbracket \theta\rrbracket(X)$ pour $\theta(X) = \psi \lor (\phi \land \nxt X)$.
  \end{lem}

  \begin{nota}
    On note 
    \[
    \phi \wuntil \psi := \lnot (\lnot \psi \until (\lnot \psi \land \lnot \phi))
    ,\] 
    pour \textit{weak until}.
  \end{nota}

  \begin{nota}
    On note
    \[
    \eventually \phi := \mathsf{True} \until \phi \quad\quad \always \phi := \phi \wuntil \mathsf{False}
    ,\] 
    et ont la même sémantique que pour $\mathsf{LML}$.
  \end{nota}

  \begin{rmk}
    On peut montrer que le plus grand point fixe de~$\theta(X) = \mathsf{a} \land \nxt \nxt X$ n'est pas définissable dans $\mathsf{LTL}$.
  \end{rmk}

  \subsection{Équivalences logiques pour $\mathsf{LTL}$.}
  On note, comme avant, $\phi \equiv \psi$ si $\llbracket \phi\rrbracket = \llbracket \psi\rrbracket$.
  On a des lois, comme pour $\mathsf{LML}$, \textit{c.f.} figure 8 dans les notes de cours (section 7.3.3).

  \begin{exm}
    On a 
    \begin{gather*}
      \eventually \mathsf{False} \equiv \mathsf{False}\\
      \eventually (\phi \lor \psi) \equiv \eventually \phi \lor \eventually \psi\\
      \always \mathsf{True} \equiv \mathsf{True}\\
      \always (\phi \land \psi) \equiv \always \phi \land \always \psi\\
      \nxt (\phi \until \psi) \equiv \nxt \phi \until \nxt \psi
    \end{gather*}
  \end{exm}

  \begin{lem}
    On a
    \begin{enumerate}
      \item $\lnot (\phi \wuntil \psi) \equiv \lnot \psi \until (\lnot \psi \land \lnot \phi)$ ;
      \item $\lnot (\phi \until \psi) \equiv \lnot \psi \wuntil (\lnot \psi \land \lnot \phi)$.
    \end{enumerate}
    \qed
  \end{lem}

  \begin{lem}
    On a 
    \[
    \phi \wuntil \psi \equiv (\phi \until \psi) \lor \always \phi
    .\]
  \end{lem}
  \begin{prv}
    Soit $P = \llbracket (\phi \until \psi) \lor \always \phi \rrbracket$.
    On va montrer que $P = \nu\big(\llbracket \theta\rrbracket(X) \big)$ où $\theta = \psi \lor (\phi \land \nxt X)$.
    On a que  $P = \llbracket \theta\rrbracket(P)$ (c'est "facile").
    Soit $Q$ tel que $Q = \llbracket \theta\rrbracket(Q) $ et on va montrer que $Q \subseteq P$.
    Soit $\sigma \in Q$.
    \begin{itemize}
      \item Si $\sigma \in \llbracket \always \phi\rrbracket $ alors $\sigma \in P$ par définition de $P$.
      \item Sinon soit $i$ le plus petit tel que $\sigma \upharpoonright i \not\in \llbracket \phi\rrbracket$.
        Ainsi, par définition, $\sigma \upharpoonright j \in \phi$ pour $j < i$.
        On montre qu'il existe un certain $k \le i$ tel que $\sigma \upharpoonright k \in \llbracket \psi\rrbracket$.
        Par l'absurde, supposons que $\sigma \upharpoonright k \not\in \llbracket \psi\rrbracket$ pour tout $k \le i$.
        On voit que pour tout $k < i$, on a $\sigma \upharpoonright k \in \llbracket \theta\rrbracket(Q) = Q$ donc $\sigma \upharpoonright (i-1) \in \llbracket \theta\rrbracket(Q) = Q$, et donc $\sigma \upharpoonright (i-1) \not\in  \llbracket \psi\rrbracket$, ce qui implique $\sigma \upharpoonright i \in Q$, ce qui contredit la définition de $i$ (car $\sigma \upharpoonright i \not\in \llbracket \psi\rrbracket $ et $\sigma \upharpoonright i \not\in \llbracket \phi\rrbracket$).
    \end{itemize}
  \end{prv}
\end{document}
