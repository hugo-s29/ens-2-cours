\documentclass[./main]{subfiles}

\begin{document}
  \selectlanguage{french}
  \setquotestyle{french}

  \chapter{Bisimulation.}

  L'idée de ce chapitre est d'identifier les systèmes de transitions avec une "même structure de branchement".
  Par exemple, on identifie les deux systèmes suivants (le premier est un "dépliement" du second).

  \begin{figure}[h]
    \centering
    \begin{tikzpicture}[node distance=0.5cm]
      \node (b0) {};
      \foreach \x [remember=\x as \lastx (initially 0)] in {1, 2, 3, 4} {
        \node[right=0.5cm of b\lastx] (b\x) {$\bullet$};
        \draw[->] (b\lastx) to (b\x);
      }
      \node[right=0.5cm of b4] (b5) {$\cdots$};
      \draw[->] (b4) to (b5);

      \node[below=of b0] (a) {};
      \node[left=of b0] {$TS_1$};
      \node[left=of a] {$TS_2$};
      \node[right=0.5cm of a] (b) {$\bullet$};
      \draw[->] (a) to (b);
      \draw[->] (b) to[loop below] (b);
    \end{tikzpicture}
    \caption{Deux systèmes de transitions identifiés par bisimulation}
  \end{figure}

  \begin{defn}
    Soient $TS_0$ et $TS_1$ où
    \[
    TS_i = (S_i, \mathrm{Act}, \to_i, I_i, \mathrm{AP}, L_i)
    ,\] 
    où l'on note $s_i \xrightarrow \alpha s_i'$ où  $\alpha \in \mathrm{Act}$ et $s_i, s_i' \in S_i$, et $L_i : S_i \to \wp(\mathrm{AP})$.

    Une \textit{bisimulation} entre $TS_0$ et $TS_1$ est une relation $\mathcal{R} \subseteq S_0 \times S_1$ telle que
    \begin{enumerate}
      \item si $s_0 \mathrel{\mathcal{R}} s_1$ alors $L_0(s_0) = L_1(s_1)$ ;
      \item si $s_0 \mathrel{\mathcal{R}} s_1$ et $s_0 \xrightarrow \alpha s_0'$, alors il existe $s_1' \in S_1$ tel que $s_1 \xrightarrow \alpha s_1'$ et $s_0' \mathrel{\mathcal{R}} s_1'$.
      \item de même symétriquement.
    \end{enumerate}

    Ces deux dernières conditions peuvent être visualisées comme les deux diagrammes ci-dessous.
    \[
    \begin{tikzcd}
      s_0 \arrow{d}{\alpha} \arrow[dash]{r}{\mathcal{R}} & s_1 \arrow[dashed]{d}{\alpha}\\
      s_0' \arrow[dashed,dash]{r}{\mathcal{R}} & s_1'
    \end{tikzcd}
    \quad\quad
    \begin{tikzcd}
      s_0 \arrow[dashed]{d}{\alpha} \arrow[dash]{r}{\mathcal{R}} & s_1 \arrow{d}{\alpha}\\
      s_0' \arrow[dash, dashed]{r}{\mathcal{R}} & s_1'
    \end{tikzcd}
    .\]
  \end{defn}

  \begin{exm}
    Avec les deux systèmes de transitions suivants, on peut construire une bisimulation $\mathcal{R}$ avec
    \[
    \mathcal{R} = \mleft\{\,
      \begin{array}{l}
        (\mathsf{coin}_0, \mathsf{coin}_1),
        (\mathsf{select}_0, \mathsf{select}_1),
        (\mathsf{beer}_0, \mathsf{beer}_1),\\
        (\mathsf{beer}_0, \mathsf{beer}_1'),
        (\mathsf{soda}_0, \mathsf{soda}_1)
      \end{array}
     \,\mright\} 
    .\]

    \begin{figure}[H]
      \centering
      \begin{tikzpicture}
        \node[state, initial,initial where=above] (coin) {$\mathsf{coin}_0$} ;
        \node[state, below of=coin] (select) {$\mathsf{select}_0$} ;
        \node[state, right of=select] (soda) {$\mathsf{soda}_0$} ;
        \node[state, left of=select] (beer) {$\mathsf{beer}_0$} ;
        \node[below=0.1cm of beer] {$\{\mathsf{b}\}$};
        \node[below=0.1cm of soda] {$\{\mathsf{s}\}$};
        \node[right=0.1cm of coin] {$\{\mathsf{pay}\}$};
        \node[below=0.1cm of select] {$\emptyset$};
        \draw[->] (soda) edge node[swap] {$\mathtt{gs}$} (coin);
        \draw[->] (beer) edge node {$\mathtt{gb}$} (coin);
        \draw[->] (select) edge node[swap] {$\tau$} (soda);
        \draw[->] (coin) edge node {$\mathtt{ic}$} (select);
        \draw[->] (select) edge node {$\tau$} (beer);
      \end{tikzpicture}
      \begin{tikzpicture}
        \node[state, initial,initial where=above] (coin) {$\mathsf{coin}_1$} ;
        \node[state, below of=coin] (select) {$\mathsf{select}_1$} ;
        \node[state, right of=select] (soda) {$\mathsf{soda}_1$} ;
        \node[state, left of=select] (beer) {$\mathsf{beer}_1$} ;
        \node[state, below of=select] (beer2) {$\mathsf{beer}'_1$} ;
        \node[below=0.1cm of beer] {$\{\mathsf{b}\}$};
        \node[right=0.1cm of beer2] {$\{\mathsf{b}\}$};
        \node[below=0.1cm of soda] {$\{\mathsf{s}\}$};
        \node[right=0.1cm of coin] {$\{\mathsf{pay}\}$};
        \node[below left=0.1cm of select] {$\emptyset$};
        \draw[->] (soda) edge node[swap] {$\mathtt{gs}$} (coin);
        \draw[->] (beer) edge node {$\mathtt{gb}$} (coin);
        \draw[->] (select) edge node[swap] {$\tau$} (soda);
        \draw[->] (coin) edge node {$\mathtt{ic}$} (select);
        \draw[->] (select) edge node {$\tau$} (beer);
        \draw[->] (select) edge node {$\tau$} (beer2);
        \node[left=0.4cm of beer,inner sep=0,outer sep=0] (temp) {};
        \node[left=0.1cm of temp] {$\mathtt{gb}$};
        \draw[rounded corners, straight-left] (beer2) edge (temp.center);
        \draw[->, rounded corners, straight-right] (temp.center) to (coin);
      \end{tikzpicture}
    \end{figure}
  \end{exm}

  \begin{defn}
    Soient $TS_0$ et $TS_1$ deux systèmes de transitions.
    La relation de \textit{bisimilarité} entre $TS_0$ et $TS_1$ est donnée par 
    \[
      s_0 \sim s_1 \iff \exists \mathcal{R} \text{ une bisimulation}, \: s_0 \mathrel{\mathcal{R}} s_1
    .\]
  \end{defn}
\end{document}
