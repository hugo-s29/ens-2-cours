\documentclass[./main]{subfiles}

\begin{document}
  \selectlanguage{french}
  \setquotestyle{french}

  \chapter{Bisimulation.}

  L'idée de ce chapitre est d'identifier les systèmes de transitions avec une "même structure de branchement".
  Par exemple, on identifie les deux systèmes suivants (le premier est un "dépliement" du second).

  \begin{figure}[h]
    \centering
    \begin{tikzpicture}[node distance=0.5cm]
      \node (b0) {};
      \foreach \x [remember=\x as \lastx (initially 0)] in {1, 2, 3, 4} {
        \node[right=0.5cm of b\lastx] (b\x) {$\bullet$};
        \draw[->] (b\lastx) to (b\x);
      }
      \node[right=0.5cm of b4] (b5) {$\cdots$};
      \draw[->] (b4) to (b5);

      \node[below=of b0] (a) {};
      \node[left=of b0] {$TS_1$};
      \node[left=of a] {$TS_2$};
      \node[right=0.5cm of a] (b) {$\bullet$};
      \draw[->] (a) to (b);
      \draw[->] (b) to[loop below] (b);
    \end{tikzpicture}
    \caption{Deux systèmes de transitions identifiés par bisimulation}
  \end{figure}

  \begin{defn}
    Soient $TS_0$ et $TS_1$ où
    \[
    TS_i = (S_i, \mathrm{Act}, \to_i, I_i, \mathrm{AP}, L_i)
    ,\] 
    où l'on note $s_i \xrightarrow \alpha s_i'$ où  $\alpha \in \mathrm{Act}$ et $s_i, s_i' \in S_i$, et $L_i : S_i \to \wp(\mathrm{AP})$.

    Une \textit{bisimulation} entre $TS_0$ et $TS_1$ est une relation $\mathcal{R} \subseteq S_0 \times S_1$ telle que
    \begin{enumerate}
      \item si $s_0 \mathrel{\mathcal{R}} s_1$ alors $L_0(s_0) = L_1(s_1)$ ;
      \item si $s_0 \mathrel{\mathcal{R}} s_1$ et $s_0 \xrightarrow \alpha s_0'$, alors il existe $s_1' \in S_1$ tel que $s_1 \xrightarrow \alpha s_1'$ et $s_0' \mathrel{\mathcal{R}} s_1'$.
      \item de même symétriquement.
    \end{enumerate}

    Ces deux dernières conditions peuvent être visualisées comme les deux diagrammes ci-dessous.
    \[
    \begin{tikzcd}
      s_0 \arrow{d}{\alpha} \arrow[dash]{r}{\mathcal{R}} & s_1 \arrow[dashed]{d}{\alpha}\\
      s_0' \arrow[dashed,dash]{r}{\mathcal{R}} & s_1'
    \end{tikzcd}
    \quad\quad
    \begin{tikzcd}
      s_0 \arrow[dashed]{d}{\alpha} \arrow[dash]{r}{\mathcal{R}} & s_1 \arrow{d}{\alpha}\\
      s_0' \arrow[dash, dashed]{r}{\mathcal{R}} & s_1'
    \end{tikzcd}
    .\]
  \end{defn}

  \begin{exm}
    Avec les deux systèmes de transitions suivants, on peut construire une bisimulation $\mathcal{R}$ avec
    \[
    \mathcal{R} = \mleft\{\,
      \begin{array}{l}
        (\mathsf{coin}_0, \mathsf{coin}_1),
        (\mathsf{select}_0, \mathsf{select}_1),
        (\mathsf{beer}_0, \mathsf{beer}_1),\\
        (\mathsf{beer}_0, \mathsf{beer}_1'),
        (\mathsf{soda}_0, \mathsf{soda}_1)
      \end{array}
     \,\mright\} 
    .\]

    \begin{figure}[H]
      \centering
      \begin{tikzpicture}
        \node[state, initial,initial where=above] (coin) {$\mathsf{coin}_0$} ;
        \node[state, below of=coin] (select) {$\mathsf{select}_0$} ;
        \node[state, right of=select] (soda) {$\mathsf{soda}_0$} ;
        \node[state, left of=select] (beer) {$\mathsf{beer}_0$} ;
        \node[below=0.1cm of beer] {$\{\mathsf{b}\}$};
        \node[below=0.1cm of soda] {$\{\mathsf{s}\}$};
        \node[right=0.1cm of coin] {$\{\mathsf{pay}\}$};
        \node[below=0.1cm of select] {$\emptyset$};
        \draw[->] (soda) edge node[swap] {$\mathtt{gs}$} (coin);
        \draw[->] (beer) edge node {$\mathtt{gb}$} (coin);
        \draw[->] (select) edge node[swap] {$\tau$} (soda);
        \draw[->] (coin) edge node {$\mathtt{ic}$} (select);
        \draw[->] (select) edge node {$\tau$} (beer);
      \end{tikzpicture}
      \begin{tikzpicture}
        \node[state, initial,initial where=above] (coin) {$\mathsf{coin}_1$} ;
        \node[state, below of=coin] (select) {$\mathsf{select}_1$} ;
        \node[state, right of=select] (soda) {$\mathsf{soda}_1$} ;
        \node[state, left of=select] (beer) {$\mathsf{beer}_1$} ;
        \node[state, below of=select] (beer2) {$\mathsf{beer}'_1$} ;
        \node[below=0.1cm of beer] {$\{\mathsf{b}\}$};
        \node[right=0.1cm of beer2] {$\{\mathsf{b}\}$};
        \node[below=0.1cm of soda] {$\{\mathsf{s}\}$};
        \node[right=0.1cm of coin] {$\{\mathsf{pay}\}$};
        \node[below left=0.1cm of select] {$\emptyset$};
        \draw[->] (soda) edge node[swap] {$\mathtt{gs}$} (coin);
        \draw[->] (beer) edge node {$\mathtt{gb}$} (coin);
        \draw[->] (select) edge node[swap] {$\tau$} (soda);
        \draw[->] (coin) edge node {$\mathtt{ic}$} (select);
        \draw[->] (select) edge node {$\tau$} (beer);
        \draw[->] (select) edge node {$\tau$} (beer2);
        \node[left=0.4cm of beer,inner sep=0,outer sep=0] (temp) {};
        \node[left=0.1cm of temp] {$\mathtt{gb}$};
        \draw[rounded corners, straight-left] (beer2) edge (temp.center);
        \draw[->, rounded corners, straight-right] (temp.center) to (coin);
      \end{tikzpicture}
    \end{figure}
  \end{exm}

  \begin{defn}
    Soient $TS_0$ et $TS_1$ deux systèmes de transitions.
    La relation de \textit{bisimilarité} entre $TS_0$ et $TS_1$ est donnée par 
    \[
      s_0 \sim s_1 \iff \exists \mathcal{R} \text{ une bisimulation}, \: s_0 \mathrel{\mathcal{R}} s_1
    .\]
    On dit alors que $s_0$ est \textit{bisimilaire} à $s_1$.
  \end{defn}

  \begin{lem}
    \begin{enumerate}
      \item Étant donné $TS$, on a que $s$ est bisimilaire à~$s$ (car $\{(s,s)  \mid s \in S\}$ est une bisimulation entre $TS$ et lui-même).
      \item Si $\mathcal{R}$ est une bisimulation entre $TS_1$ et $TS_2$, alors 
        \[
        \mathcal{R}^{-1} := \{(s_2, s_1)  \mid (s_1, s_2) \in \mathcal{R}\}
        \] 
        est une bisimulation entre $TS_2$ et $TS_1$.
      \item Si $\mathcal{R}$ est une bisimulation entre $TS_1$ et $TS_2$, et $\mathcal{S}$ est une bisimulation entre $TS_2$ et $TS_3$, alors 
        \[
        \mathcal{S} \circ \mathcal{R} := \mleft\{\,(s_1, s_3) \;\middle|\; \exists s_2,
        \begin{array}{l}
          (s_1, s_2) \in \mathcal{R}\\
          (s_2, s_3) \in \mathcal{S}
        \end{array}\,\mright\}  
        \]
        est une bisimulation entre $TS_1$ et $TS_2$.
    \end{enumerate}
    \qed
  \end{lem}

  \begin{lem}
    \begin{enumerate}
      \item La relation de bisimilarité $\sim$ entre $TS_1$ et $TS_2$ est une bisimulation.
      \item Si $\mathcal{R}$ est une bisimulation entre $TS_1$ et $TS_2$ alors $\mathcal{R} \subseteq {\sim}$.
      \item La relation $\sim$ entre $TS$ et lui-même est une relation d'équivalence.
    \end{enumerate}
  \end{lem}
  \begin{prv}
    (Idée)
    \begin{enumerate}
      \item Soient $s_1 \sim s_2$, et soit donc $\mathcal{R}$ telle que $s_1 \mathrel{\mathcal{R}} s_2$.
        Donc,
        \begin{itemize}
          \item $L_1(s_1) = L_2(s_2)$ ;
          \item comme $s_1' \mathrel{\mathcal{R}} s_2'$, on a $s_1' \sim s_2'$ où
            \[
            \begin{tikzcd}
              s_0 \arrow{d}{\alpha} \arrow[dash]{r}{\mathcal{R}} & s_1 \arrow[dashed]{d}{\alpha}\\
              s_0' \arrow[dashed,dash]{r}{\mathcal{R}} & s_1'
            \end{tikzcd}
            ;\]
          \item et symétriquement pour le cas dual.
        \end{itemize}
      \item Si $s_1 \mathrel{\mathcal{R}} s_2$ alors $s_1 \sim s_2$ par définition de $\sim$.
      \item On a les points suivants.
        \begin{description}
          \item[Réflexivité.]
            On a $s \sim s$.
          \item[Symétrie.] Si $s \sim s'$ alors  $s \mathrel{\mathcal{R}} s'$ pour une certaine bisimulation $\mathcal{R}$, et donc $s' \mathrel{\mathcal{R}^{-1}} s$ et donc $s \sim s'$.
          \item[Transitivité.]
            Si $s \sim s'$ et  $s' \sim s''$ alors  $s \mathrel{\mathcal{R}} s'$ et $s' \mathrel{\mathcal{S}} s''$, et donc $s \mathrel{(\mathcal{S} \circ \mathcal{R})} s''$ d'où $s \sim s''$.
        \end{description}
    \end{enumerate}
  \end{prv}

  \section{Quotient par bisimulation.}

  \begin{defn}[Quotient par bisimulation.]
    Soit un système de transitions $TS = (S, \mathrm{Act}, {\to}, I, \mathrm{AP}, L)$.
    On définit $TS_\sim$ par :
     \begin{itemize}
       \item $S_\sim := \{[s]_\sim  \mid s \in S\}$ ;
       \item $\mathrm{Act}_\sim := \mathrm{Act}$ ;
       \item $[s]_\sim \xrightarrow{\alpha} [s']_\sim$ ssi  $s \xrightarrow{\alpha} s'$ ;
       \item $I_\sim = \{[s]_\sim  \mid s \in I\}$ ;
       \item $\mathrm{AP}_\sim := \mathrm{AP}$ ;
       \item $L_\sim([s]_\sim) = L(s)$.
    \end{itemize}
  \end{defn}

  \begin{lem}
    Pour tout $s \in I$ et $s \sim [s]_\sim$.
  \end{lem}

  \begin{defn}
    Soient $TS_1 = (S_1, \mathrm{Act}, {\to_1}, I_1, \mathrm{AP}, L_1)$ et
    $TS_2 = (S_2, \mathrm{Act}, {\to_2}, I_2, \mathrm{AP}, L_2)$ deux systèmes de transitions.

    On note $TS_1 \approx TS_2$ s'il existe une bisimulation $\mathcal{R}$ entre $TS_1$ et $TS_2$ telle que
    \begin{enumerate}
      \item pour tout $s_1 \in I_1$, il existe $s_2 \in I_2$ tel que $s_1 \mathrel{\mathcal{R}} s_2$ ;
      \item pour tout $s_2 \in I_2$, il existe $s_1 \in I_1$ tel que $s_1 \mathrel{\mathcal{R}} s_2$.
    \end{enumerate}
  \end{defn}

  \section{Bisimilarité et équivalence de traces.}

  \begin{prop}
    Soient $TS_1, TS_2$ sur un même ensemble d'actions $\mathrm{Act}$ et de propositions atomiques $\mathrm{AP}$.
    Si $TS_1 \approx TS_2$ alors \[
    \mathrm{Tr}^\omega (TS_1) = \mathrm{Tr}^\omega(TS_2)
    .\] 
  \end{prop}

  \begin{crlr}
    Si $TS_1 \approx TS_2$ alors, pour toute propriété LT $P \subseteq (\mathbf{2}^\mathrm{AP})^\omega$, on a 
    \[
    TS_1 \md P \quad \iff \quad TS_2 \md P
    .\]
  \end{crlr}

  \begin{exm}
    Pour un système de transitions $TS$, on a $TS \approx TS_\sim$, on peut donc se ramener au quotient par bisimilarité, pour tester si $TS \md P$.
  \end{exm}

  \begin{exm}
    Le quotient par bisimilarité du système de transitions
    \begin{center}
      \begin{tikzpicture}[every state/.style={minimum size=12pt,draw=deepblue,thick,rounded rectangle}]
        \node[state, initial, initial where=left] (A) {};
        \node[state, right of=A] (B) {};
        \node[state, right of=B] (C) {};
        \node[state, right of=C] (D) {};
        \node[state, right of=D] (E) {};
        \node[right of=E] (F) {$\cdots$};
        \draw[->] (A) edge (B);
        \draw[->] (B) edge (C);
        \draw[->] (C) edge (D);
        \draw[->] (D) edge (E);
        \draw[->] (E) edge (F);
        \node[above=0.1cm of A] {$\{\mathsf{a}\}$};
        \node[above=0.1cm of B] {$\{\mathsf{a}\}$};
        \node[above=0.1cm of C] {$\{\mathsf{a}\}$};
        \node[above=0.1cm of D] {$\{\mathsf{a}\}$};
        \node[above=0.1cm of E] {$\{\mathsf{a}\}$};
      \end{tikzpicture}
    \end{center}
    est le système de transitions
    \begin{center}
      \begin{tikzpicture}[every state/.style={minimum size=12pt,draw=deepblue,thick,rounded rectangle}]
        \node[state, initial, initial where=left] (A) {};
        \draw[->] (A) edge[loop below] (A);
        \node[above=0.1cm of A] {$\{\mathsf{a}\}$};
      \end{tikzpicture}
    \end{center}
    car tous les états sont bisimilaires dans le système de transitions original.
  \end{exm}
\end{document}
