\documentclass[./main]{subfiles}

\begin{document}
  \selectlanguage{french}
  \setquotestyle{french}

  \chapter{Logique modale pour systèmes de transitions.}

  \section{Modèle de Kripke.}

  Dans un système de transition $TS = (S, \mathrm{Act}, {\to}, I, \mathrm{AP}, L)$, on différencie 
  \begin{itemize}
    \item la partie \textit{\textbf{transitions}} $K = (S, \mathrm{Act}, {\to})$ ;
    \item la partie \textit{\textbf{logique}} $M = (K, \mathrm{AP}, L)$ ;
    \item la partie \textit{\textbf{structure "pointée"}} $(M, I)$.
  \end{itemize}

  \begin{defn}
    Une \textit{structure de Kripke} (sur $\mathrm{AP}$) est de la forme \[
    K = (S, \mathrm{Act}, \to)
    .\]

    Un \textit{modèle de Kripke} sur $\mathrm{Act}$ et $\mathrm{AP}$ est de la forme 
    \[
    M = (K, \mathrm{AP}, L)
    \] 
    où $K$ est une structure de Kripke sur $\mathrm{Act}$.
  \end{defn}

  \section{Logique de Hennessy-Milner.}

  Fixons $\mathrm{Act}$ et $\mathrm{AP}$.

  \begin{defn}
    On définit les formules de la logique $\mathsf{HML}$ par :
    \begin{align*}
      \phi, \psi ::={}& \texttt{a} && \texttt{a} \in \mathrm{AP}\\
      \mid {} & \phi \land \psi  \mid \top  \\
      \mid {} & \phi \lor \psi  \mid \bot  \\
      \mid {} & \lnot \phi \\
      \mid {} & \underbrace{\langle \alpha \rangle}_{\eventually} \phi  \mid \underbrace{[\alpha]}_{\always} \phi && \alpha \in \mathrm{Act}
    .\end{align*}
  \end{defn}

  \begin{defn}
    Pour $M = (S, \mathrm{Act}, {\to}, \mathrm{AP}, L)$, on définit $\llbracket \phi\rrbracket \in \wp(S) $ par induction sur $\phi$ :
     \begin{itemize}
      \item $\llbracket \texttt{a}\rrbracket := \{s \in S  \mid \texttt{a} \in L(s)\}$ ;
      \item $\llbracket \top \rrbracket := S$ ;
      \item $\llbracket \bot \rrbracket := \emptyset$ ;
      \item $\llbracket \phi \land \psi \rrbracket := \llbracket \phi\rrbracket  \cap \llbracket \psi\rrbracket  $ ;
      \item $\llbracket \phi \lor \psi \rrbracket := \llbracket \phi\rrbracket  \cup \llbracket \psi\rrbracket  $ ;
      \item $\llbracket \lnot \phi \rrbracket := \wp(S) \setminus \llbracket \phi\rrbracket$ ;
      \item $\llbracket \langle \alpha \rangle \phi\rrbracket  := \{s \in S  \mid \exists s' \in S, s \xrightarrow{\alpha} s' \text{ et } s' \in \llbracket \phi\rrbracket \}$ ;
      \item $\llbracket [ \alpha ] \phi\rrbracket  := \{s \in S  \mid\forall s' \in S, s \xrightarrow{\alpha} s' \text{ implique } s' \in \llbracket \phi\rrbracket \}$ ;
    \end{itemize}
  \end{defn}

  \begin{nota}
    On note $s \Vdash \phi$ si  $s \in \llbracket \phi\rrbracket$.
  \end{nota}

  \begin{defn}
    Soient $\mathrm{Act}$, $\mathrm{AP}$ et soit $\phi$ une formule $\mathsf{HML}$.
     \begin{enumerate}
       \item Pour $M = (S, \mathrm{Act}, \to, \mathrm{AP}, L)$, on dit que $s \in S$ \textit{\textbf{satisfait}} $\phi$ si $s \Vdash \phi$. 
       \item On dit que $\phi$ est \textit{\textbf{valide}} dans $M = (S, \mathrm{Act}, \to, \mathrm{AP}, L)$, que l'on note $M \models \phi$, si 
          \[
         \forall s \in S, \quad s \Vdash \phi
         .\] 
       \item On dit que $\phi$ est \textit{\textbf{valide}}, que l'on note $\models \phi$, si  $\phi$ est  valide dans tout modèle $M$.
    \end{enumerate}
  \end{defn}

  \begin{exm}[Logique $\mathsf{LML}$.]
    Soit  $\mathrm{Act} = \mathbf{1} = \{\bullet\}$.
    Soit \[
    M((\mathbf{2}^\mathrm{AP})^\omega) := (S, \mathbf{1}, \to, \mathrm{AP}, L)
    \] 
    où
    \begin{itemize}
      \item $S = (\mathbf{2}^\mathrm{AP})^\omega$ ;
      \item pour tout $\sigma$, on a que $\sigma \xrightarrow\bullet \sigma \upharpoonright 1$ ;
      \item $L(\sigma) = \sigma(0)$.
    \end{itemize}
    On a 
    \[
      \sigma \Vdash \langle \bullet \rangle \phi \iff \sigma \upharpoonright 1 \Vdash \phi \iff \phi \Vdash  [\bullet] \phi
    ,\]
    donc $\langle \bullet \rangle$ et $[\bullet]$ correspondent ainsi à la modalité $\nxt$ "next".

    De plus, sur ce système, on a $\sigma \sim s'$ ssi $\sigma = \sigma'$.
  \end{exm}

  \section{Équivalences logiques.}

  \subsection{Équivalences logiques pour les formules.}

  \begin{defn}
    Soient $\mathrm{AP}$, $\mathrm{Act}$, et soient $\phi, \psi$ des formules.
    On note $\phi \equiv \psi$ ssi
     \[
    \forall M = (S, \mathrm{Act}, \to, \mathrm{AP}, L),
    \quad \llbracket \phi = \psi\rrbracket  
    ,\] 
    autrement dit $\forall  M, \forall s \in S, \quad s \Vdash \phi \iff s \Vdash \psi$.
  \end{defn}

  \begin{lem}
    On a que $[\alpha]-$ et  $\langle \alpha\rangle -$ sont duaux par De Morgan :
    \[
      [\alpha] \phi \equiv \lnot \langle \alpha \rangle \lnot \phi
      \quad\text{et}\quad
      \langle\alpha\rangle \phi \equiv \lnot [\alpha] \lnot \phi
    .\]

    De plus, comme $\langle \alpha \rangle -$ est défini comme quantification existentielle, on a 
     \[
    \langle \alpha \rangle (\phi \lor \psi) \equiv \langle \alpha \rangle \phi \lor \langle \alpha \rangle \psi
    \quad\text{et}\quad
    \langle \alpha \rangle \bot \equiv\bot
    .\]
    De même, comme $[\alpha]-$ est défini comme quantification universelle, on a 
     \[
       [\alpha] (\phi \land \psi) \equiv [ \alpha ] \phi \land [ \alpha ] \psi
    \quad\text{et}\quad
      [\alpha] \top \equiv \top
    .\]
  \end{lem}

  \subsection{Équivalences logiques pour les états.}

  \begin{defn}
    Soient $\mathrm{Act}$ et $\mathrm{AP}$.
    On se donne $M_1$, et $M_2$  deux modèles de Kripke.
    Pour $s_1 \in S_1$ et $s_2 \in S_2$, on note \[
    s_1 \equiv s_2 \quad\quad \text{ ssi } \quad\quad \forall \phi, \quad s_1 \Vdash \phi \iff s_2 \Vdash \phi
    .\]
  \end{defn}

  \begin{rmk}
    On aurait pu définir une telle équivalence pour $\mathsf{LML}$, mais on aurait que $\sigma_1 \equiv \sigma_2$ ssi $\sigma_1 = \sigma_2$, ce qui n'est pas très intéressant.
    Avec des modèles de Kripke et la logique $\mathsf{HML}$, on a des états différents qui sont équivalents.
  \end{rmk}

  \begin{thm}
    Si $s_1 \sim s_2$ alors $s_1 \equiv s_2$.
  \end{thm}
  \begin{prv}
    Par induction sur $\phi$, on montre que 
    \[
    s_1 \sim s_2 \implies (s_1 \Vdash \phi \iff s_2 \Vdash \phi)
    .\] 
    \begin{description}
      \item[Cas de $\texttt{a} \in \mathrm{AP}$.]
        Si $s_1 \sim s_2$ alors $L_1(s_1) = L_2(s_2)$ donc $s_1 \Vdash \texttt{a}$ ssi $s_2 \Vdash \texttt{a}$.
      \item[Cas de $-\land-$, $-\lor-$, $\top$, $\bot$, et $\lnot-$.]
        On applique l'hypothèse d'induction et/ou l'hypothèse $s_1 \sim s_2$, et on conclut.
      \item[Cas de $\langle \alpha \rangle -$.]
        Supposons $s_1 \sim s_2$.
        Supposons $s_1 \Vdash \langle \alpha \rangle \phi$.
        Il existe donc $s_1'$ tel que $s_1 \xrightarrow\alpha s_1'$ et $s_1' \Vdash \phi$.
        Par bisimulation
        \[
        \begin{tikzcd}
          s_1 \arrow{d}{\alpha} \arrow[dash]{r}{\sim} & s_2 \arrow[dashed]{d}{\alpha}\\
          s_1' \arrow[dashed,dash]{r}{\sim} & s_2'
        \end{tikzcd}
        ,\] 
        il existe donc $s_2'$ tel que $s_2 \xrightarrow \alpha s_2'$ et $s_1' \sim s_2'$.
        Donc, par hypothèse d'induction, on a que $s_2' \Vdash \phi$.
        On a donc $s_2 \Vdash \langle \alpha \rangle \phi$.
        On procédant de même dans l'autre cas, on a bien que \[
        s_1 \Vdash \langle \alpha \rangle \phi \iff
        s_2 \Vdash \langle \alpha \rangle \phi
        .\]
      \item[Cas de $\lbracket \alpha \rbracket -$.]
        On a que $[\alpha] \phi \equiv \lnot \langle \alpha \rangle \lnot \phi$.
    \end{description}
  \end{prv}

  \section{Propriété de Hennessy-Milner.}

  \begin{rmk}[Question]
    Est-ce que $s_1 \equiv s_2$ implique $s_1 \sim s_2$ ?
    Autrement dit, est-ce que $\equiv$ est une bisimulation ?

    La réponse est \textit{\textbf{non}} en général.
  \end{rmk}

  \begin{exm}
    Considérons les système de transitions suivants.
    \begin{center}
      \begin{tikzpicture}[every state/.style={minimum size=12pt,draw=deepblue,thick,rounded rectangle}]
        \node[state, initial, initial where=above] (A) {$s_1$};
        \node[state, below left of=A] (B) {};
        \draw[->] (A) edge (B);
        \node[state, below of=A] (B) {};
        \node[state, below of=B] (C) {};
        \draw[->] (A) edge (B);
        \draw[->] (B) edge (C);
        \node[state, below right of=A] (B) {};
        \node[state, below right of=B] (C) {};
        \node[state, below right of=C] (D) {};
        \draw[->] (A) edge (B);
        \draw[->] (B) edge (C);
        \draw[->] (C) edge (D);
        \node[right of=B] {$\adots$};
      \end{tikzpicture}
    \end{center}
    \begin{center}
      \begin{tikzpicture}[every state/.style={minimum size=12pt,draw=deepblue,thick,rounded rectangle}]
        \node[state, initial, initial where=above] (A) {$s_2$};
        \node[state, below left of=A] (B) {};
        \draw[->] (A) edge (B);
        \node[state, below of=A] (B) {};
        \node[state, below of=B] (C) {};
        \draw[->] (A) edge (B);
        \draw[->] (B) edge (C);
        \node[state, below right of=A] (B) {};
        \node[state, below right of=B] (C) {};
        \node[state, below right of=C] (D) {};
        \draw[->] (A) edge (B);
        \draw[->] (B) edge (C);
        \draw[->] (C) edge (D);
        \node[right of=B] {$\adots$};
        \node[state, right of=A] (B) {};
        \node[state, right of=B] (C) {};
        \node[state, right of=C] (D) {};
        \node[right of=D] (E) {$\cdots$};
        \draw[->] (A) edge (B);
        \draw[->] (B) edge (C);
        \draw[->] (C) edge (D);
        \draw[->] (D) edge (E);
      \end{tikzpicture}
    \end{center}
    On remarque que $s_1 \nsim s_2$ mais $s_1 \equiv s_2$ (la logique $\mathsf{HML}$ ne peut "voir" qu'à une profondeur finie).
  \end{exm}

  \begin{defn}
    Une classe $\mathfrak{M}$ de modèles a la \textit{\textbf{propriété de Hennessy-Milner}} si, dans $\mathfrak{M}$, l'équivalence  $\equiv$ entre états est une bisimulation.
  \end{defn}

  \begin{defn}
    Soit $K = (S, \mathrm{Act}, \to)$ alors, pour $\alpha \in \mathrm{Act}$ et $s \in S$, on note 
    \[
      \mathrm{Succ}^\alpha(s) := \{s' \in S  \mid s \xrightarrow \alpha s' \} 
    .\]
  \end{defn}

  \begin{defn}
    Un modèle de Kripke $M$ est dit \textit{\textbf{à image finie}} si, pour tout $s \in S$ et $\alpha \in \mathrm{Act}$, l'ensemble $\mathrm{Succ}^\alpha(s)$ des successeurs de $s$ par une action $\alpha$ fixée est finie.
  \end{defn}

  \begin{prop}[Hennessy-Milner]
    La classe des modèles à image finie a la propriété de Hennessy-Milner.
  \end{prop}
\end{document}
