\documentclass[./main]{subfiles}

\begin{document}
  \selectlanguage{french}
  \setquotestyle{french}

  \chapter{Logique modale pour systèmes de transitions.}

  \section{Modèle de Kripke.}

  Dans un système de transition $TS = (S, \mathrm{Act}, {\to}, I, \mathrm{AP}, L)$, on différencie 
  \begin{itemize}
    \item la partie \textit{\textbf{transitions}} $K = (S, \mathrm{Act}, {\to})$ ;
    \item la partie \textit{\textbf{logique}} $M = (K, \mathrm{AP}, L)$ ;
    \item la partie \textit{\textbf{structure "pointée"}} $(M, I)$.
  \end{itemize}

  \begin{defn}
    Une \textit{structure de Kripke} (sur $\mathrm{AP}$) est de la forme \[
    K = (S, \mathrm{Act}, \to)
    .\]

    Un \textit{modèle de Kripke} sur $\mathrm{Act}$ et $\mathrm{AP}$ est de la forme 
    \[
    M = (K, \mathrm{AP}, L)
    \] 
    où $K$ est une structure de Kripke sur $\mathrm{Act}$.
  \end{defn}

  \section{Logique de Hennessy-Milner.}

  Fixons $\mathrm{Act}$ et $\mathrm{AP}$.

  \begin{defn}
    On définit les formules de la logique $\mathsf{HML}$ par :
    \begin{align*}
      \phi, \psi ::={}& \texttt{a} && \texttt{a} \in \mathrm{AP}\\
      \mid {} & \phi \land \psi  \mid \top  \\
      \mid {} & \phi \lor \psi  \mid \bot  \\
      \mid {} & \lnot \phi \\
      \mid {} & \underbrace{\langle \alpha \rangle}_{\eventually} \phi  \mid \underbrace{[\alpha]}_{\always} \phi && \alpha \in \mathrm{Act}
    .\end{align*}
  \end{defn}

  \begin{defn}
    Pour $M = (S, \mathrm{Act}, {\to}, \mathrm{AP}, L)$, on définit $\llbracket \phi\rrbracket \in \wp(S) $ par induction sur $\phi$ :
     \begin{itemize}
      \item $\llbracket \texttt{a}\rrbracket := \{s \in S  \mid \texttt{a} \in L(s)\}$ ;
      \item $\llbracket \top \rrbracket := S$ ;
      \item $\llbracket \bot \rrbracket := \emptyset$ ;
      \item $\llbracket \phi \land \psi \rrbracket := \llbracket \phi\rrbracket  \cap \llbracket \psi\rrbracket  $ ;
      \item $\llbracket \phi \lor \psi \rrbracket := \llbracket \phi\rrbracket  \cup \llbracket \psi\rrbracket  $ ;
      \item $\llbracket \lnot \phi \rrbracket := \wp(S) \setminus \llbracket \phi\rrbracket$ ;
      \item $\llbracket \langle \alpha \rangle \phi\rrbracket  := \{s \in S  \mid \exists s' \in S, s \xrightarrow{\alpha} s' \text{ et } s' \in \llbracket \phi\rrbracket \}$ ;
      \item $\llbracket [ \alpha ] \phi\rrbracket  := \{s \in S  \mid\forall s' \in S, s \xrightarrow{\alpha} s' \text{ implique } s' \in \llbracket \phi\rrbracket \}$ ;
    \end{itemize}
  \end{defn}

  \begin{nota}
    On note $s \Vdash \phi$ si  $s \in \llbracket \phi\rrbracket$.
  \end{nota}

  \begin{defn}
    Soient $\mathrm{Act}$, $\mathrm{AP}$ et soit $\phi$ une formule $\mathsf{HML}$.
     \begin{enumerate}
       \item Pour $M = (S, \mathrm{Act}, \to, \mathrm{AP}, L)$, on dit que $s \in S$ \textit{\textbf{satisfait}} $\phi$ si $s \Vdash \phi$. 
       \item On dit que $\phi$ est \textit{\textbf{valide}} dans $M = (S, \mathrm{Act}, \to, \mathrm{AP}, L)$, que l'on note $M \models \phi$, si 
          \[
         \forall s \in S, \quad s \Vdash \phi
         .\] 
       \item On dit que $\phi$ est \textit{\textbf{valide}}, que l'on note $\models \phi$, si  $\phi$ est  valide dans tout modèle $M$.
    \end{enumerate}
  \end{defn}

  \begin{exm}[Logique $\mathsf{LML}$.]
    Soit  $\mathrm{Act} = \mathbf{1} = \{\bullet\}$.
    Soit \[
    M((\mathbf{2}^\mathrm{AP})^\omega) := (S, \mathbf{1}, \to, \mathrm{AP}, L)
    \] 
    où
    \begin{itemize}
      \item $S = (\mathbf{2}^\mathrm{AP})^\omega$ ;
      \item pour tout $\sigma$, on a que $\sigma \xrightarrow\bullet \sigma \upharpoonright 1$ ;
      \item $L(\sigma) = \sigma(0)$.
    \end{itemize}
    On a 
    \[
      \sigma \Vdash \langle \bullet \rangle \phi \iff \sigma \upharpoonright 1 \Vdash \phi \iff \phi \Vdash  [\bullet] \phi
    ,\]
    donc $\langle \bullet \rangle$ et $[\bullet]$ correspondent ainsi à la modalité $\nxt$ "next".

    De plus, sur ce système, on a $\sigma \sim s'$ ssi $\sigma = \sigma'$.
  \end{exm}

  \section{Équivalences logiques.}

  \subsection{Équivalences logiques pour les formules.}

  \begin{defn}
    Soient $\mathrm{AP}$, $\mathrm{Act}$, et soient $\phi, \psi$ des formules.
    On note $\phi \equiv \psi$ ssi
     \[
    \forall M = (S, \mathrm{Act}, \to, \mathrm{AP}, L),
    \quad \llbracket \phi = \psi\rrbracket  
    ,\] 
    autrement dit $\forall  M, \forall s \in S, \quad s \Vdash \phi \iff s \Vdash \psi$.
  \end{defn}

  \begin{lem}
    On a que $[\alpha]-$ et  $\langle \alpha\rangle -$ sont duaux par De Morgan :
    \[
      [\alpha] \phi \equiv \lnot \langle \alpha \rangle \lnot \phi
      \quad\text{et}\quad
      \langle\alpha\rangle \phi \equiv \lnot [\alpha] \lnot \phi
    .\]

    De plus, comme $\langle \alpha \rangle -$ est défini comme quantification existentielle, on a 
     \[
    \langle \alpha \rangle (\phi \lor \psi) \equiv \langle \alpha \rangle \phi \lor \langle \alpha \rangle \psi
    \quad\text{et}\quad
    \langle \alpha \rangle \bot \equiv\bot
    .\]
    De même, comme $[\alpha]-$ est défini comme quantification universelle, on a 
     \[
       [\alpha] (\phi \land \psi) \equiv [ \alpha ] \phi \land [ \alpha ] \psi
    \quad\text{et}\quad
      [\alpha] \top \equiv \top
    .\]
  \end{lem}

  \subsection{Équivalences logiques pour les états.}

  \begin{defn}
    Soient $\mathrm{Act}$ et $\mathrm{AP}$.
    On se donne $M_1$, et $M_2$  deux modèles de Kripke.
    Pour $s_1 \in S_1$ et $s_2 \in S_2$, on note \[
    s_1 \equiv s_2 \quad\quad \text{ ssi } \quad\quad \forall \phi, \quad s_1 \Vdash \phi \iff s_2 \Vdash \phi
    .\]
  \end{defn}

  \begin{rmk}
    On aurait pu définir une telle équivalence pour $\mathsf{LML}$, mais on aurait que $\sigma_1 \equiv \sigma_2$ ssi $\sigma_1 = \sigma_2$, ce qui n'est pas très intéressant.
    Avec des modèles de Kripke et la logique $\mathsf{HML}$, on a des états différents qui sont équivalents.
  \end{rmk}

  \begin{thm}
    Si $s_1 \sim s_2$ alors $s_1 \equiv s_2$.
  \end{thm}
  \begin{prv}
    Par induction sur $\phi$, on montre que 
    \[
    s_1 \sim s_2 \implies (s_1 \Vdash \phi \iff s_2 \Vdash \phi)
    .\] 
    \begin{description}
      \item[Cas de $\texttt{a} \in \mathrm{AP}$.]
        Si $s_1 \sim s_2$ alors $L_1(s_1) = L_2(s_2)$ donc $s_1 \Vdash \texttt{a}$ ssi $s_2 \Vdash \texttt{a}$.
      \item[Cas de $-\land-$, $-\lor-$, $\top$, $\bot$, et $\lnot-$.]
        On applique l'hypothèse d'induction et/ou l'hypothèse $s_1 \sim s_2$, et on conclut.
      \item[Cas de $\langle \alpha \rangle -$.]
        Supposons $s_1 \sim s_2$.
        Supposons $s_1 \Vdash \langle \alpha \rangle \phi$.
        Il existe donc $s_1'$ tel que $s_1 \xrightarrow\alpha s_1'$ et $s_1' \Vdash \phi$.
        Par bisimulation
        \[
        \begin{tikzcd}
          s_1 \arrow{d}{\alpha} \arrow[dash]{r}{\sim} & s_2 \arrow[dashed]{d}{\alpha}\\
          s_1' \arrow[dashed,dash]{r}{\sim} & s_2'
        \end{tikzcd}
        ,\] 
        il existe donc $s_2'$ tel que $s_2 \xrightarrow \alpha s_2'$ et $s_1' \sim s_2'$.
        Donc, par hypothèse d'induction, on a que $s_2' \Vdash \phi$.
        On a donc $s_2 \Vdash \langle \alpha \rangle \phi$.
        On procédant de même dans l'autre cas, on a bien que \[
        s_1 \Vdash \langle \alpha \rangle \phi \iff
        s_2 \Vdash \langle \alpha \rangle \phi
        .\]
      \item[Cas de $\lbracket \alpha \rbracket -$.]
        On a que $[\alpha] \phi \equiv \lnot \langle \alpha \rangle \lnot \phi$.
    \end{description}
  \end{prv}

  \section{Propriété de Hennessy-Milner.}

  \begin{rmk}[Question]
    Est-ce que $s_1 \equiv s_2$ implique $s_1 \sim s_2$ ?
    Autrement dit, est-ce que $\equiv$ est une bisimulation ?

    La réponse est \textit{\textbf{non}} en général.
  \end{rmk}

  \begin{exm}
    Considérons les système de transitions suivants.
    \begin{center}
      \begin{tikzpicture}[every state/.style={minimum size=12pt,draw=deepblue,thick,rounded rectangle}]
        \node[state, initial, initial where=above] (A) {$s_1$};
        \node[state, below left of=A] (B) {};
        \draw[->] (A) edge (B);
        \node[state, below of=A] (B) {};
        \node[state, below of=B] (C) {};
        \draw[->] (A) edge (B);
        \draw[->] (B) edge (C);
        \node[state, below right of=A] (B) {};
        \node[state, below right of=B] (C) {};
        \node[state, below right of=C] (D) {};
        \draw[->] (A) edge (B);
        \draw[->] (B) edge (C);
        \draw[->] (C) edge (D);
        \node[right of=B] {$\adots$};
      \end{tikzpicture}
    \end{center}
    \begin{center}
      \begin{tikzpicture}[every state/.style={minimum size=12pt,draw=deepblue,thick,rounded rectangle}]
        \node[state, initial, initial where=above] (A) {$s_2$};
        \node[state, below left of=A] (B) {};
        \draw[->] (A) edge (B);
        \node[state, below of=A] (B) {};
        \node[state, below of=B] (C) {};
        \draw[->] (A) edge (B);
        \draw[->] (B) edge (C);
        \node[state, below right of=A] (B) {};
        \node[state, below right of=B] (C) {};
        \node[state, below right of=C] (D) {};
        \draw[->] (A) edge (B);
        \draw[->] (B) edge (C);
        \draw[->] (C) edge (D);
        \node[right of=B] {$\adots$};
        \node[state, right of=A] (B) {};
        \node[state, right of=B] (C) {};
        \node[state, right of=C] (D) {};
        \node[right of=D] (E) {$\cdots$};
        \draw[->] (A) edge (B);
        \draw[->] (B) edge (C);
        \draw[->] (C) edge (D);
        \draw[->] (D) edge (E);
      \end{tikzpicture}
    \end{center}
    On remarque que $s_1 \nsim s_2$ mais $s_1 \equiv s_2$ (la logique $\mathsf{HML}$ ne peut "voir" qu'à une profondeur finie).
  \end{exm}

  \begin{defn}
    Une classe $\mathfrak{M}$ de modèles a la \textit{\textbf{propriété de Hennessy-Milner}} si, dans $\mathfrak{M}$, l'équivalence  $\equiv$ entre états est une bisimulation.
  \end{defn}

  \begin{defn}
    Soit $K = (S, \mathrm{Act}, \to)$ alors, pour $\alpha \in \mathrm{Act}$ et $s \in S$, on note 
    \[
      \mathrm{Succ}^\alpha(s) := \{s' \in S  \mid s \xrightarrow \alpha s' \} 
    .\]
  \end{defn}

  \begin{defn}
    Un modèle de Kripke $M$ est dit \textit{\textbf{à image finie}} si, pour tout $s \in S$ et $\alpha \in \mathrm{Act}$, l'ensemble $\mathrm{Succ}^\alpha(s)$ des successeurs de $s$ par une action $\alpha$ fixée est finie.
  \end{defn}

  \begin{prop}[Hennessy-Milner]
    La classe des modèles à image finie a la propriété de Hennessy-Milner.
  \end{prop}

  \section{Saturation modale.}

  \begin{defn}
    Soit $M = (S, \mathrm{Act}, \to , \mathrm{AP}, L)$.
    \begin{enumerate}
      \item Soient $T \subseteq S$ et $\Phi \subseteq \mathsf{HML}$.
        On dit que $\Phi$ est \textit{satisfiable} dans $T$ s'il existe $s \in T$ tel que pour tout $ \phi \in \Phi$, on a $s \Vdash \phi$.
      \item Soient $T \subseteq S$ et $\Phi \subseteq \mathsf{HML}$.
        On dit que $\Phi$ est \textit{finiement satisfiable} dans $T$ si, tout $\Psi \subseteq_{\mathrm{fin}} \Phi$ est satisfiable dans $T$, autrement dit 
        \[
        \forall \Psi \subseteq_\mathrm{fin} \Phi
        \quad \exists s \in T 
        \quad s \Vdash \bigwedge \Psi
        .\]
      \item On dit que $M$ est \textit{modalement saturée} si, pour tout $s \in S$, tout $\alpha \in \mathrm{Act}$ et tout $\Phi \subseteq \mathsf{HML}$ tel que $\Phi$ est finiement satisfiable dans $\mathrm{Succ}^\alpha(s)$ alors $\Phi$ est satisfiable dans  $\mathrm{Succ}^\alpha(s)$, autrement dit 
        \begin{gather*}
          \forall s \in S, \forall \alpha \in \mathrm{Act}, \forall \Phi \subseteq \mathsf{HML},\\
          \begin{pmatrix}
            \forall \Psi \subseteq_{\mathrm{fin}} \Phi \quad \exists s' \xleftarrow{\alpha} s \quad s' \Vdash \bigwedge \Psi\\
            \vertical{\implies}\\
            \exists s' \xleftarrow{\alpha} s \quad\forall  \phi \in \Phi \quad s' \Vdash \phi
          \end{pmatrix} 
        .\end{gather*}
    \end{enumerate}
  \end{defn}

  \begin{prop}
    Si $M_1$ et $M_2$ sont modalement saturés alors $\equiv$ entre $S_1$ et $S_2$ est une bisimulation.
  \end{prop}
  \begin{prv}
    Soient $s_1 \equiv s_2$.
    \begin{enumerate}
      \item On a bien $L_1(s_1) = L_2(s_2)$.
      \item Supposons $s_1 \xrightarrow{\alpha}s_1'$.
        On pose $\Phi = \{\phi  \mid s_1' \Vdash \phi\}$.
        Soit $\Psi \subseteq_\mathrm{fin} \Phi$ quelconque.
        On a $s_1 \Vdash \langle \alpha \rangle \bigwedge \Psi$ donc $s_2 \Vdash \langle \alpha \rangle \bigwedge \Psi$.
        Il existe donc $s_2' \xleftarrow{\alpha} s_2$ telle que $s_2' \Vdash \bigwedge \Psi$.
        Par \textit{\textbf{saturation modale}} (de $M_2$), il existe $s_2' \xleftarrow{\alpha} s_2$ tel que, pour tout $\phi \in \Phi$, $s_2' \Vdash \phi$.
        Autrement dit, on a bien :
        \[
        \begin{tikzcd}
          s_1 \arrow{d}{\alpha} \arrow[dash]{r}{\equiv} & s_2 \arrow[dashed]{d}{\alpha}\\
          s_1' \arrow[dashed,dash]{r}{\equiv} & s_2'
        \end{tikzcd}
        .\]
      \item Symétriquement.
    \end{enumerate}
  \end{prv}

  \begin{prop}
    Si $M$ est à image finie alors $M$ est modalement saturé.
  \end{prop}
  \begin{prv}
    Soit $s \in S$, $\alpha \in \mathrm{Act}$, et $\Phi \subseteq \mathsf{HML}$.
    Supposons que $\Phi$ finiement saturé dans $\mathrm{Succ}^\alpha(s)$.
    Par l'absurde, supposons 
    \[
      \forall t \xrightarrow{\alpha} s \quad \exists \phi_t \in \Phi \quad t \nVdash \phi_t
    .\]
    On pose 
    \[
      \Psi := \{\phi_t  \mid t \in \mathrm{Succ}^\alpha(s)\} \subseteq_\mathrm{fin} \Phi
    .\]
    Il existe donc $t \xleftarrow{\alpha} s$ tel que  $t \Vdash \bigwedge \Psi$, donc $t \Vdash \phi_t$.
    \textit{\textbf{Absurde}}.
  \end{prv}

  \section{Algèbres de Boole avec opérateurs.}

  Soit $\mathfrak{L}(\mathsf{HML}) = \{[\phi]_\equiv  \mid \phi \in \mathsf{HML}\}$.
  On notera dans la suite $\phi$ pour $[\phi]_\equiv$.
  Comme pour le \textit{Homework}, on a que $(\mathfrak L(\mathsf{HML}), \le)$ est une algèbre de Boole où $\phi \le \psi$ ssi $\phi \land \psi \equiv \phi$.

  Pour un modèle de Kripke $M$, on a que $(\wp(M), \subseteq)$ est une algèbre de Boole et 
  \begin{align*}
    \llbracket -\rrbracket  : \mathfrak L(\mathsf{HML}) &\longrightarrow \wp(S) \\
    [\phi]_\equiv &\longmapsto \llbracket \phi\rrbracket  
  \end{align*}
  est un morphisme d'algèbres de Boole.

  \begin{lem}
    Soit $M$ un modèle de Kripke.
    On a que 
    \begin{align*}
      \llbracket \langle \alpha \rangle\rrbracket : \wp(S) &\longrightarrow \wp(S) \\
      A &\longmapsto \{s  \mid \exists s' \xleftarrow{\alpha} s \quad s' \in A\}
    \end{align*}
    est un morphisme de $\vee$-semi-treillis,
    et on a $\llbracket \langle \alpha \rangle \phi \rrbracket = \llbracket \langle \alpha \rangle \rrbracket (\llbracket \phi\rrbracket)$.

    De même, on a que 
    \begin{align*}
      \llbracket [\alpha]\rrbracket : \wp(S) &\longrightarrow \wp(S) \\
      A &\longmapsto \{s  \mid \forall s' \xleftarrow{\alpha} s \quad s' \in A\}
    \end{align*}
    est un morphisme de $\wedge$-semi-treillis,
    et on a $\llbracket [\alpha] \phi \rrbracket = \llbracket [\alpha] \rrbracket (\llbracket \phi\rrbracket)$.
    \qed
  \end{lem}

  \begin{lem}
    On a que
    \begin{align*}
      \langle \alpha \rangle: \mathfrak L(\mathsf{HML}) &\longrightarrow \mathfrak L(\mathsf{HML}) \\
      [\phi]_\equiv &\longmapsto [\langle \alpha \rangle \phi]_\equiv
    \end{align*}
    est un morphisme de $\vee$-semi-treillis
    et 
    \begin{align*}
      [\alpha]: \mathfrak L(\mathsf{HML}) &\longrightarrow \mathfrak L(\mathsf{HML}) \\
      [\phi]_\equiv &\longmapsto [[\alpha] \phi]_\equiv
    \end{align*}
    est un morphisme de $\wedge$-semi-treillis

    On a $\llbracket [\alpha]\rrbracket = \llbracket \langle \alpha \rangle\rrbracket^\partial$ et $[\alpha] = \langle \alpha \rangle^\partial$ avec la notation définie ci-après.
  \end{lem}

  \begin{defn}
    Pour $f : B \to B'$ où $B, B'$ sont deux algèbres de Boole, on note
    \begin{align*}
      f^\partial: B &\longrightarrow B' \\
      a &\longmapsto \lnot' f(\lnot a)
    .\end{align*}
    C'est le dual par De Morgan de $f$.
  \end{defn}

  \begin{lem}
    Pour $f : B \to B'$ où $B, B'$ sont deux algèbres de Boole,
    \begin{enumerate}
      \item on a $(f^\partial)^\partial  = f$ ;
      \item si $f$ est un morphisme de $\vee$-semi-treillis alors  $f^\partial $ est un morphisme de $\wedge$-semi-treillis ;
      \item si $f$ est un morphisme de $\wedge$-semi-treillis alors  $f^\partial $ est un morphisme de $\vee$-semi-treillis ;
      \item si $f$ est un morphisme de treillis alors $f^\partial = f$.
    \end{enumerate}
  \end{lem}

  \begin{rmk}
    Dans $\mathsf{LML}$, on a une modalité $\nxt$ qui est auto duale car morphisme de treillis.
  \end{rmk}

  \begin{defn}[BAO]
    Une \textit{algèbre de Boole avec opérateurs} (BAO) est de la forme 
    \[
      (B, \le, (f_\alpha)_{\alpha \in \mathrm{Act}})
    \] 
    où $(B, \le)$ est une algèbre de Boole et $f_\alpha : B \to B$ est un morphisme de $\vee$-semi-treillis pour tout  $\alpha \in \mathrm{Act}$.
  \end{defn}

  \begin{exm}
    On a que $\mathfrak{L}(\mathsf{HML})^+ := (\mathfrak{L}(\mathsf{HML}), \le , (\langle \alpha \rangle)_{\alpha \in \mathrm{Act}})$ est une algèbre de Boole avec opérateurs.
  \end{exm}

  \begin{exm}
    Pour $L = (S, \mathrm{Act}, \to)$ une structure de Kripke, alors 
    \[
    K^+ := \big(\wp(S), \subseteq, (\llbracket \langle \alpha \rangle \rrbracket)_{\alpha \in \mathrm{Act}}\big) 
    \] est une algèbre de Boole avec opérateurs.
  \end{exm}

  \section{Structures ultrafiltres.}

  \begin{defn}
    Soit $(B, \le , (f_\alpha)_{\alpha \in \mathrm{Act}})$ une algèbre de Boole avec opérateurs.
    On définit 
    \[
      \mathfrak{Uf}(B) := (\mathbf{Sp}(B), \mathrm{Act}, \to)
    ,\] 
    où $\mathcal{F} \xrightarrow{\alpha} \mathcal{H}$ ssi pour tout $b \in B$, $b \in \mathcal{H}$ implique $f_\alpha(b) \in \mathcal{F}$.
  \end{defn}

  \begin{lem}
    On a $\mathcal{F} \xrightarrow{\alpha} \mathcal{H}$ ssi pour tout $b \in B$, $f_\alpha^\partial (b) \in \mathcal{F}$ implique $b \in \mathcal{H}$. 
  \end{lem}
  \begin{prv}
    (Idée)
    Supposons $\mathcal{F} \xrightarrow{\alpha} \mathcal{H}$.
    Soit $b \in B$.
    Par contraposée, supposons $b \not\in \mathcal{H}$.
    On a donc $\lnot b \in \mathcal{H}$ et donc $f_\alpha(\lnot b) \in \mathcal{F}$.
    Mais, $f_\alpha(\lnot b) = \lnot f_\alpha^\partial (b)$ et donc $f_\alpha^\partial (b) \not\in \mathcal{F}$.

    Réciproquement, supposons $\mathcal{F} \not\xrightarrow{\alpha} \mathcal{H}$.
    Il existe donc $b \in B$ tel que $b \in \mathcal{H}$ et $f_\alpha(b) \not\in \mathcal{F}$.
    Donc $\lnot f_\alpha(b) = f_\alpha^\partial (\lnot b) \in \mathcal{F}$ et $\lnot b \not\in \mathcal{H}$.
  \end{prv}

  \subsection{Extensions d'ultrafiltres aux modèles de Kripke.}

  Soit $M = (S, \mathrm{Act}, \to , \mathrm{AP}, L)$.

  \begin{rmk}
    Soit $X$ un ensemble.
    \begin{itemize}
      \item Un \textit{filtre} (\textit{propre}) sur $X$ est un filtre (propre) sur $(\wp(X), \subseteq)$.
      \item Un \textit{ultrafiltre} sur $X$ est un ultrafiltre sur $(\wp(X), \subseteq)$.
      \item Si $G \subseteq \wp(X)$ qui a la propriété des intersections finies (\textit{finite intersection property}) alors 
        \[
        \bigcap \mleft\{\,E \;\middle|\; E \text{ filtre propre  et } E \supseteq G\,\mright\}  
        \] est un filtre propre.
      \item Si $G \subseteq \wp(X)$ a la propriété des intersections finies alors il existe un ultrafiltre $\mathcal{F}$ sur $X$ tel que $G \subseteq \mathcal{F}$.
      \item Pour tout $x \in X$, on définit 
        \[
        \pi(x) := \{A \in \wp(X)  \mid x \in A\}
        \]
        est l'\textit{ultrafiltre principal} induit par $x$.
        On a que
        \[
        \pi : X \to \mathfrak{Uf}(X) =: \mathbf{Sp}(\wp(X), \subseteq)
        .\]
    \end{itemize}
  \end{rmk}

  \begin{rmk}
    La fonction de \textit{labelling} $L : S \to \mathbf{2}^\mathrm{AP}$ peut être décrite par $V : \mathrm{AP} \to \mathbf{2}^S$.
  \end{rmk}

  \begin{defn}
    Soit $M = (S, \mathrm{Act}, \to, \mathrm{AP}, L)$ un modèle de Kripke.
    L'\textit{extension ultrafiltre} de $M$ est 
    \[
      \mathfrak{Uf}(M) = (\underbrace{\mathfrak{Uf}(S), \mathrm{Act}, \to}_{(S, \mathrm{Act}, \to)^+}, \mathrm{AP}, L')
    ,\]
    où $L' : \mathfrak{Uf}(S) \to \mathbf{2}^\mathrm{AP}$ est générée par 
    \begin{align*}
      V': \mathrm{AP} &\longrightarrow \mathbf{2}^{\mathfrak{Uf}(S)} \\
      \texttt{a} &\longmapsto \{\mathcal{F}  \mid V(\texttt{a}) \in \mathcal{F}\} 
    .\end{align*}
  \end{defn}

  \begin{rmk}
    Dans $\mathfrak{Uf}(M)$ les propriétés suivantes sont équivalentes :
    \begin{itemize}
      \item $\mathcal{F} \xrightarrow{\alpha} \mathcal{H}$
      \item pour tout $A \in \wp(S)$, $A \in \mathcal{H}$ implique $\llbracket \langle\alpha \rangle\rrbracket (A) \in \mathcal{F}$ ;
      \item pour tout $A \in \wp(S)$, $\llbracket [\alpha]\rrbracket (A) \in \mathcal{F} $ implique $A \in \mathcal{H}$.
    \end{itemize}
  \end{rmk}

  \begin{rmk}
    Si $M$ est \textit{\textbf{fini}} alors
    \begin{itemize}
      \item $\pi : S \to \mathfrak{Uf}(S)$ est une bijection ;
      \item $\pi(s) \xrightarrow{\alpha} \pi(t)$ ssi $s \xrightarrow{\alpha} t$ ;
      \item  $L'(\pi(s)) = L(s)$.
    \end{itemize}
  \end{rmk}

  \begin{prop}
    Pour tout $\mathcal{F} \in \mathfrak{Uf}(S)$, et tout $\phi$ une formule $\mathsf{HML}$, on a 
    \[
      \underbrace{\llbracket \phi\rrbracket  \in \mathcal{F}}_{\text{\clap{dans $M$}}}
      \iff
      \underbrace{\mathcal{F} \Vdash \phi}_{\text{\clap{dans $\mathfrak{Uf}(M)$}}}
    .\]
  \end{prop}

  \begin{prv}
    On procède par induction sur $\phi$.
    \begin{description}
      \item[Cas $\texttt{a} \in \mathrm{AP}$.]
        Par définition de $\mathfrak{Uf}(M)$, on a  $\mathcal{F} \Vdash \texttt{a}$ ssi $\llbracket \texttt{a}\rrbracket \in \mathcal{F}$.
      \item[Cas $-\lor-$,  $-\land-$,  $\lnot-$,  $\top$,  $\bot$.]
        Vu dans le  \textit{Homework}, partie II.
      \item[Cas $\langle \alpha \rangle -$.]
        On montre que $\mathcal{F} \Vdash \langle \alpha \rangle \phi \iff \llbracket \langle \alpha \rangle\rrbracket (\llbracket \phi\rrbracket) \in \mathcal{F}$.
        \begin{itemize}
          \item "$\implies$".
            Si $\mathcal{F} \xrightarrow{\alpha} \mathcal{H}$ tel que $\mathcal{H} \Vdash \phi$ alors, par hypothèse d'induction $\llbracket \phi\rrbracket \in \mathcal{H} $ et donc $\llbracket \langle \alpha \rangle\rrbracket (\llbracket \phi\rrbracket) \in \mathcal{F}$.
          \item "$\impliedby$".
            Supposons $\llbracket \langle \alpha \rangle\rrbracket (\llbracket \phi\rrbracket) \in \mathcal{F}$.
            On va montrer qu'il existe $\mathcal{H}$ tel que $\mathcal{F} \xrightarrow{\alpha} \mathcal{H}$ et $\llbracket \phi\rrbracket \in \mathcal{H}$.
            On va utiliser le lemme de l'ultrafiltre.
            Soit 
            \[
            H := \mleft\{\,A \cap \llbracket \phi\rrbracket  \;\middle|\; 
            \begin{array}{c}
              A \in \wp(S)\\
              \text{ et }\\
              \llbracket [\alpha]\rrbracket (A) \in \mathcal{F} 
            \end{array}\,\mright\} 
            .\]
            \begin{enumerate}
              \item Remarquons que $H$ est stable par intersections binaires : si $A_1 \cap \llbracket \phi\rrbracket, A_2 \cap \llbracket \phi\rrbracket \in H $ alors $(A_1 \cap A_2) \cap \llbracket \phi\rrbracket \in H$ car $\llbracket [\alpha]\rrbracket (A_1),  \llbracket [\alpha]\rrbracket (A_2) \in \mathcal{F}$ et donc \[
              \llbracket [\alpha]\rrbracket (A_1 \cap A_2) = \llbracket [\alpha]\rrbracket (A_1)  \cap \llbracket [\alpha]\rrbracket (A_2) \in \mathcal{F}
              .\]

              \item De plus, on a que $\emptyset \not\in H$. En effet, soit $A \cap \llbracket \phi\rrbracket \in H$ alors 
              \[
                \llbracket \langle \alpha \rangle \phi\rrbracket  \cap  \llbracket [\alpha] \rrbracket (A) \neq \emptyset 
              \] (intersection de deux éléments de $\mathcal{F}$), il existe donc $s \in S$ tel que $s \Vdash \langle \alpha \rangle \phi$ et pour tout $s' \xleftarrow{\alpha} s$, on a  $s' \in A$.
              Donc $A \neq \emptyset$ (car il existe $s' \xleftarrow{\alpha} s$).
              \item On a  $S \cap \llbracket \phi\rrbracket \in H $ car $\llbracket [\alpha]\rrbracket (S) = S \in \mathcal{F}$.
            \end{enumerate}
            Donc $H$ a la propriété des intersections finies, et par le lemme de l'ultrafiltre, il existe $\mathcal{H} \in \mathfrak{Uf}(S)$ tel que $\mathcal{H} \supseteq H$.
            De plus, on a $\mathcal{F} \xrightarrow{\alpha} \mathcal{H}$ car, pour tout $A \in \wp(S)$, 
            $\llbracket [\alpha]\rrbracket (A) \in \mathcal{F} $ implique $A \cap \llbracket \phi\rrbracket \in H $ qui implique $A \in \mathcal{H}$.
            On a aussi $\llbracket \phi\rrbracket \in \mathcal{H} $ car $S \cap \llbracket \phi\rrbracket  \in \mathcal{H}$.
            Ainsi, par hypothèse d'induction, on a $\mathcal{H} \Vdash \phi$ et donc $\mathcal{F} \Vdash \langle \alpha \rangle \phi$.
        \end{itemize}
      \item[Cas $\lbracket \alpha \rbracket -$.] Par dualité.
    \end{description}
  \end{prv}

  \begin{crlr}
    Soit $\phi$ une formule $\mathsf{HML}$.
    Pour tout $s \in S$, on a
    \[
    s \Vdash \phi \iff s \in \llbracket \phi\rrbracket  \iff \llbracket \phi\rrbracket \in \pi(s) 
    .\]
    De plus,  \[
      M \models \phi \iff \llbracket \phi\rrbracket = S \iff \mathfrak{Uf} (M) \models \phi
    .\] 
  \end{crlr}

  \begin{prop}
    L'extension ultrafiltre $\mathfrak{Uf}(M)$ de $M$ est modalement saturée.
    En particulier, étant donnés $M_1$ et $M_2$ deux modèles de Kripke, on a 
    \[
    s_1 \equiv s_2 \iff \pi(s_1) \equiv \pi(s_2)
    \iff \pi(s_1) \sim_{\mathfrak{Uf}(M)} \pi(s_2)
    .\] 
  \end{prop}

\end{document}
