\documentclass[fontsize=16pt,a4paper,DIV=17,parskip=half]{scrartcl}

\usepackage[utf8]{inputenc}
\usepackage[dvipsnames]{xcolor}
\usepackage[hyperindex]{hyperref}
\usepackage{lastpage}
\usepackage{tikz}
\usepackage{tikzpagenodes}
\usepackage{pgfplots}
\usepackage{enumitem}
\usepackage{scrlayer-scrpage}
\usepackage{xspace}
\usepackage{float}
\usepackage{amsfonts,amsmath,amsthm,amssymb}
\usepackage{thmtools}
\usepackage[english]{babel}
\usepackage[autostyle, style=english]{csquotes}
\usepackage{subfigure}
\usepackage{mleftright}
%\usepackage{BOONDOX-calo}
%\usepackage{dsfont}
\usepackage{tikz-cd}
\usepackage[framemethod=TikZ]{mdframed}
\usepackage{soulutf8}
\usepackage{mathtools}
\usepackage{multicol}
\usepackage{stmaryrd}
\usepackage{fvextra}
\usepackage{adjustbox}
\usepackage{fontspec}
\usepackage{etoolbox}
\usepackage{todonotes}
\usepackage{verbatim}
\usepackage{ebproof}
\usepackage{cancel}
\usepackage{setspace}
\usepackage{fourier-otf}
%\usepackage{pxfonts}
%\usepackage[scaled=0.92]{mathpazo}
\usepackage{juliamono}
\usepackage[osf]{Alegreya}
\usepackage[osf]{AlegreyaSans}
\let\mathds\mathbb
%\DeclareMathAlphabet{\mathsf}{OT1}{cmss}{m}{n}

\renewcommand{\mathsf}[1]{\textup{\textsf{#1}}}
\newcommand\Agda{\textsf{Agda}\xspace}

\ebproofset{right label template=$\inserttext$, left label template=\tiny$\inserttext$, center=false}

\RedeclareSectionCommand[beforeskip=0.10em, afterskip=0.10em]{section}
\RedeclareSectionCommand[beforeskip=0.05em, afterskip=0.001em plus 0em]{subsection}


\fvset{bgcolor=lightgray!10,backgroundcolorpadding=3pt}

\MakeOuterQuote{"}

\colorlet{deeppurple}{DarkOrchid}
\colorlet{deepgreen}{ForestGreen!70!black}
\colorlet{deepblue}{NavyBlue!70!black}
\colorlet{deepred}{RawSienna!70!black}
\colorlet{nicered}{BrickRed!70!white}

\makeatletter
\g@addto@macro\bfseries{\boldmath}
\makeatother

\hypersetup{
    colorlinks,
    citecolor=deepgreen,
    filecolor=nicered,
    linkcolor=deepblue,
    pdfencoding=auto,
    psdextra,
    urlcolor=deepred
}

\usetikzlibrary{positioning,shadings,arrows.meta}

\clearpairofpagestyles

\newcommand\showpage{\itshape\hfill--~\thepage/\pageref*{LastPage}~--\hfill}
\cofoot[\showpage]{\showpage} \cefoot[\showpage]{\showpage}

\setlist[enumerate]{font={\bfseries\color{deepblue}}}
\AtBeginDocument{
  \renewcommand{\labelitemi}{\bfseries\color{deepblue}$\triangleright$}
  \renewcommand{\labelitemii}{\bfseries\color{deepblue}–}
  \renewcommand{\labelitemiii}{\bfseries\color{deepblue}•}
}

\newcommand\separatorBlock{
  \raisebox{-0.2em}{
    \tikz{ \draw[deepblue,ultra thick, line cap=round] (0,0) -- (0,1em); }
  }
}

\newcommand\vertical[1]{
  \rotatebox[origin=c]{270}{\ensuremath{#1}}
}

\mdfsetup{skipabove=1em,skipbelow=0em,linewidth=0pt,rightline=false, topline=false, bottomline=false}


\theoremstyle{definition}

\declaretheoremstyle[
  headfont=\bfseries\sffamily\color{deepgreen}, bodyfont=\normalfont,
% mdframed={
%   linecolor=ForestGreen, % backgroundcolor=ForestGreen!5,
% },
]{thmgreenbox}

\declaretheoremstyle[
  headfont=\bfseries\sffamily\color{deepblue}, bodyfont=\normalfont\sffamily\itshape,
% mdframed={
%   linecolor=NavyBlue,% backgroundcolor=NavyBlue!5,
% },
]{thmbluebox}

\declaretheoremstyle[
  headfont=\bfseries\sffamily\color{deepblue}, bodyfont=\normalfont,
  mdframed={
    linecolor=deepblue,
    linewidth=2pt,
  },
  numbered=no,
]{thmblueline}

\declaretheoremstyle[
  headfont=\bfseries\sffamily\color{deepred}, bodyfont=\normalfont,
% mdframed={
%   linecolor=RawSienna,% backgroundcolor=RawSienna!5,
% },
]{thmredbox}

\declaretheoremstyle[
  headfont=\itshape\sffamily\color{deepred}, bodyfont=\normalfont,
% mdframed={
%   linecolor=RawSienna,% backgroundcolor=RawSienna!1,
%   linewidth=0pt,
% },
  numbered=no,
  qed=\qedsymbol,
]{thmproofbox}

\newcommand\defineMarkerColor[2]{
  \AtBeginEnvironment{#1}{
    \setlist[enumerate]{font={\color{#2}}}
    \renewcommand{\labelitemi}{\color{#2}\small$\triangleright$}
    \renewcommand{\labelitemii}{\color{#2}–}
    \renewcommand{\labelitemiii}{\color{#2}•}
    \renewcommand\emph[1]{{\bfseries\em\color{#2}##1}}
  }
}


\AtBeginDocument{
  \setlist[enumerate]{font={\bfseries\color{deepblue}}}
  \renewcommand{\labelitemi}{\bfseries\color{deepblue}\small$\triangleright$}
  \renewcommand{\labelitemii}{\bfseries\color{deepblue}–}
  \renewcommand{\labelitemiii}{\bfseries\color{deepblue}•}
}


\setlist[enumerate]{font={\color{deepblue}}}
\renewcommand{\labelitemi}{\color{deepblue}\small$\triangleright$}
\renewcommand{\labelitemii}{\color{deepblue}–}
\renewcommand{\labelitemiii}{\color{deepblue}•}

\declaretheorem[style=thmgreenbox, name=Axiom, numbered=no]{axi} \defineMarkerColor{axi}{deepgreen}
\declaretheorem[style=thmgreenbox, name=Definition]{defn} \defineMarkerColor{defn}{deepgreen}
\declaretheorem[style=thmbluebox, name=Example]{exm}      \defineMarkerColor{exm}{deepblue}
\declaretheorem[style=thmbluebox, name=Exercise]{exo}     \defineMarkerColor{exo}{deepblue}
\declaretheorem[style=thmbluebox, name=Question]{que}     \defineMarkerColor{que}{deepblue}
\declaretheorem[style=thmredbox, name=Proposition]{prop}  \defineMarkerColor{prop}{deepred}
\declaretheorem[style=thmredbox, name=Theorem]{thm}      \defineMarkerColor{thm}{deepred}
\declaretheorem[style=thmredbox, name=Lemma]{lem}         \defineMarkerColor{lem}{deepred}
\declaretheorem[style=thmredbox, name=Corollary]{crlr}   \defineMarkerColor{crlr}{deepred}
\declaretheorem[style=thmblueline, name=Remark]{rmk}    \defineMarkerColor{rmk}{deepblue}
\declaretheorem[style=thmblueline, name=Note]{note}       \defineMarkerColor{note}{deepblue}
\declaretheorem[style=thmproofbox, name=Proof]{replacementproof}
\newenvironment{prv}[1][\proofname]{\vspace{-12pt}%
\begin{replacementproof}}{\end{replacementproof}} \defineMarkerColor{prv}{deepred}
\declaretheorem[style=thmproofbox, name=Proof idea]{replacementideaproof}
\newenvironment{prvid}[1][\proofname]{\vspace{-12pt}%
\begin{replacementideaproof}}{\end{replacementideaproof}} \defineMarkerColor{prvid}{deepred}

\RequirePackage{caption}
\DeclareCaptionLabelFormat{labelformat}{\textbf{#1~#2}\separatorBlock}
\captionsetup{labelformat=labelformat,labelsep=none,textfont=sl}

\DeclareMathSizes{11}{9}{7}{5}

\title{Semantics and Verification -- \textit{Homework}}
\author{Hugo \textsc{Salou}}

\let\emph\relax
\DeclareTextFontCommand{\emph}{\bfseries\em\color{deepblue}}

\renewcommand{\thefootnote}{\alph{footnote}}

\makeatletter
\def\moverlay{\mathpalette\mov@rlay}
\def\mov@rlay#1#2{\leavevmode\vtop{%
   \baselineskip\z@skip \lineskiplimit-\maxdimen
   \ialign{\hfil$\m@th#1##$\hfil\cr#2\crcr}}}
\newcommand{\charfusion}[3][\mathord]{
    #1{\ifx#1\mathop\vphantom{#2}\fi
        \mathpalette\mov@rlay{#2\cr#3}
      }
    \ifx#1\mathop\expandafter\displaylimits\fi}
\makeatother

\usepackage{pifont}


\tikzcdset{arrow style=math font}
\tikzset{
  equiv/.style={-,preaction={draw,double equal sign distance}},
  >=Straight Barb,
}

\begin{document}
  \begin{center}
    \bfseries
    \sffamily

    {\large\itshape ---\hspace{1em}Homework\hspace{1em}---}

    {\huge Semantics and Verification}

    {\large \itshape Hugo SALOU}
  \end{center}

  \section{Toward Stone Duality.}

  \begin{que}
    Show that every Stone space $(X, \Omega)$ is Hausdorff (if $x, y \in X$ are distinct, there there are disjoint $U, V \in \Omega$ such that $x \in U$ and $y \in V$).
  \end{que}

  Let $x, y \in X$ be two distinct points of a Stone space $(X, \Omega)$.
  As, $(X, \Omega)$ is $\mathrm{T_0}$ and without loss of generality, there exists $W \in \Omega$ such that $x \in W$ and $y \not\in W$.
  As $(X, \Omega)$ is zero-dimensional, we can write $W \eqqcolon \bigcup_{i \in  I} W_i$ where $W_i \in \mathbf{K}\Omega$ for every $i \in I$.
  Thus, there exists a clopen set $U \coloneqq W_i \in \Omega$ such that ${x \in W_i \subseteq U}$.
  Define $V \coloneqq X \setminus U \in \Omega$, and we have that $x \in U$, $y \in V$ (as $y \not\in W \supseteq U$) and the open sets $U$ and $V$ are disjoint.
  We can conclude that every Stone space is Hausdorff.

  \begin{que}
    \label{q2}
    Show that $\le$ is a partial order on $\mathfrak L(\mathsf{LML})$.
  \end{que}

  We start by showing the following lemma.
  \begin{lem}
    \label{lem1}
    We have $\phi \le \psi$ if and only if $\llbracket \phi\rrbracket \subseteq \llbracket \psi\rrbracket$.
  \end{lem}
  \begin{prv}
    We have that $\phi \le \psi$ iff $\phi \equiv \phi \land \psi$ iff $\llbracket \phi\rrbracket = \llbracket \phi \land \psi\rrbracket = \llbracket \phi\rrbracket \cap \llbracket \psi\rrbracket$ (that last equality is by definition of $\llbracket -\rrbracket$) iff $\llbracket \phi\rrbracket \subseteq \llbracket \psi\rrbracket$.
  \end{prv}

  We can thus easily show that $\le$ is a partial order.
  \begin{itemize}
    \item \textit{Reflexivity}. As $\llbracket \phi\rrbracket \subseteq \llbracket \phi\rrbracket$, we have that $\phi \le \phi$ for every $\phi \in \mathfrak L(\mathsf{LML})$.
    \item \textit{Transitivity}. For any $\phi, \psi, \vartheta \in \mathfrak L(\mathsf{LML})$, if $\phi \le \psi$ and $\psi \le \vartheta$ then, by the lemma, $\llbracket \phi\rrbracket \subseteq \llbracket \psi\rrbracket \subseteq \llbracket \vartheta\rrbracket$, thus we have $\llbracket \phi \rrbracket \subseteq \llbracket \vartheta\rrbracket$, \textit{i.e.}\ $\phi \le \vartheta$.
    \item \textit{Antisymmetry}.
      For any $\phi, \psi \in \mathfrak L(\mathsf{LML})$, if $\phi \le \psi$ and $\psi \le \phi$ then, by double inclusion with the above lemma, $\llbracket \phi\rrbracket = \llbracket \psi\rrbracket$ thus $\phi = \psi$ as we consider $\mathsf{LML}$-formulae quotiented by $\equiv$.
  \end{itemize}

  \section{Lattices and Boolean Algebras.}

  \subsection{Semilattices.}

  \begin{que}
    \label{q3}
    Let $(L, \le)$ be a partial order.
    \begin{enumerate}
      \item Show that $(L, \le)$ is a meet semilattice if, and only if, $L$ has binary meets $\wedge : {L \times L} \to L$ and greatest element $\top \in L$.
        \label{q3-1}
      \item Show that $(L, \le)$ is a join semilattice if, and only if, $L$ has binary joins $\vee : {L \times L} \to L$ and least element $\bot \in L$.
    \end{enumerate}
  \end{que}

  \begin{enumerate}
    \item If $(L, \le)$ is a meet semilattice, then $L$ has binary meets and a greatest element $\top = \bigwedge \emptyset$ (any element is a lower bound of $\emptyset$, thus the greatest lower bound of $\emptyset$ is the greatest element).

      Now, suppose $(L, \le)$ has a binary meet $\wedge$ and a greatest element $\top$.
      Consider $\{a_i  \mid i \in I\}$ a finite subset of elements of $L$.
      By induction on~$\# I \in \mathds{N}$, we define $\bigwedge_{i \in I} a_i \in I$ and show that $\bigwedge_{i \in I}a_i$ is a meet of the finite set $\{a_i  \mid  i \in I\}$ (like the notation suggests).
      \begin{itemize}
        \item Define $\bigwedge_{i \in \emptyset} a_i \coloneqq \top \in L$; as any element is a lower bound of $\emptyset$, the greatest lower bound of $\emptyset$ is the greatest element.
        \item Consider $I \coloneqq J \sqcup \{i\}$.
          By induction hypothesis, we have that $\bigwedge_{j \in J} a_j$ exists in $L$ and is a meet of $\{a_j  \mid j \in J\}$ in $(L, \le)$.
          Define \[
          \bigwedge_{k \in I} a_k \coloneqq \big(\bigwedge_{j \in J} a_j \big) \wedge a_i \in I
          .\]

          We have that $\bigwedge_{k \in I} a_k$ is a lower bound of $\{a_k  \mid k \in I\}$. Consider an element $a_k$ with $k \in I$.
          If $k \in J$ then $a_k \le \bigwedge_{j \in J}a_j \le \bigwedge_{k' \in I} a_{k'}$.
          Otherwise~$k = i$ and we immediacy have that $a_i \le \bigwedge_{k' \in I} a_{k'}$.

          Consider a lower bound $b \in L$ of $\{a_k  \mid k \in I\}$, then $b$ is a lower bound of $\{a_j  \mid j \in J\}$ and $b \le a_i$.
          We have $b \le \bigwedge_{j \in J} a_j$ and $b \le a_i$, therefore $b \le \bigwedge_{k \in I} a_k$.

          We can conclude that $\bigwedge_{k \in I} a_k$ is a meet of $\{a_k  \mid k \in K\}$.
      \end{itemize}
      Finally, we have that $(L, \le)$ has finite meets.
    \item This results follows from~\ref{q3-1} when considering the partial order $(L, \ge)$, by duality.
      Meets in $(L, \ge)$ are exactly joins in $(L, \le)$, and the greatest element of $(L, \ge)$ is the least element of $(L, \le)$, and \textit{vice versa}.
  \end{enumerate}

  \begin{note}
    In the following, when I will be dealing with multiple partial orders on the same set (\textit{e.g.}\ $\le$ and $\ge$), I will write $\bigwedge_\le$ for the meet operator in poset $(I, \le)$, $\bigvee_\le$ for the join operator in poset $(I, \le)$, $\top_{\le}$ for the greatest element in poset $(I, \le)$ and $\bot_{\le}$ for the least element in poset $(I, \le)$.
  \end{note}

  \begin{que}
    Prove the following.
    \begin{enumerate}
      \item Let $(L, \le)$ be a meet semilattice with binary meets $\wedge : L \times L \to L$ and greatest element $\top \in L$.
        Then $(L, \wedge, \top)$ is a commutative mooned in which every element is idempotent.
        Moreover, we have $a \le b$ iff $a = a \wedge b$.
      \item Let $(L, \le)$ be a join semilattice with binary joins $\vee : L \times L \to L$ and least element~$\bot$.
        Then $(L, \vee, \bot)$ is a commutative mooned in which every element is idempotent.
        Moreover, we have $a \le b$ iff $b = a \wedge b$.
    \end{enumerate}
  \end{que}

  \begin{enumerate}
    \item Let $a, b, c \in L$. First, we have that $a \wedge b = \bigwedge \{a,b\} = \bigwedge \{b,a\} = b \wedge a$ thus the binary meet operation $\wedge$ is commutative.
      Then, as a special case of the previous question, we have that $a$ and $\top \wedge a$ are both meets of $\{a\}$. And, by unicity of meets (\textit{i.e.}\ antisymmetry of $\le$, mainly), they are equals.
      Also as a special case of the previous question, we have that elements \[
      a \wedge (b \wedge c) = \top \wedge (a \wedge (b \wedge c))
      \] and \[
      (a \wedge b) \wedge c = c \wedge (a \wedge b) = \top \wedge (c \wedge (a \wedge b))
      \]  are both meets of the set $\{a,b,c\}$, thus are equal.
      Next, we have that \[
      a \wedge a = \bigwedge \{a, a\} = \bigwedge \{a\} = \top \wedge a = a
      \](penultimate equality is from last question), thus~$a \wedge a = a$.
      Finally, we have that:
      \begin{itemize}
        \item if $a = a \wedge b$ then $a$ is a lower bound of $\{a,b\}$ thus $a \le b$;
        \item if $a \le b$ then $a = a \wedge b$ as $a$ is a lower bound of $\{a,b\}$ and any lower bound $c$ of $\{a,b\}$ must satisfy $c \le a$.
      \end{itemize}
    \item Consider the meet semilattice $(L, \ge)$ and apply the results above.
      Meets in $(L, \ge)$ are exactly joins in $(L, \le)$, and the greatest element of $(L, \ge)$ is the least element of $(L, \le)$, and \textit{vice versa}.
      The last statement follows from the equivalence:
      \[
      a \le b \quad\text{iff}\quad b\ge a\quad\text{iff}\quad b = a \wedge_\ge b\quad\text{iff}\quad b = a \vee_\le b
      ,\] 
      where the second "iff" follows from the result above for $(L, \ge)$, and the last one follows from the equality $a \wedge_\ge b = a \vee_\le b$.
  \end{enumerate}

  \begin{que}
    Prove the following.
    \begin{enumerate}
      \item Given a commutative monoid $(L, \wedge, \top)$ in which every element is idempotent, let ${a \le_\wedge b}$ iff  $a = a \wedge b$.
        Then $(L, \le_\wedge)$ is a meet semilattice with binary meets given by $\wedge$ and greatest element $\top$.
      \item Given a commutative monoid $(L, \vee, \bot)$ in which every element is idempotent, let ${a \le_\vee b}$ iff  $b = a \vee b$.
        Then $(L, \le_\vee)$ is a join semilattice with binary joins given by $\vee$ and least element $\bot$.
    \end{enumerate}
  \end{que}

  \begin{enumerate}
    \item Let us start by showing that $(L, \le_\wedge)$ is a partial order.
      \begin{itemize}
        \item \textit{Reflexivity.} As $a \wedge a = a$ by idempotence, we have $a \le_\wedge a$.
        \item \textit{Antisymmetry.} If $a \le_\wedge b$ and $b \le_\wedge a$ then, by commutativity, we have $a \wedge b = a = b$.
        \item \textit{Transitivity.} If $a \le_\wedge b$ and $b \le_\wedge c$ then, by associativity,
          \[
          a = a \wedge b = a \wedge (b \wedge c) = (a \wedge b) \wedge c = a \wedge c
          ,\] 
          thus $a \le_\wedge c$.
      \end{itemize}
      By question~\ref{q3}, it suffices to show that $(L, \le_\wedge)$ that binary meets for poset $(L, \le_\wedge)$ are $\wedge$ and that $\top$ is the greatest element of poset $(L, \le_\wedge)$.
      Consider~$a, b, c$ three arbitrary elements of $L$.
      \begin{itemize}
        \item For any $b \in L$, we have $b \wedge \top = b$ (as $\top$ is a neutral element) and thus $b \le_\wedge \top$ for all $b \in L$, so $\top$ is the greatest element of $(L, \le_\wedge)$.
        \item Firstly, element $a \wedge b$ is a lower bound of $\{a,b\}$ as 
          \begin{gather*}
            a \wedge b \le_\wedge a \quad\text{iff}\quad a \wedge b = (a \wedge b) \wedge a\\
            a \wedge b \le_\wedge b \quad\text{iff}\quad a \wedge b = (a \wedge b) \wedge b\\
          \end{gather*}
          and the latter equalities are true by idempotence, associativity, and finally commutativity.
          Secondly, consider $c \in L$ such that we have $c \le_\wedge a$ and $c \le_\wedge b$, then $c \wedge a = c = c \wedge b$.
          We therefore have that $c \le_\wedge a \wedge b$, as
          \[
          c \wedge (a \wedge b) = (c \wedge a) \wedge b = c \wedge b = c
          .\] 
          We can conclude that $\wedge$ is the binary meet operator in  $(L, \le_\wedge)$.
      \end{itemize}
    \item Applying the previous result with the commutative monoid $(L, \vee, \bot)$, we obtain that $(L, \ge_\vee)$\footnote{The notation is, in a way, "context-sensitive," as for an arbitrary monoid $(M, \oast, \mathbf{I})$, we can either define $a \le_\oast b$ as $a \oast b = a$ or $a \le_\oast b$ as $a \oast b = b$.} is a meet semilattice where binary meets for $\ge_\vee$ are given by $\vee$ and the greatest element for $\ge_\vee$ is $\bot$.
      We can thus conclude that $(L, \le_\vee)$ is a join semilattice where binary joins for $\le_\vee$ are given by $\vee$ and the least element for $\le_\vee$ is $\bot$.
  \end{enumerate}

  \begin{que}
    \label{q6}
    Show the following, for the partial order $(\mathfrak{L}(\mathsf{LML}), \le)$:
    \begin{enumerate}
      \item $(\mathfrak{L}(\mathsf{LML}), \le)$ is a meet semilattice with greatest element $\top$ and binary joins given by
        \begin{align*}
          - \wedge -: \mathfrak{L}(\mathsf{LML}) \times  \mathfrak{L}(\mathsf{LML}) &\longrightarrow  \mathfrak{L}(\mathsf{LML}) \\
          (\phi, \psi) &\longmapsto \phi \wedge \psi\;
        ;\end{align*}
      \item $(\mathfrak{L}(\mathsf{LML}), \le)$ is a join semilattice with least element $\bot$ and binary joins given by
        \begin{align*}
          - \vee -: \mathfrak{L}(\mathsf{LML}) \times  \mathfrak{L}(\mathsf{LML}) &\longrightarrow  \mathfrak{L}(\mathsf{LML}) \\
          (\phi, \psi) &\longmapsto \phi \vee \psi
        .\end{align*}
    \end{enumerate}
  \end{que}

  We will use the lemma proven in question~\ref{q2} (lemma~\ref{lem1}, page~\pageref{lem1}).
  
  \begin{enumerate}
    \item We only need to show that $- \wedge -$ defines a binary meet for $(\mathfrak L(\mathsf{LML}), \le)$ and that $\top$ is a greatest element.

      For any $\phi \in \mathfrak L(\mathsf{LML})$, we have $\phi \le \top$ as $\llbracket \phi\rrbracket \subseteq \llbracket \top\rrbracket = (\mathbf{2}^\mathrm{AP})^\omega$, thus $\top$ is the greatest element.

      For any formulae~$\phi, \psi \in \mathfrak L(\mathsf{LML})$, we have that $\phi \land \psi \le \phi$ and $\phi \land \psi \le \psi$ as both $\llbracket \phi\rrbracket  $ and $\llbracket \psi\rrbracket  $ are supersets of $\llbracket \phi \land \psi\rrbracket = \llbracket \phi\rrbracket  \cap \llbracket \psi\rrbracket$ (by definition of interpretation~$\llbracket -\rrbracket$).
      Then, if $\vartheta \le \phi$ and $\vartheta \le \psi$, we have that $\llbracket \vartheta\rrbracket \subseteq \llbracket \phi\rrbracket$ and $\llbracket \vartheta\rrbracket \subseteq \llbracket \psi\rrbracket$ thus $\llbracket \vartheta\rrbracket \subseteq \llbracket \phi\rrbracket \cap \llbracket \psi\rrbracket = \llbracket \phi \land \psi\rrbracket$, therefore $\vartheta \le \phi\land\psi$.

      We can conclude that $(\mathfrak L(\mathsf{LML}), \le)$ is a meet semilattice with greatest element $\top$ and binary meets given by $- \wedge -$.

    \item We only need to show that $- \vee -$ defines a binary join for $(\mathfrak L(\mathsf{LML}), \le)$ and that $\bot$ is a least element.

      For any $\phi \in \mathfrak L(\mathsf{LML})$, we have $\bot \le \phi$ as $\emptyset \subseteq \llbracket \bot\rrbracket \subseteq \llbracket \phi\rrbracket$, thus $\bot$ is the least element.

      For any formulae~$\phi, \psi \in \mathfrak L(\mathsf{LML})$, we have that $\phi \le \phi \lor \psi$ and $\psi\le \phi \lor \psi$ as both $\llbracket \phi\rrbracket$ and $\llbracket \psi\rrbracket  $ are subsets of $\llbracket \phi \lor \psi\rrbracket = \llbracket \phi\rrbracket  \cup \llbracket \psi\rrbracket$ (by definition of interpretation~$\llbracket -\rrbracket$).
      Then, if $\phi \le \vartheta $ and $\psi \le \vartheta $, we have that $\llbracket \phi\rrbracket \subseteq \llbracket \vartheta\rrbracket$ and $\llbracket \psi\rrbracket \subseteq \llbracket \vartheta\rrbracket$ thus $\llbracket \phi \lor \psi \rrbracket = \llbracket \phi\rrbracket \cup \llbracket \psi\rrbracket  \subseteq \llbracket \vartheta\rrbracket$, therefore $\phi \lor \psi \le \vartheta$.

      We can conclude that $(\mathfrak L(\mathsf{LML}), \le)$ is a join semilattice with least element $\bot$ and binary joins given by $- \vee -$.
  \end{enumerate}


  \begin{que}
    Show that a map of meet (\textup{resp}.\ join) semilattices is monotone.
  \end{que}

  Let $f : L \to L'$ be an arbitrary function where  $(L, \le)$ and $(L', \le')$ are partial orders.

  \begin{enumerate}
    \item Suppose $f : (L, \le) \to (L', \le')$ is a map of meet semilattices.
      Let $a, b \in L$.
      If $a \le b$, then $a \wedge b = a$ and, as $f$ preserves finite meets, \[
      f(a) \wedge' f(b) = f(a \wedge b) = f(a)
      ,\]and thus $f(a) \le' f(b)$.
      Therefore, $f$ is monotone.
    \item Suppose $f : (L, \le) \to (L', \le')$ is a map of join semilattices.
      Let $a, b \in L$.
      If $a \le b$, then $a \vee b = b$ and, as $f$ preserves finite joins, \[
      f(a) \vee' f(b) = f(a \vee b) = f(b)
      ,\]and thus $f(a) \le' f(b)$.
      Therefore, $f$ is monotone.
  \end{enumerate}

  \subsection{Lattices.}

  \begin{que}
    \label{q8}
    Consider the partial order $(L, \sqsubseteq)$ where
    \[
    L \coloneqq \mathds{N} \cup \{\alpha, \beta, \top\} 
    ,\] 
    where $\sqsubseteq$ is the reflexive-transitive closure of $\sqsubset$, where 
    \[
      a \sqsubset b \quad \text{ iff } \quad \begin{cases}
        a < b \text{ in } \mathds{N}\\
        \quad\quad\text{or}\\
        a \in \mathds{N} \text{ and }b \in \{\alpha, \beta\}\\
        \quad\quad\text{or}\\
        a \in \{\alpha, \beta\}  \text{ and } b = \top.
      \end{cases}
    \]
    Show that $(L, \sqsubseteq)$ is a join semilattice but is not a lattice.
  \end{que}

  \begin{figure}[H]
    \centering
    \begin{tikzpicture}[node distance=2.7cm]
      \node (0) {$0$};
      \node[right of=0] (1) {$1$};
      \node[right of=1] (2) {$2$};
      \node[right of=2] (3) {$3$};
      \node[right of=3] (d) {$\ldots$};
      \node[right of=d] (x) {};
      \node[above=0.5cm of x] (a) {$\alpha$};
      \node[below=0.5cm of x] (b) {$\beta$};
      \node[right of=x] (t) {$\top$};
      \draw (0) to node[midway, fill=white, inner sep=0em] {$\sqsubset$} (1);
      \draw (1) to node[midway, fill=white, inner sep=0em] {$\sqsubset$} (2);
      \draw (2) to node[midway, fill=white, inner sep=0em] {$\sqsubset$} (3);
      \draw (3) to node[midway, fill=white, inner sep=0em] {$\sqsubset$} (d);
      \draw (d) to[out=0, in=180] node[midway, fill=white, inner sep=0em, sloped, anchor=center] {$\sqsubset$} (a);
      \draw (d) to[out=0, in=180] node[midway, fill=white, inner sep=0em, sloped, anchor=center] {$\sqsubset$} (b);
      \draw (a) to[out=0, in=180] node[midway, fill=white, inner sep=0em, sloped, anchor=center] {$\sqsubset$} (t);
      \draw (b) to[out=0, in=180] node[midway, fill=white, inner sep=0em, sloped, anchor=center] {$\sqsubset$} (t);
    \end{tikzpicture}
    \caption{Hasse diagram of $(L, \sqsubseteq)$ from question~\ref{q8}}
    \vspace{0.6em}
    \footnotesize\textsf{\textbf{Note:} Hasse diagrams are usually read bottom-to-top,\\ but this one is drawn left-to-right for convenience.}
  \end{figure}

  The relation $\sqsubseteq$ is a partial order. Reflexivity and transitivity is true by definition of $\sqsubseteq$ as the reflexive and transitive closure of $\sqsubseteq$.
  For antisymmetry, we have that:
  \begin{itemize}
    \item for $n,m \in \mathds{N}$,  $n \sqsubseteq m$ iff $n \le m$;
    \item for any $n \in \mathds{N}$ and $m \in L \setminus \mathds{N}$, we have $n \sqsubseteq m$ and $m \not\sqsubseteq n$;
    \item $\alpha \sqsubseteq\top$, $\beta \sqsubseteq \top$, $\top \not\sqsubseteq \alpha$, $\top \not\sqsubseteq \beta$, $\alpha \not\sqsubseteq \beta$ and $\beta \not\sqsubseteq \alpha$;
  \end{itemize}
  (this can be shown by induction on the relation $\sqsubseteq$).

  We have that $0$ is the least element in $(L, \sqsubseteq)$: we have that $0 \sqsubseteq a$ for all~$a \in L$.
  For $a, b \in L$, we can define $a \vee b$ as:
   \begin{itemize}
     \item if $a, b \in \mathds{N}$, let $a \vee b \coloneqq \min_{\le_{\mathds{N}}}(a, b)$;
     \item if $a \in \mathds{N}$ and $b \in L \setminus \mathds{N}$, let $a \vee b, b \vee a \coloneqq b$;
     \item otherwise let $\alpha \vee \beta \coloneqq \top$, $a \wedge a \coloneqq a$, $a \wedge \top, \top \wedge a \coloneqq \top$ for $a \in \{\alpha,\beta, \top \}$.
  \end{itemize}
  Using the previous results on $\sqsubseteq$, we have that $-\vee-$ \textit{really} is a join.

  This concludes the proof that $(L, \sqsubseteq)$ is a join semilattice.

  We also have that $(L, \sqsubseteq)$ is not a lattice.
  Suppose it is a lattice, and consider the element $a \coloneqq \alpha \wedge \beta$.
  Necessarily, we have that $a \in \mathds{N}$ (if $a = \alpha$ then we would have $\alpha \sqsubseteq \beta$, which is false).
  As $a = \alpha \wedge \beta$ and $a + 1 \sqsubseteq \alpha, \beta$ we have, by definition of meet, that $a + 1  \sqsubseteq a$, thus $a + 1 \le a$ (since $a, a+1 \in \mathds{N}$) which is \textit{\textbf{absurd}}.
  We can conclude that $(L, \sqsubseteq)$ is not a lattice.


  \begin{que}
    Consider a set $L$ equipped with two binary operations $\wedge, \vee : L \times L \to L$ and two constants $\top , \bot  \in L$.
    Assume that $(L, \wedge, \top)$ and $(L, \vee, \bot)$ are commutative monoids in which every element is idempotent.
    Show that the following are equivalent.
    \begin{enumerate}
      \item The partial order $\le_\vee$  induced by $(L, \vee, \bot)$ coincides with the partial order $\le_\wedge$ induces by $(L, \wedge, \top)$.
        \label{q9-1}
      \item $(L, \vee, \wedge, \bot, \top)$ satisfies the two following \textbf{absorptive laws}:
        \label{q9-2}
        \begin{align}
          \tag{\textsf{abs}\textsubscript{1}}
          \forall a, b \in L, \quad &a \vee (a \wedge b) = a
          \label{abs1}
          \\
          \tag{\textsf{abs}\textsubscript{2}}
          \forall a, b \in L, \quad &a \wedge (a \vee b) = a
          \label{abs2}
        \end{align}
    \end{enumerate}
  \end{que}

  \begin{itemize}
    \item Let us show that \ref{q9-1} implies \ref{q9-2}.
      Let $a, b \in L$.
      We have that $a \wedge b \le_\wedge a$ and, assuming $\le_\wedge$ and $\le_\vee$ coincide, $a \wedge b \le_\vee a$, thus $a \vee (a \wedge b) = a$, \textit{i.e.}\ (\ref{abs2}) holds.
      Similarly,  $a \le_\vee a \vee b$ thus $a \le_\wedge a \vee b$, so $(a \wedge b) \vee a = a$ holds, and we can recover (\ref{abs1}) by using commutativity.
    \item Let us show that \ref{q9-2} implies \ref{q9-1}.
      \begin{itemize}
        \item Suppose $b \le_\wedge a$, then $b \wedge a = b$.
          By (\ref{abs1}) and commutativity, we have $b \vee a = (b \wedge a) \vee a = a$, thus $b \le_\vee a$.
        \item Suppose $b \le_\vee a$, then $b \vee a = a$.
          By (\ref{abs2}), we have
          \[
          b \wedge a = b \wedge (b \vee a) = b
          ,\] thus $b \le_\wedge a$.
      \end{itemize}
      Thus the two order coincide.
  \end{itemize}

  \begin{que}
    Show that the partial order $(\mathfrak L (\mathsf{LML}), \le )$ is a lattice.
  \end{que}

  We have shown that $(\mathfrak L (\mathsf{LML}), \le )$ has a greatest element $\top$, a least element $\bot$, binary meets given by $- \wedge -$ and binary joins given by  $- \vee -$ (question~\ref{q6}).
  Thus it has all finite meets and finite joins (as seen in question~\ref{q3}),  \textit{i.e.}\ $(\mathfrak{L}(\mathsf{LML}), \le)$ is a lattice.
  \let\nxt\bigcirc

  \begin{que}
    Show that the function
    \begin{align*}
      \nxt: \mathfrak{L}(\mathsf{LML}) &\longrightarrow \mathfrak{L}(\mathsf{LML}) \\
      \phi &\longmapsto \nxt\phi
    \end{align*}
    is a morphism of lattices.
  \end{que}

  We know $\nxt$ is a map of meet iff $\nxt \top = \top$ and  $\nxt (\phi \land \psi) = \nxt \phi \land \nxt \psi$.
  Both are true as,
  \begin{gather*}
    \llbracket \nxt \top\rrbracket  = \big\{\,\sigma \in (\mathbf{2}^\mathrm{AP})^\omega  \;\big|\; \sigma \upharpoonright 1 \in \llbracket \top\rrbracket = (\mathbf{2}^\mathrm{AP})^\omega\,\big\}  = (\mathbf{2}^\mathrm{AP})^\omega = \llbracket \top\rrbracket  \\
    \llbracket \nxt (\phi \land \psi)\rrbracket = \big\{\,\sigma \in (\mathbf{2}^\mathrm{AP})^\omega  \;\big|\; \sigma \upharpoonright 1 \in \llbracket \phi\rrbracket \cap \llbracket \psi\rrbracket  \,\big\}  = \llbracket \nxt \phi\rrbracket   \cap \llbracket \nxt \psi\rrbracket  = \llbracket \nxt \phi \land \nxt \psi\rrbracket  .
  \end{gather*}
  Very similarly, $\nxt$  is a map of joins iff $\nxt \bot = \bot$ and  $\nxt (\phi \lor \psi) = \nxt \phi \lor \nxt \psi$.
  One can show that both equalities hold by applying $\llbracket -\rrbracket$ and showing the equality of the sets like above.

  Thus $\nxt : (\mathfrak L(\mathsf{LML}), \le) \to (\mathfrak L(\mathsf{LML}), \le)$ is a morphism of lattices.

  \subsection{Distributive Lattices.}

  \begin{que}
    Show that the following two \textbf{distributive laws} are equivalent in a lattice $(L, \vee, \wedge, \bot, \top)$:
    \begin{align}
      \tag{\textsf{dist}\textsubscript{1}}
      \forall a, b, c \in L, \quad &a \wedge (b \vee c) = (a \wedge b) \vee (a \wedge c)
      \label{dist1}
      \\
      \tag{\textsf{dist}\textsubscript{2}}
      \forall a, b, c \in L, \quad &a \vee (b \wedge c) = (a \vee b) \wedge (a \vee c)
      \label{dist2}
    \end{align}
  \end{que}

  Suppose (\ref{dist1}) holds and let us show (\ref{dist2}) is true for $a, b, c \in L$:
  \begin{align*}
    (a \vee b) \wedge (a \vee c)
    &= ((a \vee b) \wedge a) \vee ((a \vee b) \wedge c) && \text{by (\ref{dist1})}\\
    &= a \vee ((a \vee b) \wedge c) && \text{by (\ref{abs2})} \\
    &= a \vee (a \wedge b) \vee (b \wedge c) && \text{by (\ref{dist1})} \\
    &= a \vee (b \wedge c) && \text{by (\ref{abs1})} .
  \end{align*}

  To prove that (\ref{dist1}) holds when (\ref{dist2}) is true, we can apply the previous result to the lattice $(L, \le)^\mathrm{op} = (L, \ge)$.
  This gives exactly the implication "(\ref{dist2}) implies (\ref{dist1})," as wanted.

  Thus, the two distributive laws (\ref{dist1}) and (\ref{dist2}) are equivalent.

  \begin{que}
    Show that the lattice $(\mathfrak L(\mathsf{LML}), \le)$ is distributive.
  \end{que}

  Let $\phi, \psi, \vartheta \in \mathfrak{L}(\mathsf{LML})$.
  We have that \[
    \textstyle
  \llbracket \phi \land (\psi \lor \vartheta) \rrbracket 
  = \llbracket \phi\rrbracket \cap (\llbracket \psi\rrbracket \cup \llbracket \vartheta\rrbracket)  
  = (\llbracket \phi\rrbracket \cap \llbracket \psi\rrbracket) \cup (\llbracket \phi\rrbracket \cap \llbracket \vartheta\rrbracket)  
  = \llbracket (\phi \land \psi) \lor (\phi \land \vartheta)\rrbracket 
  ,\] 
  thus $\phi \land (\psi \lor \vartheta) = (\phi \land \psi) \lor (\phi \land \vartheta)$.

  \begin{que}
    Consider the following lattice $\mathrm{M}_3$:
    \begin{figure}[H]
      \centering
      \begin{tikzpicture}[node distance=2cm]
        \node (b) {$b$};
        \node[above of=b] (T) {$\top$};
        \node[below of=b] (B) {$\bot$};
        \node[left of=b] (a) {$a$};
        \node[right of=b] (c) {$c$};
        \draw (B) to (a);
        \draw (B) to (b);
        \draw (B) to (c);
        \draw (a) to (T);
        \draw (b) to (T);
        \draw (c) to (T);
      \end{tikzpicture}
    \end{figure}
    (\textup{i.e.}\ $\bot \le a, b, c \le \top$ with $a,b,c$ incomparable).
    Show that $\mathrm{M_3}$ is not distributive.
  \end{que}

  Suppose $\mathrm{M_3}$ is distributive.
  As $a, b, c$ are incomparable, we have that 
   \[
  a \wedge b = a \wedge c = \bot \quad\quad \text{ and }\quad\quad b \vee c = \top
  ,\]
  and thus,
  \[
  a = a \wedge \top = a \wedge (b \vee c) = (a \wedge b) \vee (a \wedge c) = \bot \vee \bot = \bot
  ,\] 
  which is \textit{\textbf{absurd}}.
  Thus $\mathrm{M_3}$ is not distributive.

  \subsection{Booleans algebras.}

  \begin{que}
    Show that if $(L, \le)$ is a distributive lattice then $a \in L$ has at most one complement.
  \end{que}

  Consider $c, c' \in L$ two complements of $a \in L$.
  Then, we have that 
  \[
    c = c \wedge \top = c \wedge (a \vee c') \overset {\text{(\ref{dist1})}} =
    (c \wedge a) \vee (c \wedge c') = \bot \vee (c \wedge c') = c \wedge c'
  ,\]
  and,
  \[
    c' = c' \wedge \top = c' \wedge (a \vee c) \overset {\text{(\ref{dist1})}} =
    (c' \wedge a) \vee (c' \wedge c) = \bot \vee (c' \wedge c) = c' \wedge c
  .\] 
  We can conclude that $c = c'$ by commutativity of meets.

  \begin{que}
    Show that $(\mathfrak L(\mathsf{LML}), \le)$ is a Boolean algebra.
  \end{que}

  Let us show that $\lnot \phi$ is a complement for  $\phi \in \mathfrak{L}(\mathsf{LML})$.
  We have to check that $\phi \wedge \lnot \phi = \bot$ and $\phi \vee \lnot \phi = \top$ hold.
  Both equalities can be easily checked with interpretations:
  \[
  \llbracket \phi \land \lnot \phi\rrbracket  = \llbracket \phi\rrbracket \cap \llbracket \phi\rrbracket^\complement   = \emptyset = \llbracket \bot\rrbracket  
  ,\] 
  and
  \[
  \llbracket \phi \lor \lnot \phi\rrbracket  = \llbracket \phi\rrbracket \cup \llbracket \phi\rrbracket^\complement = (\mathbf{2}^\mathrm{AP})^\omega = \llbracket \top\rrbracket  
  .\]
  Thus, $\lnot \phi$ is \textit{\textbf{the}} complement of $\phi$ in $(\mathfrak{L}(\mathsf{LML}), \le)$, which is, as a consequence, a Boolean algebra.

  \begin{que}
    Show that the following \textbf{De Morgan Laws} hold in every Boolean algebra $(B, \vee, \wedge, \bot, \top)$:
    \[
    a \wedge b = \lnot (\lnot a \vee \lnot b)
    \quad\quad
    a \vee b = \lnot (\lnot a \wedge \lnot b)
    \quad\quad
    a = \lnot \lnot a
    .\] 
  \end{que}

  We start by showing that $\lnot a \lor \lnot b$ is a complement to $(a \land b)$:
  \begin{gather*}
    (a \wedge b) \wedge (\lnot a \lor \lnot b)
    =
    (a \wedge b \wedge \lnot b) 
    \lor
    (a \wedge b \wedge \lnot a) 
    = \bot  \lor \bot 
    = \bot\\
    (a \wedge b) \vee (\lnot a \vee \lnot b)
    = (a \vee \lnot a \vee \lnot b) \wedge (b \vee \lnot a \vee \lnot b)
    = \top \wedge \top = \top 
  ,\end{gather*}
  by using the Boolean algebra laws.
  Thus $a \wedge b = \lnot (\lnot a \vee \lnot b)$.

  For  $a \vee b = \lnot (\lnot a \land \lnot b)$, we proceed by duality by applying the previous result to the Boolean algebra $(B, \wedge, \vee, \top, \bot)$, as a complement for $a$ in $(B, \ge)$ is exactly a complement for $a$ in $(B, \le)$.

  We can easily check that $\lnot \top = \bot$, and then
  \[
  a = a \wedge \top = \lnot (\lnot a \vee \lnot \top) = \lnot (\lnot a \vee \bot) = \lnot \lnot a
  .\] 

  \begin{que}
    Show that if $f$ is a map of Boolean algebras from $(B, \le )$ to $(B', \le')$ then $f$ preserves complements.
  \end{que}

  We have that 
  \[
  \bot ' = f(\bot ) = f(a \wedge \lnot a) = f(a) \wedge' f(\lnot a)
  ,\] 
  and
  \[
  \top = f(\top) = f(a \vee \lnot a) = f(a) \vee' f(\lnot a)
  ,\]
  thus $f(\lnot a)$ is a complement of $f(a)$ and, by unicity, $f(\lnot a) = \lnot f(a)$.
  We can conclude that a map of Boolean algebras preserves complements.

  \section{Representation of Boolean Algebras.}

  \hfill

  \begin{center}
    \large\sffamily\itshape\bfseries
    The rest will be given in the next part.
  \end{center}

\end{document}
