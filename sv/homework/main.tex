\documentclass[fontsize=16pt,a4paper,DIV=17,parskip=half]{scrartcl}

\usepackage[utf8]{inputenc}
\usepackage[dvipsnames]{xcolor}
\usepackage[hyperindex]{hyperref}
\usepackage{lastpage}
\usepackage{tikz}
\usepackage{tikzpagenodes}
\usepackage{pgfplots}
\usepackage{enumitem}
\usepackage{scrlayer-scrpage}
\usepackage{xspace}
\usepackage{float}
\usepackage{amsfonts,amsmath,amsthm,amssymb}
\usepackage{thmtools}
\usepackage[english]{babel}
\usepackage[autostyle, style=english]{csquotes}
\usepackage{subfigure}
\usepackage{mleftright}
%\usepackage{BOONDOX-calo}
%\usepackage{dsfont}
\usepackage{tikz-cd}
\usepackage[framemethod=TikZ]{mdframed}
\usepackage{soulutf8}
\usepackage{mathtools}
\usepackage{multicol}
\usepackage{stmaryrd}
\usepackage{fvextra}
\usepackage{adjustbox}
\usepackage{fontspec}
\usepackage{etoolbox}
\usepackage{todonotes}
\usepackage{verbatim}
\usepackage{ebproof}
\usepackage{cancel}
\usepackage{setspace}
\usepackage{fourier-otf}
%\usepackage{pxfonts}
%\usepackage[scaled=0.92]{mathpazo}
%\usepackage{juliamono}
\usepackage[osf]{Alegreya}
\usepackage[osf]{AlegreyaSans}
\let\mathds\mathbb
%\DeclareMathAlphabet{\mathsf}{OT1}{cmss}{m}{n}

\renewcommand{\mathsf}[1]{\textup{\textsf{#1}}}
\newcommand\Agda{\textsf{Agda}\xspace}

\ebproofset{right label template=$\inserttext$, left label template=\tiny$\inserttext$, center=false}

\RedeclareSectionCommand[beforeskip=0.10em, afterskip=0.10em]{section}
\RedeclareSectionCommand[beforeskip=0.05em, afterskip=0.001em plus 0em]{subsection}


\fvset{bgcolor=lightgray!10,backgroundcolorpadding=3pt}

\MakeOuterQuote{"}

\colorlet{deeppurple}{DarkOrchid}
\colorlet{deepgreen}{ForestGreen!70!black}
\colorlet{deepblue}{NavyBlue!70!black}
\colorlet{deepred}{RawSienna!70!black}
\colorlet{nicered}{BrickRed!70!white}

\makeatletter
\g@addto@macro\bfseries{\boldmath}
\makeatother

\hypersetup{
    colorlinks,
    citecolor=deepgreen,
    filecolor=nicered,
    linkcolor=deepblue,
    pdfencoding=auto,
    psdextra,
    urlcolor=deepred
}

\usetikzlibrary{positioning,shadings,arrows.meta}

\clearpairofpagestyles

\newcommand\showpage{\itshape\hfill--~\thepage/\pageref*{LastPage}~--\hfill}
\cofoot[\showpage]{\showpage} \cefoot[\showpage]{\showpage}

\setlist[enumerate]{font={\bfseries\color{deepblue}}}
\AtBeginDocument{
  \renewcommand{\labelitemi}{\bfseries\color{deepblue}$\triangleright$}
  \renewcommand{\labelitemii}{\bfseries\color{deepblue}–}
  \renewcommand{\labelitemiii}{\bfseries\color{deepblue}•}
}

\newcommand\separatorBlock{
  \raisebox{-0.2em}{
    \tikz{ \draw[deepblue,ultra thick, line cap=round] (0,0) -- (0,1em); }
  }
}

\newcommand\vertical[1]{
  \rotatebox[origin=c]{270}{\ensuremath{#1}}
}

\mdfsetup{skipabove=1em,skipbelow=0em,linewidth=0pt,rightline=false, topline=false, bottomline=false}


\theoremstyle{definition}

\declaretheoremstyle[
  headfont=\bfseries\sffamily\color{deepgreen}, bodyfont=\normalfont,
% mdframed={
%   linecolor=ForestGreen, % backgroundcolor=ForestGreen!5,
% },
]{thmgreenbox}

\declaretheoremstyle[
  headfont=\bfseries\sffamily\color{deepblue}, bodyfont=\normalfont\sffamily\itshape,
% mdframed={
%   linecolor=NavyBlue,% backgroundcolor=NavyBlue!5,
% },
]{thmbluebox}

\declaretheoremstyle[
  headfont=\bfseries\sffamily\color{deepblue}, bodyfont=\normalfont,
  mdframed={
    linecolor=deepblue,
    linewidth=2pt,
  },
  numbered=no,
]{thmblueline}

\declaretheoremstyle[
  headfont=\bfseries\sffamily\color{deepred}, bodyfont=\normalfont,
% mdframed={
%   linecolor=RawSienna,% backgroundcolor=RawSienna!5,
% },
]{thmredbox}

\declaretheoremstyle[
  headfont=\itshape\sffamily\color{deepred}, bodyfont=\normalfont,
% mdframed={
%   linecolor=RawSienna,% backgroundcolor=RawSienna!1,
%   linewidth=0pt,
% },
  numbered=no,
  qed=\qedsymbol,
]{thmproofbox}

\newcommand\defineMarkerColor[2]{
  \AtBeginEnvironment{#1}{
    \setlist[enumerate]{font={\color{#2}}}
    \renewcommand{\labelitemi}{\color{#2}\small$\triangleright$}
    \renewcommand{\labelitemii}{\color{#2}–}
    \renewcommand{\labelitemiii}{\color{#2}•}
    \renewcommand\emph[1]{{\bfseries\em\color{#2}##1}}
  }
}


\AtBeginDocument{
  \setlist[enumerate]{font={\bfseries\color{deepblue}}}
  \renewcommand{\labelitemi}{\bfseries\color{deepblue}\small$\triangleright$}
  \renewcommand{\labelitemii}{\bfseries\color{deepblue}–}
  \renewcommand{\labelitemiii}{\bfseries\color{deepblue}•}
}


\setlist[enumerate]{font={\color{deepblue}}}
\renewcommand{\labelitemi}{\color{deepblue}\small$\triangleright$}
\renewcommand{\labelitemii}{\color{deepblue}–}
\renewcommand{\labelitemiii}{\color{deepblue}•}

\declaretheorem[style=thmgreenbox, name=Axiom, numbered=no]{axi} \defineMarkerColor{axi}{deepgreen}
\declaretheorem[style=thmgreenbox, name=Definition]{defn} \defineMarkerColor{defn}{deepgreen}
\declaretheorem[style=thmbluebox, name=Example]{exm}      \defineMarkerColor{exm}{deepblue}
\declaretheorem[style=thmbluebox, name=Exercise]{exo}     \defineMarkerColor{exo}{deepblue}
\declaretheorem[style=thmbluebox, name=Question]{que}     \defineMarkerColor{que}{deepblue}
\declaretheorem[style=thmbluebox,sibling=que, name=Question${}^\star$]{ques}     \defineMarkerColor{ques}{deepblue}
\declaretheorem[style=thmredbox, name=Proposition]{prop}  \defineMarkerColor{prop}{deepred}
\declaretheorem[style=thmredbox, name=Theorem]{thm}      \defineMarkerColor{thm}{deepred}
\declaretheorem[style=thmredbox, name=Lemma]{lem}         \defineMarkerColor{lem}{deepred}
\declaretheorem[style=thmredbox, name=Corollary]{crlr}   \defineMarkerColor{crlr}{deepred}
\declaretheorem[style=thmblueline, name=Remark]{rmk}    \defineMarkerColor{rmk}{deepblue}
\declaretheorem[style=thmblueline, name=Note]{note}       \defineMarkerColor{note}{deepblue}
\declaretheorem[style=thmproofbox, name=Proof]{replacementproof}
\newenvironment{prv}[1][\proofname]{\vspace{-12pt}%
\begin{replacementproof}}{\end{replacementproof}} \defineMarkerColor{prv}{deepred}
\declaretheorem[style=thmproofbox, name=Proof idea]{replacementideaproof}
\newenvironment{prvid}[1][\proofname]{\vspace{-12pt}%
\begin{replacementideaproof}}{\end{replacementideaproof}} \defineMarkerColor{prvid}{deepred}

\RequirePackage{caption}
\DeclareCaptionLabelFormat{labelformat}{\textbf{#1~#2}\separatorBlock}
\captionsetup{labelformat=labelformat,labelsep=none,textfont=sl}

\DeclareMathSizes{11}{9}{7}{5}

\title{Semantics and Verification -- \textit{Homework}}
\author{Hugo \textsc{Salou}}

\let\emph\relax
\DeclareTextFontCommand{\emph}{\bfseries\em\color{deepblue}}

\renewcommand{\thefootnote}{\alph{footnote}}

\makeatletter
\def\moverlay{\mathpalette\mov@rlay}
\def\mov@rlay#1#2{\leavevmode\vtop{%
   \baselineskip\z@skip \lineskiplimit-\maxdimen
   \ialign{\hfil$\m@th#1##$\hfil\cr#2\crcr}}}
\newcommand{\charfusion}[3][\mathord]{
    #1{\ifx#1\mathop\vphantom{#2}\fi
        \mathpalette\mov@rlay{#2\cr#3}
      }
    \ifx#1\mathop\expandafter\displaylimits\fi}
\makeatother

\usepackage{pifont}


\tikzcdset{arrow style=math font}
\tikzset{
  equiv/.style={-,preaction={draw,double equal sign distance}},
  >=Straight Barb,
}

\begin{document}
  \begin{center}
    \bfseries
    \sffamily

    {\large\itshape ---\hspace{1em}Homework\hspace{1em}---}

    {\huge Semantics and Verification}

    {\large \itshape Hugo SALOU}
  \end{center}

  \section{Toward Stone Duality.}

  \begin{que}
    Show that every Stone space $(X, \Omega)$ is Hausdorff (if $x, y \in X$ are distinct, there there are disjoint $U, V \in \Omega$ such that $x \in U$ and $y \in V$).
  \end{que}

  Let $x, y \in X$ be two distinct points of a Stone space $(X, \Omega)$.
  As, $(X, \Omega)$ is $\mathrm{T_0}$ and without loss of generality, there exists $W \in \Omega$ such that $x \in W$ and $y \not\in W$.
  As $(X, \Omega)$ is zero-dimensional, we can write $W \eqqcolon \bigcup_{i \in  I} W_i$ where $W_i \in \mathbf{K}\Omega$ for every $i \in I$.
  Thus, there exists a clopen set $U \coloneqq W_i \in \Omega$ such that ${x \in W_i \subseteq U}$.
  Define $V \coloneqq X \setminus U \in \Omega$, and we have that $x \in U$, $y \in V$ (as $y \not\in W \supseteq U$) and the open sets $U$ and $V$ are disjoint.
  We can conclude that every Stone space is Hausdorff.

  \begin{que}
    \label{q2}
    Show that $\le$ is a partial order on $\mathfrak L(\mathsf{LML})$.
  \end{que}

  We start by showing the following lemma.
  \begin{lem}
    \label{lem1}
    We have $\phi \le \psi$ if and only if $\llbracket \phi\rrbracket \subseteq \llbracket \psi\rrbracket$.
  \end{lem}
  \begin{prv}
    We have that $\phi \le \psi$ iff $\phi \equiv \phi \land \psi$ iff $\llbracket \phi\rrbracket = \llbracket \phi \land \psi\rrbracket = \llbracket \phi\rrbracket \cap \llbracket \psi\rrbracket$ (that last equality is by definition of $\llbracket -\rrbracket$) iff $\llbracket \phi\rrbracket \subseteq \llbracket \psi\rrbracket$.
  \end{prv}

  We can thus easily show that $\le$ is a partial order.
  \begin{itemize}
    \item \textit{Reflexivity}. As $\llbracket \phi\rrbracket \subseteq \llbracket \phi\rrbracket$, we have that $\phi \le \phi$ for every $\phi \in \mathfrak L(\mathsf{LML})$.
    \item \textit{Transitivity}. For any $\phi, \psi, \vartheta \in \mathfrak L(\mathsf{LML})$, if $\phi \le \psi$ and $\psi \le \vartheta$ then, by the lemma, $\llbracket \phi\rrbracket \subseteq \llbracket \psi\rrbracket \subseteq \llbracket \vartheta\rrbracket$, thus we have $\llbracket \phi \rrbracket \subseteq \llbracket \vartheta\rrbracket$, \textit{i.e.}\ $\phi \le \vartheta$.
    \item \textit{Antisymmetry}.
      For any $\phi, \psi \in \mathfrak L(\mathsf{LML})$, if $\phi \le \psi$ and $\psi \le \phi$ then, by double inclusion with the above lemma, $\llbracket \phi\rrbracket = \llbracket \psi\rrbracket$ thus $\phi = \psi$ as we consider $\mathsf{LML}$-formulae quotiented by $\equiv$.
  \end{itemize}

  \section{Lattices and Boolean Algebras.}

  \subsection{Semilattices.}

  \begin{que}
    \label{q3}
    Let $(L, \le)$ be a partial order.
    \begin{enumerate}
      \item Show that $(L, \le)$ is a meet semilattice if, and only if, $L$ has binary meets $\wedge : {L \times L} \to L$ and greatest element $\top \in L$.
        \label{q3-1}
      \item Show that $(L, \le)$ is a join semilattice if, and only if, $L$ has binary joins $\vee : {L \times L} \to L$ and least element $\bot \in L$.
    \end{enumerate}
  \end{que}

  \begin{enumerate}
    \item If $(L, \le)$ is a meet semilattice, then $L$ has binary meets and a greatest element $\top = \bigwedge \emptyset$ (any element is a lower bound of $\emptyset$, thus the greatest lower bound of $\emptyset$ is the greatest element).

      Now, suppose $(L, \le)$ has a binary meet $\wedge$ and a greatest element $\top$.
      Consider $\{a_i  \mid i \in I\}$ a finite subset of elements of $L$.
      By induction on~$\# I \in \mathds{N}$, we define $\bigwedge_{i \in I} a_i \in I$ and show that $\bigwedge_{i \in I}a_i$ is a meet of the finite set $\{a_i  \mid  i \in I\}$ (like the notation suggests).
      \begin{itemize}
        \item Define $\bigwedge_{i \in \emptyset} a_i \coloneqq \top \in L$; as any element is a lower bound of $\emptyset$, the greatest lower bound of $\emptyset$ is the greatest element.
        \item Consider $I \coloneqq J \sqcup \{i\}$.
          By induction hypothesis, we have that $\bigwedge_{j \in J} a_j$ exists in $L$ and is a meet of $\{a_j  \mid j \in J\}$ in $(L, \le)$.
          Define \[
          \bigwedge_{k \in I} a_k \coloneqq \big(\bigwedge_{j \in J} a_j \big) \wedge a_i \in I
          .\]

          We have that $\bigwedge_{k \in I} a_k$ is a lower bound of $\{a_k  \mid k \in I\}$. Consider an element $a_k$ with $k \in I$.
          If $k \in J$ then $a_k \le \bigwedge_{j \in J}a_j \le \bigwedge_{k' \in I} a_{k'}$.
          Otherwise~$k = i$ and we immediacy have that $a_i \le \bigwedge_{k' \in I} a_{k'}$.

          Consider a lower bound $b \in L$ of $\{a_k  \mid k \in I\}$, then $b$ is a lower bound of $\{a_j  \mid j \in J\}$ and $b \le a_i$.
          We have $b \le \bigwedge_{j \in J} a_j$ and $b \le a_i$, therefore $b \le \bigwedge_{k \in I} a_k$.

          We can conclude that $\bigwedge_{k \in I} a_k$ is a meet of $\{a_k  \mid k \in K\}$.
      \end{itemize}
      Finally, we have that $(L, \le)$ has finite meets.
    \item This results follows from~\ref{q3-1} when considering the partial order $(L, \ge)$, by duality.
      Meets in $(L, \ge)$ are exactly joins in $(L, \le)$, and the greatest element of $(L, \ge)$ is the least element of $(L, \le)$, and \textit{vice versa}.
  \end{enumerate}

  \begin{note}
    In the following, when I will be dealing with multiple partial orders on the same set (\textit{e.g.}\ $\le$ and $\ge$), I will write $\bigwedge_\le$ for the meet operator in poset $(I, \le)$, $\bigvee_\le$ for the join operator in poset $(I, \le)$, $\top_{\le}$ for the greatest element in poset $(I, \le)$ and $\bot_{\le}$ for the least element in poset $(I, \le)$.
  \end{note}

  \begin{que}
    Prove the following.
    \begin{enumerate}
      \item Let $(L, \le)$ be a meet semilattice with binary meets $\wedge : L \times L \to L$ and greatest element $\top \in L$.
        Then $(L, \wedge, \top)$ is a commutative mooned in which every element is idempotent.
        Moreover, we have $a \le b$ iff $a = a \wedge b$.
      \item Let $(L, \le)$ be a join semilattice with binary joins $\vee : L \times L \to L$ and least element~$\bot$.
        Then $(L, \vee, \bot)$ is a commutative mooned in which every element is idempotent.
        Moreover, we have $a \le b$ iff $b = a \wedge b$.
    \end{enumerate}
  \end{que}

  \begin{enumerate}
    \item Let $a, b, c \in L$. First, we have that $a \wedge b = \bigwedge \{a,b\} = \bigwedge \{b,a\} = b \wedge a$ thus the binary meet operation $\wedge$ is commutative.
      Then, as a special case of the previous question, we have that $a$ and $\top \wedge a$ are both meets of $\{a\}$. And, by unicity of meets (\textit{i.e.}\ antisymmetry of $\le$, mainly), they are equals.
      Also as a special case of the previous question, we have that elements \[
      a \wedge (b \wedge c) = \top \wedge (a \wedge (b \wedge c))
      \] and \[
      (a \wedge b) \wedge c = c \wedge (a \wedge b) = \top \wedge (c \wedge (a \wedge b))
      \]  are both meets of the set $\{a,b,c\}$, thus are equal.
      Next, we have that \[
      a \wedge a = \bigwedge \{a, a\} = \bigwedge \{a\} = \top \wedge a = a
      \](penultimate equality is from last question), thus~$a \wedge a = a$.
      Finally, we have that:
      \begin{itemize}
        \item if $a = a \wedge b$ then $a$ is a lower bound of $\{a,b\}$ thus $a \le b$;
        \item if $a \le b$ then $a = a \wedge b$ as $a$ is a lower bound of $\{a,b\}$ and any lower bound $c$ of $\{a,b\}$ must satisfy $c \le a$.
      \end{itemize}
    \item Consider the meet semilattice $(L, \ge)$ and apply the results above.
      Meets in $(L, \ge)$ are exactly joins in $(L, \le)$, and the greatest element of $(L, \ge)$ is the least element of $(L, \le)$, and \textit{vice versa}.
      The last statement follows from the equivalence:
      \[
      a \le b \quad\text{iff}\quad b\ge a\quad\text{iff}\quad b = a \wedge_\ge b\quad\text{iff}\quad b = a \vee_\le b
      ,\] 
      where the second "iff" follows from the result above for $(L, \ge)$, and the last one follows from the equality $a \wedge_\ge b = a \vee_\le b$.
  \end{enumerate}

  \begin{que}
    Prove the following.
    \begin{enumerate}
      \item Given a commutative monoid $(L, \wedge, \top)$ in which every element is idempotent, let ${a \le_\wedge b}$ iff  $a = a \wedge b$.
        Then $(L, \le_\wedge)$ is a meet semilattice with binary meets given by $\wedge$ and greatest element $\top$.
      \item Given a commutative monoid $(L, \vee, \bot)$ in which every element is idempotent, let ${a \le_\vee b}$ iff  $b = a \vee b$.
        Then $(L, \le_\vee)$ is a join semilattice with binary joins given by $\vee$ and least element $\bot$.
    \end{enumerate}
  \end{que}

  \begin{enumerate}
    \item Let us start by showing that $(L, \le_\wedge)$ is a partial order.
      \begin{itemize}
        \item \textit{Reflexivity.} As $a \wedge a = a$ by idempotence, we have $a \le_\wedge a$.
        \item \textit{Antisymmetry.} If $a \le_\wedge b$ and $b \le_\wedge a$ then, by commutativity, we have $a \wedge b = a = b$.
        \item \textit{Transitivity.} If $a \le_\wedge b$ and $b \le_\wedge c$ then, by associativity,
          \[
          a = a \wedge b = a \wedge (b \wedge c) = (a \wedge b) \wedge c = a \wedge c
          ,\] 
          thus $a \le_\wedge c$.
      \end{itemize}
      By question~\ref{q3}, it suffices to show that $(L, \le_\wedge)$ that binary meets for poset $(L, \le_\wedge)$ are $\wedge$ and that $\top$ is the greatest element of poset $(L, \le_\wedge)$.
      Consider~$a, b, c$ three arbitrary elements of $L$.
      \begin{itemize}
        \item For any $b \in L$, we have $b \wedge \top = b$ (as $\top$ is a neutral element) and thus $b \le_\wedge \top$ for all $b \in L$, so $\top$ is the greatest element of $(L, \le_\wedge)$.
        \item Firstly, element $a \wedge b$ is a lower bound of $\{a,b\}$ as 
          \begin{gather*}
            a \wedge b \le_\wedge a \quad\text{iff}\quad a \wedge b = (a \wedge b) \wedge a\\
            a \wedge b \le_\wedge b \quad\text{iff}\quad a \wedge b = (a \wedge b) \wedge b\\
          \end{gather*}
          and the latter equalities are true by idempotence, associativity, and finally commutativity.
          Secondly, consider $c \in L$ such that we have $c \le_\wedge a$ and $c \le_\wedge b$, then $c \wedge a = c = c \wedge b$.
          We therefore have that $c \le_\wedge a \wedge b$, as
          \[
          c \wedge (a \wedge b) = (c \wedge a) \wedge b = c \wedge b = c
          .\] 
          We can conclude that $\wedge$ is the binary meet operator in  $(L, \le_\wedge)$.
      \end{itemize}
    \item Applying the previous result with the commutative monoid $(L, \vee, \bot)$, we obtain that $(L, \ge_\vee)$\footnote{The notation is, in a way, "context-sensitive," as for an arbitrary monoid $(M, \oast, \mathbf{I})$, we can either define $a \le_\oast b$ as $a \oast b = a$ or $a \le_\oast b$ as $a \oast b = b$.} is a meet semilattice where binary meets for $\ge_\vee$ are given by $\vee$ and the greatest element for $\ge_\vee$ is $\bot$.
      We can thus conclude that $(L, \le_\vee)$ is a join semilattice where binary joins for $\le_\vee$ are given by $\vee$ and the least element for $\le_\vee$ is $\bot$.
  \end{enumerate}

  \begin{que}
    \label{q6}
    Show the following, for the partial order $(\mathfrak{L}(\mathsf{LML}), \le)$:
    \begin{enumerate}
      \item $(\mathfrak{L}(\mathsf{LML}), \le)$ is a meet semilattice with greatest element $\top$ and binary joins given by
        \begin{align*}
          - \wedge -: \mathfrak{L}(\mathsf{LML}) \times  \mathfrak{L}(\mathsf{LML}) &\longrightarrow  \mathfrak{L}(\mathsf{LML}) \\
          (\phi, \psi) &\longmapsto \phi \wedge \psi\;
        ;\end{align*}
      \item $(\mathfrak{L}(\mathsf{LML}), \le)$ is a join semilattice with least element $\bot$ and binary joins given by
        \begin{align*}
          - \vee -: \mathfrak{L}(\mathsf{LML}) \times  \mathfrak{L}(\mathsf{LML}) &\longrightarrow  \mathfrak{L}(\mathsf{LML}) \\
          (\phi, \psi) &\longmapsto \phi \vee \psi
        .\end{align*}
    \end{enumerate}
  \end{que}

  We will use the lemma proven in question~\ref{q2} (lemma~\ref{lem1}, page~\pageref{lem1}).
  
  \begin{enumerate}
    \item We only need to show that $- \wedge -$ defines a binary meet for $(\mathfrak L(\mathsf{LML}), \le)$ and that $\top$ is a greatest element.

      For any $\phi \in \mathfrak L(\mathsf{LML})$, we have $\phi \le \top$ as $\llbracket \phi\rrbracket \subseteq \llbracket \top\rrbracket = (\mathbf{2}^\mathrm{AP})^\omega$, thus $\top$ is the greatest element.

      For any formulae~$\phi, \psi \in \mathfrak L(\mathsf{LML})$, we have that $\phi \land \psi \le \phi$ and $\phi \land \psi \le \psi$ as both $\llbracket \phi\rrbracket  $ and $\llbracket \psi\rrbracket  $ are supersets of $\llbracket \phi \land \psi\rrbracket = \llbracket \phi\rrbracket  \cap \llbracket \psi\rrbracket$ (by definition of interpretation~$\llbracket -\rrbracket$).
      Then, if $\vartheta \le \phi$ and $\vartheta \le \psi$, we have that $\llbracket \vartheta\rrbracket \subseteq \llbracket \phi\rrbracket$ and $\llbracket \vartheta\rrbracket \subseteq \llbracket \psi\rrbracket$ thus $\llbracket \vartheta\rrbracket \subseteq \llbracket \phi\rrbracket \cap \llbracket \psi\rrbracket = \llbracket \phi \land \psi\rrbracket$, therefore $\vartheta \le \phi\land\psi$.

      We can conclude that $(\mathfrak L(\mathsf{LML}), \le)$ is a meet semilattice with greatest element $\top$ and binary meets given by $- \wedge -$.

    \item We only need to show that $- \vee -$ defines a binary join for $(\mathfrak L(\mathsf{LML}), \le)$ and that $\bot$ is a least element.

      For any $\phi \in \mathfrak L(\mathsf{LML})$, we have $\bot \le \phi$ as $\emptyset \subseteq \llbracket \bot\rrbracket \subseteq \llbracket \phi\rrbracket$, thus $\bot$ is the least element.

      For any formulae~$\phi, \psi \in \mathfrak L(\mathsf{LML})$, we have that $\phi \le \phi \lor \psi$ and $\psi\le \phi \lor \psi$ as both $\llbracket \phi\rrbracket$ and $\llbracket \psi\rrbracket  $ are subsets of $\llbracket \phi \lor \psi\rrbracket = \llbracket \phi\rrbracket  \cup \llbracket \psi\rrbracket$ (by definition of interpretation~$\llbracket -\rrbracket$).
      Then, if $\phi \le \vartheta $ and $\psi \le \vartheta $, we have that $\llbracket \phi\rrbracket \subseteq \llbracket \vartheta\rrbracket$ and $\llbracket \psi\rrbracket \subseteq \llbracket \vartheta\rrbracket$ thus $\llbracket \phi \lor \psi \rrbracket = \llbracket \phi\rrbracket \cup \llbracket \psi\rrbracket  \subseteq \llbracket \vartheta\rrbracket$, therefore $\phi \lor \psi \le \vartheta$.

      We can conclude that $(\mathfrak L(\mathsf{LML}), \le)$ is a join semilattice with least element $\bot$ and binary joins given by $- \vee -$.
  \end{enumerate}


  \begin{que}
    Show that a map of meet (\textup{resp}.\ join) semilattices is monotone.
  \end{que}

  Let $f : L \to L'$ be an arbitrary function where  $(L, \le)$ and $(L', \le')$ are partial orders.

  \begin{enumerate}
    \item Suppose $f : (L, \le) \to (L', \le')$ is a map of meet semilattices.
      Let $a, b \in L$.
      If $a \le b$, then $a \wedge b = a$ and, as $f$ preserves finite meets, \[
      f(a) \wedge' f(b) = f(a \wedge b) = f(a)
      ,\]and thus $f(a) \le' f(b)$.
      Therefore, $f$ is monotone.
    \item Suppose $f : (L, \le) \to (L', \le')$ is a map of join semilattices.
      Let $a, b \in L$.
      If $a \le b$, then $a \vee b = b$ and, as $f$ preserves finite joins, \[
      f(a) \vee' f(b) = f(a \vee b) = f(b)
      ,\]and thus $f(a) \le' f(b)$.
      Therefore, $f$ is monotone.
  \end{enumerate}

  \subsection{Lattices.}

  \begin{que}
    \label{q8}
    Consider the partial order $(L, \sqsubseteq)$ where
    \[
    L \coloneqq \mathds{N} \cup \{\alpha, \beta, \top\} 
    ,\] 
    where $\sqsubseteq$ is the reflexive-transitive closure of $\sqsubset$, where 
    \[
      a \sqsubset b \quad \text{ iff } \quad \begin{cases}
        a < b \text{ in } \mathds{N}\\
        \quad\quad\text{or}\\
        a \in \mathds{N} \text{ and }b \in \{\alpha, \beta\}\\
        \quad\quad\text{or}\\
        a \in \{\alpha, \beta\}  \text{ and } b = \top.
      \end{cases}
    \]
    Show that $(L, \sqsubseteq)$ is a join semilattice but is not a lattice.
  \end{que}

  \begin{figure}[H]
    \centering
    \begin{tikzpicture}[node distance=2.7cm]
      \node (0) {$0$};
      \node[right of=0] (1) {$1$};
      \node[right of=1] (2) {$2$};
      \node[right of=2] (3) {$3$};
      \node[right of=3] (d) {$\ldots$};
      \node[right of=d] (x) {};
      \node[above=0.5cm of x] (a) {$\alpha$};
      \node[below=0.5cm of x] (b) {$\beta$};
      \node[right of=x] (t) {$\top$};
      \draw (0) to node[midway, fill=white, inner sep=0em] {$\sqsubset$} (1);
      \draw (1) to node[midway, fill=white, inner sep=0em] {$\sqsubset$} (2);
      \draw (2) to node[midway, fill=white, inner sep=0em] {$\sqsubset$} (3);
      \draw (3) to node[midway, fill=white, inner sep=0em] {$\sqsubset$} (d);
      \draw (d) to[out=0, in=180] node[midway, fill=white, inner sep=0em, sloped, anchor=center] {$\sqsubset$} (a);
      \draw (d) to[out=0, in=180] node[midway, fill=white, inner sep=0em, sloped, anchor=center] {$\sqsubset$} (b);
      \draw (a) to[out=0, in=180] node[midway, fill=white, inner sep=0em, sloped, anchor=center] {$\sqsubset$} (t);
      \draw (b) to[out=0, in=180] node[midway, fill=white, inner sep=0em, sloped, anchor=center] {$\sqsubset$} (t);
    \end{tikzpicture}
    \caption{Hasse diagram of $(L, \sqsubseteq)$ from question~\ref{q8}}
    \vspace{0.6em}
    \footnotesize\textsf{\textbf{Note:} Hasse diagrams are usually read bottom-to-top,\\ but this one is drawn left-to-right for convenience.}
  \end{figure}

  The relation $\sqsubseteq$ is a partial order. Reflexivity and transitivity is true by definition of $\sqsubseteq$ as the reflexive and transitive closure of $\sqsubseteq$.
  For antisymmetry, we have that:
  \begin{itemize}
    \item for $n,m \in \mathds{N}$,  $n \sqsubseteq m$ iff $n \le m$;
    \item for any $n \in \mathds{N}$ and $m \in L \setminus \mathds{N}$, we have $n \sqsubseteq m$ and $m \not\sqsubseteq n$;
    \item $\alpha \sqsubseteq\top$, $\beta \sqsubseteq \top$, $\top \not\sqsubseteq \alpha$, $\top \not\sqsubseteq \beta$, $\alpha \not\sqsubseteq \beta$ and $\beta \not\sqsubseteq \alpha$;
  \end{itemize}
  (this can be shown by induction on the relation $\sqsubseteq$).

  We have that $0$ is the least element in $(L, \sqsubseteq)$: we have that $0 \sqsubseteq a$ for all~$a \in L$.
  For $a, b \in L$, we can define $a \vee b$ as:
   \begin{itemize}
     \item if $a, b \in \mathds{N}$, let $a \vee b \coloneqq \min_{\le_{\mathds{N}}}(a, b)$;
     \item if $a \in \mathds{N}$ and $b \in L \setminus \mathds{N}$, let $a \vee b, b \vee a \coloneqq b$;
     \item otherwise let $\alpha \vee \beta \coloneqq \top$, $a \wedge a \coloneqq a$, $a \wedge \top, \top \wedge a \coloneqq \top$ for $a \in \{\alpha,\beta, \top \}$.
  \end{itemize}
  Using the previous results on $\sqsubseteq$, we have that $-\vee-$ \textit{really} is a join.

  This concludes the proof that $(L, \sqsubseteq)$ is a join semilattice.

  We also have that $(L, \sqsubseteq)$ is not a lattice.
  Suppose it is a lattice, and consider the element $a \coloneqq \alpha \wedge \beta$.
  Necessarily, we have that $a \in \mathds{N}$ (if $a = \alpha$ then we would have $\alpha \sqsubseteq \beta$, which is false).
  As $a = \alpha \wedge \beta$ and $a + 1 \sqsubseteq \alpha, \beta$ we have, by definition of meet, that $a + 1  \sqsubseteq a$, thus $a + 1 \le a$ (since $a, a+1 \in \mathds{N}$) which is \textit{\textbf{absurd}}.
  We can conclude that $(L, \sqsubseteq)$ is not a lattice.


  \begin{que}
    Consider a set $L$ equipped with two binary operations $\wedge, \vee : L \times L \to L$ and two constants $\top , \bot  \in L$.
    Assume that $(L, \wedge, \top)$ and $(L, \vee, \bot)$ are commutative monoids in which every element is idempotent.
    Show that the following are equivalent.
    \begin{enumerate}
      \item The partial order $\le_\vee$  induced by $(L, \vee, \bot)$ coincides with the partial order $\le_\wedge$ induces by $(L, \wedge, \top)$.
        \label{q9-1}
      \item $(L, \vee, \wedge, \bot, \top)$ satisfies the two following \textbf{absorptive laws}:
        \label{q9-2}
        \begin{align}
          \tag{\textsf{abs}\textsubscript{1}}
          \forall a, b \in L, \quad &a \vee (a \wedge b) = a
          \label{abs1}
          \\
          \tag{\textsf{abs}\textsubscript{2}}
          \forall a, b \in L, \quad &a \wedge (a \vee b) = a
          \label{abs2}
        \end{align}
    \end{enumerate}
  \end{que}

  \begin{itemize}
    \item Let us show that \ref{q9-1} implies \ref{q9-2}.
      Let $a, b \in L$.
      We have that $a \wedge b \le_\wedge a$ and, assuming $\le_\wedge$ and $\le_\vee$ coincide, $a \wedge b \le_\vee a$, thus $a \vee (a \wedge b) = a$, \textit{i.e.}\ (\ref{abs2}) holds.
      Similarly,  $a \le_\vee a \vee b$ thus $a \le_\wedge a \vee b$, so $(a \wedge b) \vee a = a$ holds, and we can recover (\ref{abs1}) by using commutativity.
    \item Let us show that \ref{q9-2} implies \ref{q9-1}.
      \begin{itemize}
        \item Suppose $b \le_\wedge a$, then $b \wedge a = b$.
          By (\ref{abs1}) and commutativity, we have $b \vee a = (b \wedge a) \vee a = a$, thus $b \le_\vee a$.
        \item Suppose $b \le_\vee a$, then $b \vee a = a$.
          By (\ref{abs2}), we have
          \[
          b \wedge a = b \wedge (b \vee a) = b
          ,\] thus $b \le_\wedge a$.
      \end{itemize}
      Thus the two order coincide.
  \end{itemize}

  \begin{que}
    Show that the partial order $(\mathfrak L (\mathsf{LML}), \le )$ is a lattice.
  \end{que}

  We have shown that $(\mathfrak L (\mathsf{LML}), \le )$ has a greatest element $\top$, a least element $\bot$, binary meets given by $- \wedge -$ and binary joins given by  $- \vee -$ (question~\ref{q6}).
  Thus it has all finite meets and finite joins (as seen in question~\ref{q3}),  \textit{i.e.}\ $(\mathfrak{L}(\mathsf{LML}), \le)$ is a lattice.
  \let\nxt\bigcirc

  \begin{que}
    Show that the function
    \begin{align*}
      \nxt: \mathfrak{L}(\mathsf{LML}) &\longrightarrow \mathfrak{L}(\mathsf{LML}) \\
      \phi &\longmapsto \nxt\phi
    \end{align*}
    is a morphism of lattices.
  \end{que}

  We know $\nxt$ is a map of meet iff $\nxt \top = \top$ and  $\nxt (\phi \land \psi) = \nxt \phi \land \nxt \psi$.
  Both are true as,
  \begin{gather*}
    \llbracket \nxt \top\rrbracket  = \big\{\,\sigma \in (\mathbf{2}^\mathrm{AP})^\omega  \;\big|\; \sigma \upharpoonright 1 \in \llbracket \top\rrbracket = (\mathbf{2}^\mathrm{AP})^\omega\,\big\}  = (\mathbf{2}^\mathrm{AP})^\omega = \llbracket \top\rrbracket  \\
    \llbracket \nxt (\phi \land \psi)\rrbracket = \big\{\,\sigma \in (\mathbf{2}^\mathrm{AP})^\omega  \;\big|\; \sigma \upharpoonright 1 \in \llbracket \phi\rrbracket \cap \llbracket \psi\rrbracket  \,\big\}  = \llbracket \nxt \phi\rrbracket   \cap \llbracket \nxt \psi\rrbracket  = \llbracket \nxt \phi \land \nxt \psi\rrbracket  .
  \end{gather*}
  Very similarly, $\nxt$  is a map of joins iff $\nxt \bot = \bot$ and  $\nxt (\phi \lor \psi) = \nxt \phi \lor \nxt \psi$.
  One can show that both equalities hold by applying $\llbracket -\rrbracket$ and showing the equality of the sets like above.

  Thus $\nxt : (\mathfrak L(\mathsf{LML}), \le) \to (\mathfrak L(\mathsf{LML}), \le)$ is a morphism of lattices.

  \subsection{Distributive Lattices.}

  \begin{que}
    Show that the following two \textbf{distributive laws} are equivalent in a lattice $(L, \vee, \wedge, \bot, \top)$:
    \begin{align}
      \tag{\textsf{dist}\textsubscript{1}}
      \forall a, b, c \in L, \quad &a \wedge (b \vee c) = (a \wedge b) \vee (a \wedge c)
      \label{dist1}
      \\
      \tag{\textsf{dist}\textsubscript{2}}
      \forall a, b, c \in L, \quad &a \vee (b \wedge c) = (a \vee b) \wedge (a \vee c)
      \label{dist2}
    \end{align}
  \end{que}

  Suppose (\ref{dist1}) holds and let us show (\ref{dist2}) is true for $a, b, c \in L$:
  \begin{align*}
    (a \vee b) \wedge (a \vee c)
    &= ((a \vee b) \wedge a) \vee ((a \vee b) \wedge c) && \text{by (\ref{dist1})}\\
    &= a \vee ((a \vee b) \wedge c) && \text{by (\ref{abs2})} \\
    &= a \vee (a \wedge b) \vee (b \wedge c) && \text{by (\ref{dist1})} \\
    &= a \vee (b \wedge c) && \text{by (\ref{abs1})} .
  \end{align*}

  To prove that (\ref{dist1}) holds when (\ref{dist2}) is true, we can apply the previous result to the lattice $(L, \le)^\mathrm{op} = (L, \ge)$.
  This gives exactly the implication "(\ref{dist2}) implies (\ref{dist1})," as wanted.

  Thus, the two distributive laws (\ref{dist1}) and (\ref{dist2}) are equivalent.

  \begin{que}
    Show that the lattice $(\mathfrak L(\mathsf{LML}), \le)$ is distributive.
  \end{que}

  Let $\phi, \psi, \vartheta \in \mathfrak{L}(\mathsf{LML})$.
  We have that \[
    \textstyle
  \llbracket \phi \land (\psi \lor \vartheta) \rrbracket 
  = \llbracket \phi\rrbracket \cap (\llbracket \psi\rrbracket \cup \llbracket \vartheta\rrbracket)  
  = (\llbracket \phi\rrbracket \cap \llbracket \psi\rrbracket) \cup (\llbracket \phi\rrbracket \cap \llbracket \vartheta\rrbracket)  
  = \llbracket (\phi \land \psi) \lor (\phi \land \vartheta)\rrbracket 
  ,\] 
  thus $\phi \land (\psi \lor \vartheta) = (\phi \land \psi) \lor (\phi \land \vartheta)$.

  \begin{que}
    Consider the following lattice $\mathrm{M}_3$:
    \begin{figure}[H]
      \centering
      \begin{tikzpicture}[node distance=2cm]
        \node (b) {$b$};
        \node[above of=b] (T) {$\top$};
        \node[below of=b] (B) {$\bot$};
        \node[left of=b] (a) {$a$};
        \node[right of=b] (c) {$c$};
        \draw (B) to (a);
        \draw (B) to (b);
        \draw (B) to (c);
        \draw (a) to (T);
        \draw (b) to (T);
        \draw (c) to (T);
      \end{tikzpicture}
    \end{figure}
    (\textup{i.e.}\ $\bot \le a, b, c \le \top$ with $a,b,c$ incomparable).
    Show that $\mathrm{M_3}$ is not distributive.
  \end{que}

  Suppose $\mathrm{M_3}$ is distributive.
  As $a, b, c$ are incomparable, we have that 
   \[
  a \wedge b = a \wedge c = \bot \quad\quad \text{ and }\quad\quad b \vee c = \top
  ,\]
  and thus,
  \[
  a = a \wedge \top = a \wedge (b \vee c) = (a \wedge b) \vee (a \wedge c) = \bot \vee \bot = \bot
  ,\] 
  which is \textit{\textbf{absurd}}.
  Thus $\mathrm{M_3}$ is not distributive.

  \subsection{Booleans algebras.}

  \begin{que}
    Show that if $(L, \le)$ is a distributive lattice then $a \in L$ has at most one complement.
  \end{que}

  Consider $c, c' \in L$ two complements of $a \in L$.
  Then, we have that 
  \[
    c = c \wedge \top = c \wedge (a \vee c') \overset {\text{(\ref{dist1})}} =
    (c \wedge a) \vee (c \wedge c') = \bot \vee (c \wedge c') = c \wedge c'
  ,\]
  and,
  \[
    c' = c' \wedge \top = c' \wedge (a \vee c) \overset {\text{(\ref{dist1})}} =
    (c' \wedge a) \vee (c' \wedge c) = \bot \vee (c' \wedge c) = c' \wedge c
  .\] 
  We can conclude that $c = c'$ by commutativity of meets.

  \begin{que}
    Show that $(\mathfrak L(\mathsf{LML}), \le)$ is a Boolean algebra.
  \end{que}

  Let us show that $\lnot \phi$ is a complement for  $\phi \in \mathfrak{L}(\mathsf{LML})$.
  We have to check that $\phi \wedge \lnot \phi = \bot$ and $\phi \vee \lnot \phi = \top$ hold.
  Both equalities can be easily checked with interpretations:
  \[
  \llbracket \phi \land \lnot \phi\rrbracket  = \llbracket \phi\rrbracket \cap \llbracket \phi\rrbracket^\complement   = \emptyset = \llbracket \bot\rrbracket  
  ,\] 
  and
  \[
  \llbracket \phi \lor \lnot \phi\rrbracket  = \llbracket \phi\rrbracket \cup \llbracket \phi\rrbracket^\complement = (\mathbf{2}^\mathrm{AP})^\omega = \llbracket \top\rrbracket  
  .\]
  Thus, $\lnot \phi$ is \textit{\textbf{the}} complement of $\phi$ in $(\mathfrak{L}(\mathsf{LML}), \le)$, which is, as a consequence, a Boolean algebra.

  \begin{que}
    Show that the following \textbf{De Morgan Laws} hold in every Boolean algebra $(B, \vee, \wedge, \bot, \top)$:
    \[
    a \wedge b = \lnot (\lnot a \vee \lnot b)
    \quad\quad
    a \vee b = \lnot (\lnot a \wedge \lnot b)
    \quad\quad
    a = \lnot \lnot a
    .\] 
    \label{q17}
  \end{que}

  We start by showing that $\lnot a \lor \lnot b$ is a complement to $(a \land b)$:
  \begin{gather*}
    (a \wedge b) \wedge (\lnot a \lor \lnot b)
    =
    (a \wedge b \wedge \lnot b) 
    \lor
    (a \wedge b \wedge \lnot a) 
    = \bot  \lor \bot 
  = \bot\\
    (a \wedge b) \vee (\lnot a \vee \lnot b)
    = (a \vee \lnot a \vee \lnot b) \wedge (b \vee \lnot a \vee \lnot b)
    = \top \wedge \top = \top 
  ,\end{gather*}
  by using the Boolean algebra laws.
  Thus $a \wedge b = \lnot (\lnot a \vee \lnot b)$.

  For  $a \vee b = \lnot (\lnot a \land \lnot b)$, we proceed by duality by applying the previous result to the Boolean algebra $(B, \wedge, \vee, \top, \bot)$, as a complement for $a$ in $(B, \ge)$ is exactly a complement for $a$ in $(B, \le)$.

  We can easily check that $\lnot \top = \bot$, and then
  \[
  a = a \wedge \top = \lnot (\lnot a \vee \lnot \top) = \lnot (\lnot a \vee \bot) = \lnot \lnot a
  .\] 

  \begin{que}
    Show that if $f$ is a map of Boolean algebras from $(B, \le )$ to $(B', \le')$ then $f$ preserves complements.
  \end{que}

  We have that 
  \[
  \bot ' = f(\bot ) = f(a \wedge \lnot a) = f(a) \wedge' f(\lnot a)
  ,\] 
  and
  \[
  \top = f(\top) = f(a \vee \lnot a) = f(a) \vee' f(\lnot a)
  ,\]
  thus $f(\lnot a)$ is a complement of $f(a)$ and, by unicity, $f(\lnot a) = \lnot f(a)$.
  We can conclude that a map of Boolean algebras preserves complements.

  \begin{comment}
  \pagebreak

  \begin{center}
    \bfseries
    \sffamily

    {\large\itshape ---\hspace{1em}Homework\hspace{1em}---}

    {\huge Semantics and Verification}

    {\large \itshape Hugo SALOU}
  \end{center}
  \end{comment}

  \section{Representation of Boolean Algebras.}
  \subsection{Filters and Ultrafilters.}

  \begin{que}
    Let $(L, \wedge, \top)$ be a meet semilattice. Show that $\mathcal{F} \subseteq L$ is a filter iff
    \begin{enumerate}
      \item $\mathcal{F}$ is upward-closed and,
      \item $\top \in \mathcal{F}$ and,
      \item $a \wedge b \in \mathcal{F}$ whenever $a \in \mathcal{F}$ and $b \in \mathcal{F}$.
    \end{enumerate}
  \end{que}

  Let $\mathcal{F} \subseteq L$ be upward-closed.

  We only need to show that $\mathcal{F}$ is codirected if, and only if, $\top \in \mathcal{F}$ and $a \wedge b \in \mathcal{F}$  whenever $a$ and $b$ are in $\mathcal{F}$.

  \begin{itemize}
    \item If $\mathcal{F}$ is codirected then there exists $a \in \mathcal{F}$ and, as $a \le \top$, we have that $\top \in \mathcal{F}$.
      Also, for $a, b \in \mathcal{F}$, there exists $c \in \mathcal{F}$ such that $c \le a$ and $c \le b$ therefore, as $c \le a \wedge b$, we have that $a \wedge b\in \mathcal{F}$.
    \item Suppose $\top \in \mathcal{F}$ and that $\mathcal{F}$ is stable under the meet operator.
      Let us show that  $\mathcal{F}$ is codirected.
      First, $\top \in \mathcal{F}$ so  $\mathcal{F}$ is non-empty.
      Second, for any two $a, b \in \mathcal{F}$ then $c \coloneqq a\wedge b \in \mathcal{F}$ satisfies that $c \le a, b$.
  \end{itemize}

  \begin{que}
    Let $(L, \le)$ be a lattice. Show that if $F \subseteq L$ has the finite intersection property, then 
    \[
    \mathrm{Filt}(F) \coloneqq \big\{\, a \in L \;\big|\; a \ge \bigwedge S \text{ for some finite } S \subseteq F\,\big\} 
    \] 
    is a proper filter on $(L, \le)$.
  \end{que}

  Suppose $F \subseteq L$ has the finite intersection property.
  \begin{itemize}
    \item First, $\top \in \mathrm{Filt}(F)$ as $\top \ge \bigwedge S$ for, in particular, $S = \emptyset \subseteq F$.
    \item Second, if $a, b \in \mathrm{Filt}(F)$ with $a \ge \bigwedge S_1$, $b \ge \bigwedge S_2$ and finite $S_1, S_2 \subseteq F$, then $a \wedge b \ge (\bigwedge S_1) \wedge (\bigwedge S_2) = \bigwedge (S_1 \cup S_2)$, thus $a \wedge b \in \mathrm{Filt}(F)$ (as $S_1 \cup S_2$ is a finite subset of $F$).
    \item Third, if $a \in \mathrm{Filt}(F)$ (with $a \ge \bigwedge S$ and a finite $S \subseteq F$) and $a \le b$, then $b \ge \bigwedge S$ thus $b \in \mathrm{Filt}(F)$.
  \end{itemize}
  We have that $\mathrm{Filt}(F)$ is a filter. 

  Suppose $\bot \in \mathrm{Filt}(F)$, then there exists some finite $S \subseteq F$ such that $\bot \ge \bigwedge S$.
  Also, $\bot \le \bigwedge S$, so $\bigwedge S = \bot$.
  However, this is absurd by the finite intersection property for $F$.
  Thus $\bot \not\in \mathrm{Filt}(F)$ and we can conclude that $\mathrm{Filt}(F)$ is a proper filter.

  \begin{ques}
    Let $\mathcal{F}$ be a filter on a distributive lattice. Show that if $\mathcal{F}$ is an ultrafilter then $\mathcal{F}$ is prime.
    \label{q21}
  \end{ques}

  \begin{lem}
    Any proper filter has the finite intersection property.
  \end{lem}
  \begin{prv}
    Consider some finite subset $S \subseteq \mathcal{F}$ where $\mathcal{F}$ is a proper filter.
    Then, by induction on the size of $S$, we can show that $\bigwedge S \in \mathcal{F}$, and thus $\bigwedge S \neq \bot$, as $\bot \not\in \mathcal{F}$.
  \end{prv}

  As $\mathcal{F}$ is an ultrafilter, $\mathcal{F}$ is a proper filter and thus $\bot \not\in \mathcal{F}$.
  Now, let us show that if $a \vee b$ is in $\mathcal{F}$ then either $a$ or $b$ is in $\mathcal{F}$.
  Consider three cases.
  \begin{itemize}
    \item Either $\mathcal{F} \cup \{a\}$ has the finite intersection property, and thus ${\mathrm{Filt}(\mathcal{F} \cup \{a\})}$ is a proper filter and $\mathcal{F}\subseteq \mathrm{Filt}(\mathcal{F} \cup \{a\})$ (simply take $S = \{f\}$ for every~$f \in \mathcal{F}$ in the definition of $\mathrm{Filt}(-)$), thus $a \in \mathcal{F} = \mathrm{Filt}(\mathcal{F} \cup \{a\})$ since~$\mathcal{F}$ is an ultrafilter.
    \item Either $\mathcal{F} \cup \{b\}$ has the finite intersection property, and we can show~$b \in \mathcal{F}$ very  similarly.
    \item Either $\mathcal{F} \cup \{a\}$ and $\mathcal{F} \cup \{b\} $ do not have the finite intersection property, \textit{i.e.}\ there exists some finite $S \subseteq \mathcal{F} \cup \{a\}$ and $S' \subseteq \mathcal{F} \cup \{b\}$ such that~ $\bigwedge S = \bigwedge S' = \bot$.
      Necessarily $a \in S$ and $b \in S'$ (if $S \subseteq \mathcal{F}$ then with the above lemma, we immediately have that~$\bigwedge S \neq \bot$, and similarly for $S'$).
      Then, writing $U \coloneqq \bigwedge(S \setminus \{a\})$ and $V \coloneqq \bigwedge (S' \setminus \{b\}) $, we have
      \begin{align*}
        & (a \vee b) \wedge (U \wedge V) \\
        &= (a \wedge (U \wedge V)) \vee (b \wedge (U \wedge V)) && \text{by distributivity, commutativity}  \\
        &= (V \wedge \bot) \vee (U \wedge \bot) && \text{as } a \wedge U = \wedge S = \bot \\
        &= \bot \vee \bot \\
        &= \bot
      .\end{align*}
      However, $a \vee b$, $U$ and $V$ are in $\mathcal{F}$ thus so is $(a \vee b) \wedge U \wedge V = \bot \in \mathcal{F}$.
      This is absurd as $\mathcal{F}$ is a proper filter.
  \end{itemize}

  \begin{que}
    Let $(B, \le)$ be a Boolean algebra and let $\mathcal{F} \subseteq B$ be a filter. Show that the following are equivalent:
    \begin{enumerate}
      \item $\mathcal{F}$ is an ultrafilter; \label{q22a}
      \item $\mathcal{F}$ is prime; \label{q22b}
      \item for each $a \in B$, we have $a \in \mathcal{F}$ iff $\lnot a \not\in \mathcal{F}$. \label{q22c}
    \end{enumerate}
    \label{q22}
  \end{que}

  By question~\ref{q21}, we have that \ref{q22a} implies \ref{q22b}.

  To prove that \ref{q22b} implies \ref{q22c}, consider some $a \in B$: we have $a \vee \lnot a = \top \in \mathcal{F}$, so either $a \in \mathcal{F}$ or $\lnot a \in \mathcal{F}$ $(\star)$.
  If $a \in \mathcal{F}$ then  $\lnot a \not\in \mathcal{F}$ since, if $\lnot a \in \mathcal{F}$ then~$a \wedge \lnot a = \bot \in \mathcal{F}$, a contradiction since $\mathcal{F}$ is supposed prime.
  On the other hand, if $\lnot a \not\in \mathcal{F}$ then necessarily $a \in \mathcal{F}$ by $(\star)$.

  To prove \ref{q22c} implies \ref{q22a}, consider some proper filter $\mathcal{H} \supsetneq \mathcal{F}$, then there exists some $a \in \mathcal{H} \setminus \mathcal{F}$, and so $\lnot a \in \mathcal{F} \subseteq \mathcal{H}$ (applying \ref{q22c} to $\lnot a$ with $\lnot\lnot a = a$).
  Thus $a, \lnot a \in \mathcal{H}$ and so $a \wedge \lnot a = \bot  \in \mathcal{H}$, a contradiction since $\mathcal{H}$ is a proper filter.

  \begin{que}
    Let $(A, \le )$ be a partial order and consider some $\mathcal{F} \subseteq A$.
    \begin{enumerate}
      \item Show that $\mathcal{F}$ is upward-closed if and only if $\chi_{\mathcal{F}}$ is monotone.
      \item Assume that $A$ is a meet semilattice.
        Show that $\mathcal{F}$ is a filter if and only if $\chi_{\mathcal{F}} : A \to \mathbf{2}$ is a morphism of meet semilattices.
      \item Assume that $A$ is a lattice.
        Show that $\mathcal{F}$ is a prime filter if and only if $\chi_{\mathcal{F}} : A \to \mathbf{2}$ is a morphism of lattices.
    \end{enumerate}
  \end{que}

  In the following subquestions, we will implicitly use the results when from the previous subquestions.

  \begin{enumerate}
    \item Suppose $\chi_{\mathcal{F}}$ monotone. Let us show that $\mathcal{F}$ is upward-closed.
      Consider $a \le b$ with $a \in \mathcal{F}$, then $1 = \chi_{\mathcal{F}}(a) \le \chi_{\mathcal{F}}(b)$, thus $\chi_{\mathcal{F}}(b) = 1$ and so $b \in \mathcal{F}$.

      Conversely suppose $\mathcal{F}$ is upward-closed, let us show $\chi_{\mathcal{F}}$ is monotone.
      Consider $a \le b$.
      \begin{itemize}
        \item If $a \not\in \mathcal{F}$ then $0 = \chi_{\mathcal{F}}(a) \le \chi_{\mathcal{F}}(b)$ is true.
        \item If $a \in \mathcal{F}$ then $b \in \mathcal{F}$ and $1 = \chi_{\mathcal{F}}(a) \le \chi_{\mathcal{F}}(b) = 1$.
      \end{itemize}
    \item Suppose $\chi_{\mathcal{F}}$ is a morphism of meet semilattices.
      Let us show that $\mathcal{F}$ is a filter.
      Take $a, b \in \mathcal{F}$, then $\chi_{\mathcal{F}}(a \wedge b) = \chi_{\mathcal{F}}(a) \wedge \chi_{\mathcal{F}}(b) = 1 \wedge 1 = 1$ thus $a \wedge b \in \mathcal{F}$.
      Also, $\chi_{\mathcal{F}}(\top) = \top_{\mathbf{2}} = 1$, thus $\top \in \mathcal{F}$.

      Conversely suppose $\mathcal{F}$ is a filter, then $\chi_{\mathcal{F}}(\top) = \top_{\mathbf{2}} = 1$ as $\top  \in \mathcal{F}$.
      And, for $a, b \in L$,
      \begin{itemize}
        \item if $a, b \in \mathcal{F}$ then $1 = \chi_{\mathcal{F}}(a \wedge b) = \chi_{\mathcal{F}}(a) \wedge \chi_{\mathcal{F}}(b) = 1 \wedge 1$;
        \item if $a \not\in \mathcal{F}$ then $a \wedge b \not\in \mathcal{F}$ (if $a \wedge b \in \mathcal{F}$, then $a \wedge b \le a$ would imply that $a \in \mathcal{F}$), thus $0 = \chi_{\mathcal{F}}(a\wedge b) = \chi_{\mathcal{F}}(a) \wedge \chi_{\mathcal{F}}(b) = 0 \wedge \chi_{\mathcal{F}}(b)$;
        \item similarly if $b \not\in \mathcal{F}$.
      \end{itemize}
    \item Suppose $\chi_{\mathcal{F}}$ is a morphism of lattices.
      Let us show that $\mathcal{F}$ is a proper filter.
      As $\chi_{\mathcal{F}}(\bot ) = \bot_{\mathbf{2}} = 0$, then $\bot \not\in \mathcal{F}$.
      If $a \vee b \in \mathcal{F}$, then \[
      1 = \chi_{\mathcal{F}}(a \vee b) = \chi_{\mathcal{F}}(a) \vee \chi_{\mathcal{F}}(b)
      ,\]  thus necessarily $\chi_{\mathcal{F}}(a) = 1$ or $\chi_{\mathcal{F}}(b) = 1$ (if $\chi_{\mathcal{F}}(a) = \chi_{\mathcal{F}}(b) = 0$ then $\chi_{\mathcal{F}}(a \vee b) = 0$ a contradiction), and so either $a \in \mathcal{F}$ or $b \in \mathcal{F}$.

      Conversely suppose $\mathcal{F}$ is a proper filter.
      Then $\chi_{\mathcal{F}}(\bot) = 0 = \bot_{\mathbf{2}}$ as $\bot \not\in \mathcal{F}$.
      Take $a, b \in L$:
      \begin{itemize}
        \item if $a, b \not\in \mathcal{F}$ then $0 = \chi_{\mathcal{F}}(a \vee b) = \chi_{\mathcal{F}}(a) \vee \chi_{\mathcal{F}}(b) = 0 \vee 0$ (if $a \vee b \in \mathcal{F}$ then either $a \in \mathcal{F}$ or $b \in \mathcal{F}$);
        \item if $a \in \mathcal{F}$ then $a \le a \vee b$ implies $a \vee b \in \mathcal{F}$ and so $1 = \chi_{\mathcal{F}}(a \vee b) = \chi_{\mathcal{F}}(a) \vee \chi_{\mathcal{F}}(b) = 1 \vee \chi_{\mathcal{F}}(b)$;
        \item similarly if $b \in \mathcal{F}$.
      \end{itemize}
  \end{enumerate}

  In these subquestions, we also proved the following result:
  \begin{lem}
    For $a, b \in L$, and $\mathcal{F} \subseteq L$,
    \begin{itemize}
      \item if $\mathcal{F}$ is a filter, then $a \wedge b \in \mathcal{F}$ iff $a, b \in \mathcal{F}$ ;
      \item if $\mathcal{F}$ is a prime filter, then $a \vee b \not\in \mathcal{F}$ iff $a, b \not\in \mathcal{F}$.
    \qed
    \end{itemize}
  \end{lem}

  \subsection{The Spectrum of a Boolean Algebra.}

  \begin{que}
    Let $(B, \le)$ be a Boolean algebra.
    Show that we have:
    \begin{align*}
      \mathsf{ext}(a \wedge b) &= \mathsf{ext}(a) \cap \mathsf{ext}(b)\\
      \mathsf{ext}(a \vee b) &= \mathsf{ext}(a) \cup \mathsf{ext}(b)\\
      \mathsf{ext}(\lnot a) &= \mathbf{Sp}(B) \setminus \mathsf{ext}(a) \\
      \mathsf{ext}(\top) &= \mathbf{Sp}(B) \\
      \mathsf{ext}(\bot) &= \emptyset
    .\end{align*}
  \end{que}
  Let $\mathcal{F} \in \mathbf{Sp}(B)$.
  We have that :
  \begin{gather*}
    \mathcal{F} \in \mathsf{ext}(a \wedge b) \iff a \wedge b \in \mathcal{F} \iff a, b \in \mathcal{F} \iff \mathcal{F} \in \mathsf{ext}(a) \cap \mathsf{ext}(b) ; \\
    \mathcal{F} \not\in \mathsf{ext}(a \vee b) \iff a \vee b \not\in \mathcal{F} \iff a, b \not\in \mathcal{F} \iff \mathcal{F} \not\in \mathsf{ext}(a) \cup \mathsf{ext}(b).
  \end{gather*}
  If $\mathcal{F} \not\in \mathsf{ext}(\top)$ then $\top \not\in  \mathcal{F}$ which is absurd.
  If $\mathcal{F} \in \mathsf{ext}(\bot )$ then $\bot \in \mathcal{F}$ which is impossible since $\mathcal{F}$ is a proper filter.
  A map $\mathsf{ext} : (B, \le) \to (\wp(\mathbf{Sp}(B)), \subseteq)$ of Boolean algebras automatically preserves complements, thus $\mathsf{ext}(\lnot a) = \mathbf{Sp}(B) \setminus \mathsf{ext}(a)$.

  \begin{que}
    Show that the spectrum $(\mathbf{Sp}(B), \Omega(\mathbf{Sp}(B)))$ of a Boolean algebra $B$ is $\mathrm{T_0}$ and zero-dimensional.
  \end{que}

  Let us show that $\mathbf{Sp}(B)$ is $\mathrm{T_0}$.
  Take $F \neq G \in \mathbf{Sp}(B)$, and $a \in F \mathbin{\triangle} G$ where~$- \mathbin{\triangle}-$ is the symmetric difference of two sets.
  Without loss of generality, assume $a \in F$ (and thus $a \not\in G$), then $F \in \mathsf{ext}(a)$ and $G \not\in \mathsf{ext}(a)$.

  Let us show that $\mathbf{Sp}(B)$ is zero-dimensional.
  Every $\mathsf{ext}(a) \in \mathcal{B}$ is a clopen, as it is obviously open, and $\mathsf{ext}(a) = \mathsf{ext}(\lnot \lnot a) = \mathbf{Sp}(B) \setminus \mathsf{ext}(\lnot a)$, thus it is also closed.
  So, the basis $\mathcal{B}$ only contains clopens.

  \begin{ques}
    Show that the spectrum $(\mathbf{Sp}(B), \Omega(\mathbf{Sp}(B)))$ of a Boolean algebra $B$ is compact.
  \end{ques}

  \begin{note}
    In the following questions, we will only deal with open covers, so a "cover" will always be an open cover.
    Similarly a "subcover" will always refer to an open subcover.
  \end{note}

  Suppose $\mathbf{Sp}(B) = \bigcup_{a \in A} \mathsf{ext}(a)$ (we can always do that as $\mathcal{B}$ is a basis of the topology on $\mathbf{Sp}(B)$).
  Assume this cover of $\mathbf{Sp}(B)$ does not admit a finite subcover.
  Define 
  \[
  \mathcal{C} \coloneqq \{\lnot a  \mid a \in A\} 
  .\]
  The set $\mathcal{C}$ has the finite intersection property: if $S$ is a finite subset of $\mathcal{C}$ with $\bigwedge S = \bot$, then
  \[
    \bot = \bigwedge_{(\lnot a) \in S} (\lnot a) \overset {(\star)} = \lnot \Big(\bigvee_{(\lnot a) \in S} a\Big)
  ,\] where $(\star)$ is proven by induction with results from question~\ref{q17}, as $S$ is finite.
  Taking the complement, we have $\top = \bigvee_{(\lnot a) \in S} a$, thus 
  \[
  \mathbf{Sp}(B) = \mathsf{ext}(\top) = \mathsf{ext}\Big(\bigvee_{(\lnot a) \in S} a\Big) = \bigcup_{(\lnot a) \in S} \mathsf{ext}(a)
  \] 
  is a finite subcover of $\bigcup_{a \in A} \mathsf{ext}(a)$, which is absurd.
  Thus $\mathcal{C}$ has the finite intersection property.
  With the Ultrafilter Lemma, we get an ultrafilter $\mathcal{F} \supseteq \mathcal{C}$.
  For all $a \in A$, we have $\lnot a \in \mathcal{C} \subseteq \mathcal{F}$ thus $a \not\in \mathcal{F}$ (question~\ref{q22}), and so $\mathcal{F} \not\in \mathsf{ext}(a)$.
  So, we have $\mathcal{F} \not\in \mathbf{Sp}(B) = \bigcup_{a \in A} \mathsf{ext}(a)$, a contradiction.
  Thus, the cover $\mathbf{Sp}(B) = \bigcup_{a \in A} \mathsf{ext}(a)$ admits a finite subcover.
  We can conclude that $\mathbf{Sp}(B)$ is compact.

  \begin{que}
    Let $(X, \Omega)$ be a topological space. Show that $(X, \Omega)$ is compact if and only if we have $\bigcap \mathcal{F} \neq\emptyset $ for every family of closed sets $\mathcal{F}$ which has the finite intersection property (w.r.t.\ the inclusion (partial) order on closed sets).
    \label{q27}
  \end{que}

  For any family $\mathcal{G} \subseteq \wp(X)$, we will write $\bar{\mathcal{G}} \coloneqq \{X \setminus G  \mid G \in \mathcal{G}\}$.
  We have that $\mathcal{G}$ is a family of closed sets iff $\bar{\mathcal{G}}$ is a family of open sets (\textit{i.e.}\ $\bar{\mathcal{G}} \subseteq \Omega$).
  Also, we have that $\bar{\bar{\mathcal{G}}} = \mathcal{G}$.

  Firstly, suppose $(X, \Omega)$ to be compact, and let $\mathcal{F} \subseteq \wp(X)$ be a family of closed sets with the finite intersection property.
  Suppose $\bigcap \mathcal{F} = \emptyset$.
  Then, $\bigcup \bar{\mathcal{F}} = X$ and, as $X$ is compact, there exists some finite $\mathcal{G} \subseteq \bar{\mathcal{F}}$ such that $\bigcup \mathcal{G} = X$.
  Thus we get that $\bigcap \bar{\mathcal{G}} = \emptyset$, a contradiction with the finite intersection property, as $\bar{\mathcal{G}}$ is finite.
  We conclude that $\bigcap \mathcal{F} \neq \emptyset$.

  Secondly, suppose $X = \bigcup \mathcal{F}$ with $\mathcal{F} \subseteq \Omega$ is an open cover of $X$.
  Assume that there are no finite subcovers $(\star)$.
  Then, by complement, we have $\bigcap \bar{\mathcal{F}} = \emptyset$ $(\star\star)$.
  But, $\bar{\mathcal{F}}$ is a family of closed sets and $\bar{\mathcal{F}}$ has the finite intersection property: if there exists some finite $\mathcal{G} \subseteq \bar{\mathcal{F}}$ with $\bigcap \mathcal{G} = \emptyset$ then $\bigcup \bar{\mathcal{G}} = X$  is a finite cover of $X$ with open sets, which is absurd by $(\star)$.
  So we conclude that $\bar{\mathcal{F}}$ has the finite intersection property, and we can apply the hypothesis to get that $\bigcap \bar{\mathcal{F}} \neq \emptyset$, a contradiction with $(\star\star)$.
  Thus a finite subcover exists, and we finally have that $X$ is compact.

  \begin{ques}
    Given a Stone space $(X, \Omega)$, consider the function
    \begin{align*}
      \eta: X  &\longrightarrow \wp(\mathbf{K}\Omega) \\
      x &\longmapsto \{U \in \mathbf{K}\Omega  \mid x \in U\}
    .\end{align*}
    Show that $\eta$ is a continuous bijection from $X$ to $\mathbf{Sp}(\mathbf{K}\Omega)$.
  \end{ques}

  We will proceed in four parts.

  Firstly, let us show that $\eta(x) \in \mathbf{Sp}(\mathbf{K}\Omega)$, \textit{i.e.}\ $\eta(x)$ is a prime filter on  $\mathbf{K}\Omega$.
  \begin{itemize}
    \item Suppose $A \in \eta(x)$ and $A \subseteq B \in \mathbf{K}\Omega$, then $x \in A \subseteq B$, we have $x \in B$ and so $B \in \eta(X)$.
    \item We have that $X \in \eta(x)$ as $x \in X$ and $X$ is a clopen.
    \item Suppose $A, B \in \eta(x)$ then $x \in A$ and $x \in B$, so $x \in A \cap B$ and thus $A \cap B \in \eta(x)$.
  \end{itemize}
  Therefore $\eta(x)$ is a filter on  $\mathbf{K}\Omega$.
  \begin{itemize}
    \item Suppose $\bot_{\mathbf{K}\Omega} = \emptyset \in \eta(x)$, then $x \in \emptyset$, which is absurd, so $\eta(x)$  is a proper filter.
    \item Suppose $A \cup B \in \eta(x)$, then $x \in A \cup B$ so either $x \in A$ or $x \in B$, and thus either $A \in \eta(x)$ or $B \in \eta(x)$.
  \end{itemize}
  We can conclude that  $\eta : X \to \mathbf{Sp}(\mathbf{K}\Omega)$.

  Secondly, let us show that $\eta$ is injective.
  Take $x \neq y$ in $X$.
  By $\mathrm{T_0}$, we have that there exists an open set $U \in \Omega$ with $x \in U$ and $y \not\in U$, even if it requires swapping $x$ and $y$.
  As $(X, \Omega)$ is zero-dimensional, we can write  $U= \bigcup_{i \in  I} C_i$ where $C_i \in \mathbf{K}\Omega$.
  Let $C \in \{C_i  \mid i \in I\} \subseteq \mathbf{K}\Omega$ such that $x \in C_i$.
  We also have that $y \not\in C$, thus $C \in \eta(x)$ and $C \not\in \eta(y)$, so $\eta(x) \neq \eta(y)$.

  Thirdly let us show that $\eta$ is surjective.
  Take $\mathcal{F} \in \mathbf{Sp}(\mathbf{K}\Omega)$.
  Then, from question~\ref{q27} (as $X$ is, by definition, compact and $\mathcal{F}$ is a set of closed sets, with the finite intersection property), we have that $\bigcap \mathcal{F} \neq \emptyset$.
  Take $x \in \bigcap \mathcal{F} \neq \emptyset$.
  For every $U \in \mathcal{F}$, $x \in U$ so $U \in \eta(x)$ and thus $\mathcal{F} \subseteq \eta(x)$.
  Conversely suppose $V \in \mathbf{K}\Omega$ such that $x \in V$ and $V \not\in \mathcal{F}$, then $X \setminus V \in \mathcal{F}$ by question~\ref{q22}(\ref{q22c}), and so $x \in \bigcap \mathcal{F} \subseteq X \setminus V$ and $x \not\in X \setminus V$, a contradiction.
  We can thus conclude $\mathcal{F} = \eta(x)$, and so $\eta$ is surjective.

  Finally, let us show that $\eta$ is continuous.
  To show that $\eta^\bullet(V)$ is open for every  $V \in \Omega (\mathbf{Sp}(\mathbf{K}\Omega))$, it  suffices to consider $V \in \mathcal{B}$ as $\eta^\bullet$ commutes with arbitrary unions.
  Let  $A \in \mathbf{K}\Omega$, and let us show that $\eta^\bullet(\mathsf{ext}(A))$ is open in $X$:
  \[
    \eta^\bullet(\mathsf{ext}(A)) = \{x \in X  \mid \eta(x) \in \mathsf{ext}(A)\} 
    = \{x \in X  \mid \eta(x) \ni A\} 
    = \{x \in X  \mid x \in A\}
  ,\] 
  which exactly is $A \in \mathbf{K}\Omega \subseteq \Omega$.

  Thus we can conclude that $\eta$ is a continuous bijection from $X$ to $\mathbf{Sp}(\mathbf{K}\Omega)$.

  \begin{ques}
    Assume $(X, \Omega X)$ and $(Y, \Omega Y)$ are compact Hausdorff spaces. Show that if $f : (X, \Omega X) \to (Y, \Omega Y)$ is a continuous bijection, then $f$ is an homeomorphism.
  \end{ques}

  We will us the following lemma from the course:
  \begin{lem}
    If $(X, \Omega X)$ is a compact Hausdorff space, and $C \subseteq X$, then $C$~is compact iff $C$ is closed.
    \qed
  \end{lem}

  We also need the following lemma:
  \begin{lem}
    Let $(X, \Omega X)$ and  $(Y, \Omega Y)$ be arbitrary topological spaces.
    If $K$ is compact in $(X, \Omega X)$ then, for every continuous  $f : (X, \Omega X) \to (Y, \Omega Y)$, the set $f_!(K)$ is compact in  $(Y, \Omega Y)$.
  \end{lem}
  \begin{prv}
    Define $L \coloneqq f_!(K)$.
    Consider a cover $L \subseteq \bigcup_{i \in  I} V_i$ of $L$.
    Then, $f^\bullet(L) \subseteq f^\bullet(\bigcup_{i \in  I} V_i) = \bigcup_{i \in  I} f^\bullet(V_i)$, so we obtain an open (as $f$ is continuous) cover of $f^\bullet(L)$.
    As $K \subseteq f^\bullet(K) \subseteq \bigcup_{i \in  I} f^\bullet(V_i)$ and $K$ is compact, then there exists a finite set $J \subseteq I$ such that $K \subseteq \bigcup_{i \in J} f^\bullet(V_i)$. With the direct image $f_!$, we have that $f_!(K) = L \subseteq \bigcup_{i \in  J} f_!(f^\bullet(V_i)) = \bigcup_{i \in J} V_i$ is a finite subcover of $L = f_!(K)$.
  \end{prv}

  We simply have to show that $g \coloneqq f^{-1} : Y \to X$ is continuous.
  We have that~$g_! = f^\bullet$ and  $f_! = g^\bullet$ as $f$ and $g$ are inverses of each other.
  Consider an arbitrary open set $U \in \Omega X$.
  It suffices to show that $g^\bullet (U) = f_! (U)$ is open in $(Y, \Omega Y)$.
  With the two previous lemmas, we have that $f_!(C)$ is a closed set in $(Y, \Omega Y)$, for every closed $C$ in $(X, \Omega X)$. 
  So, \[
    f_!(X \setminus U) = g^\bullet(X \setminus U) = g^\bullet(X) \setminus g^\bullet(U) = Y \setminus f_!(U)
  \] 
  is closed in $(Y, \Omega Y)$, thus  $f_!(U)$ is open.
  Thus $g$ is continuous.

  We can conclude that $f$ is a homeomorphism.

  \subsection{On the Ultrafilter Lemma.}

  \begin{ques}
    Prove the Ultrafilter Lemma (assuming Zorn's Lemma).
  \end{ques}

  Consider some $F \subseteq L$ with the finite intersection property where $(L, \le)$ is a lattice.
  Define $P \coloneqq \{\mathcal{F} \text{ is a proper filter on } L  \mid F \subseteq \mathcal{F}\}$, ordered by set inclusion $\subseteq$.

  Consider a non-empty chain $\mathcal{C}$ in $P$.
  We will show that $\mathcal{G} \coloneqq \bigcup \mathcal{C}$ is a proper filter containing $F$.
  \begin{itemize}
    \item Take $a \le b$ with $a \in \mathcal{G}$.
      Then $a \in \mathcal{F}$ for some $\mathcal{F} \in \mathcal{C}$.
      As $\mathcal{F}$ is a filter, then $b \in \mathcal{F}$, and so $b \in \mathcal{G}$.
    \item We have $\top \in \mathcal{G}$ as $\mathcal{C}$ is non-empty, and any element of $\mathcal{C}$ contains $\top$.
    \item We have $F \subseteq \mathcal{G}$ as $\mathcal{C}$ is non-empty, and any element of $\mathcal{C}$ contains $F$.
    \item Take $a, b \in \mathcal{G}$, then $a \in \mathcal{F}_1$ and $b \in \mathcal{F}_2$ for some proper filters $\mathcal{F}_1, \mathcal{F}_2 \in \mathcal{C}$.
      As $\mathcal{C}$ is a chain, we can assume without loss of generality, that $\mathcal{F}_1 \subseteq \mathcal{F}_2$, thus $a, b \in \mathcal{F}_2$ and so $a \wedge b \in \mathcal{F}_2 \subseteq \mathcal{G}$.
    \item If $\bot  \in \mathcal{G}$ then $\bot  \in \mathcal{F}$ for some $\mathcal{F}$ in $\mathcal{C}$, which is absurd as $\mathcal{F}$ is a proper filter.
  \end{itemize}
  Now, we can immediately see that this is an upper bound of $\mathcal{C}$: if $\mathcal{F} \in \mathcal{C}$ then we have $\mathcal{F} \subseteq \mathcal{G} = \bigcup \mathcal{C}$.

  For an empty chain $\mathcal{C}$, we use $\mathrm{Filt}(F) \in P$ as upper bound (as $F$ has the finite intersection property).

  Then, we apply Zorn's lemma to $(P, \subseteq)$, and get a maximal element $\mathcal{U} \in P$.
  Let us show that $\mathcal{U}$ is an ultrafilter.
  Consider $\mathcal{H}$ a proper filter on $L$ such that $\mathcal{H} \supseteq \mathcal{U}$.
  Then $F \subseteq \mathcal{U} \subseteq \mathcal{H}$ so $\mathcal{H} \in P$.
  By maximality of $\mathcal{U}$ in $P$, we have that $\mathcal{U} = \mathcal{H}$.

  We can finally conclude that $\mathcal{U}$ is an ultrafilter containing $F$, finishing the proof of the Ultrafilter Lemma.

  \vfill

  \begin{center}
    \color{deepblue}
    \boxed{
      \textbf{\textit{End of Homework.}}
    }
  \end{center}

  \vfill
\end{document}
