\documentclass[./main]{subfiles}

\begin{document}
  \chapter{Transition systems.}

  \section{Transition systems.}

  \begin{defn}
    A transition system is a tuple
    \[
      TS = (S, \mathrm{Act}, {\to}, I, \mathrm{AP}, L)
    \] 
    where
    \begin{itemize}
      \item $S$ is the set of \textit{states};
      \item $\mathrm{Act}$ is the set of \textit{actions};
      \item ${\to} \subseteq S \times \mathrm{Act} \times S$ the \textit{transition relation};
      \item $I \subseteq S$ the set of \textit{initial states};
      \item $\mathrm{AP}$ is the set of \textit{atomic propositions};
      \item $L : S \to \wp(\mathrm{AP}) \cong \mathbf{2}^{\mathrm{AP}}$ is the \textit{state labelling function}.
    \end{itemize}
    We will write $s \tr\alpha s'$ when $(s, \alpha, s') \in {\to}$.
  \end{defn}

  \begin{exm}[Beverage Vending Machine, BVM]
    We can model a beverage vending machine using a diagram like in figure~\ref{fig:bvm-ts}.
    Here we have that:
    \begin{itemize}
      \item $S = \{\mathsf{pay}, \mathsf{select}, \mathsf{soda}, \mathsf{beer}\}$,
      \item $I = \{\mathsf{pay}\}$,
      \item $\mathrm{Act} = \{\mathtt{ic}, \tau, \mathtt{gb}, \mathtt{gs}\}$.\footnote{The meaning of the actions are the following: $\mathtt{ic}$ means \textit{insert coin}, $\mathtt{gb}$ means \textit{get beer} and $\mathtt{gs}$ for \textit{get soda}.}
    \end{itemize}
    We can define the labels:
    \[
    \scriptsize L(\mathsf{pay}) = \emptyset \quad L(\mathsf{soda}) = L(\mathsf{beer}) = \{\mathsf{paid}, \mathsf{drink}\}  \quad L(\mathsf{select}) = \{\mathsf{paid}\}
    ,\]
    with $\mathrm{AP} = \{\mathsf{paid}, \mathsf{drink}\}$.
  \end{exm}
  \begin{figure}
    \centering
    \begin{tikzpicture}
      \node[state, initial, initial where=above] (pay) {$\mathsf{pay}$};
      \node[state, below of=pay](select){$\mathsf{select}$};
      \node[state, left of=select](soda){$\mathsf{soda}$};
      \node[state, right of=select](beer){$\mathsf{beer}$};
      \draw[->] (soda) edge node {$\mathtt{gs}$} (pay);
      \draw[->] (beer) edge node[swap] {$\mathtt{gb}$} (pay);
      \draw[->] (select) edge node {$\tau$} (soda);
      \draw[->] (pay) edge node {$\mathtt{ic}$} (select);
      \draw[->] (select) edge node[swap] {$\tau$} (beer);
    \end{tikzpicture}
    \caption{Transition system for the BVM}
    \label{fig:bvm-ts}
  \end{figure}

  \section{Program graphs.}

  The goal is to represent the evaluation of a program.

  \begin{defn}[Typed variables]
    \begin{itemize}
      \item A set $\mathrm{Var}$ of \textit{variables}.
      \item For each variable $x \in \mathrm{Var}$, consider a set $\mathrm{Dom}(x)$.
      \item Given $TV = (\mathrm{Var}, (\mathrm{Dom}(x))_{x \in \mathrm{Var}})$, we define
        \[
          \mathrm{Eval}(TV) = \prod_{x \in \mathrm{Var}} \mathrm{Dom}(x)
        ,\]
        the set of valuations of the form $\eta : x \in \mathrm{Var} \mapsto \eta(x) \in \mathrm{Dom}(x)$ (in the sense of a dependent function type).
    \end{itemize}
  \end{defn}

  \begin{defn}[Program graph]
    A \textit{program graph} is a tuple \[
      PG = (\mathrm{Loc}, \mathrm{Act}, \mathrm{Effect}, {\hookrightarrow}, \mathrm{Loc}_0, g_0)
    ,\] 
    where
    \begin{itemize}
      \item $\mathrm{Loc}$ is the set of \textit{locations} (lines of codes);
      \item $\mathrm{Act}$ is the set of \textit{actions};
      \item $\mathrm{Effect} : \mathrm{Act} \times \mathrm{Eval}(TV) \to \mathrm{Eval}(TV)$;
      \item ${\hookrightarrow} \subseteq \mathrm{Loc} \times \mathrm{Conditions} \times \mathrm{Act} \times \mathrm{Loc}$ where conditions are propositional formula built from atoms of the forms "$x \in D$" for some variable $x$ and some set $D \subseteq \mathrm{Dom}(x)$;
      \item $\mathrm{Loc}_0 \subseteq \mathrm{Loc}$ the set of \textit{initial locations};
      \item $g_0$ is the \textit{initial condition}.
    \end{itemize}
    We will write $\ell \trh{g : \alpha} \ell'$ for $(\ell, g, \alpha, \ell') \in {\hookrightarrow}$.
  \end{defn}

  \begin{figure}
    \centering
    \begin{tikzpicture}[rotate=-45,transform shape]
      \node (start) {$\mathsf{start}$};
      \node[below of=start] (temp) {};
      \node[below of=temp] (select) {$\mathsf{select}$};
      \begin{scope}[transform canvas={xshift=0.3em,yshift=-0.3em}]
        \draw[arrows = {Hooks[left]->}] (select) to (start);
        \node[right=0.1cm of temp,rotate=90,anchor=center]{\footnotesize$\mathtt{nb}>0:\mathtt{gb}$};
      \end{scope}
      \begin{scope}[transform canvas={xshift=-0.3em,yshift=0.3em}]
        \draw[arrows = {Hooks[right]->}] (select) to (start);
        \node[left=0.1cm of temp,rotate=90,anchor=center]{\footnotesize$\mathtt{ns}>0:\mathtt{gs}$};
      \end{scope}
      \node[left=1.2cm of temp,inner sep=0,outer sep=0](temp1){};
      \node[right=1.2cm of temp,inner sep=0,outer sep=0](temp2){};
      \draw[rounded corners,arrows = {Hooks[right]-}, straight-left] (select) to (temp1.center);
      \draw[rounded corners,->,straight-right] (temp1.center) to (start);
      \draw[rounded corners,->, straight-right] (temp2.center) to (select);
      \draw[rounded corners,arrows = {Hooks[right]-},straight-left] (start) to (temp2.center);
      \node[left=0.2cm of temp1,rotate=90,anchor=center]{\footnotesize$\mathtt{ns} \in \{0\} \land \mathtt{nb} \in \{0\} :\mathtt{nt}$};
      \node[right=0.2cm of temp2,rotate=90,anchor=center]{\footnotesize$\mathsf{true}:\mathtt{ic}$};
    \end{tikzpicture}
    \caption{BVM as a program graph}
    \label{fig:bvm-pg}
  \end{figure}

  \begin{exm}[BVM as a program graph]
    In figure~\ref{fig:bvm-pg}, we use 
    \begin{itemize}
      \item $\mathrm{Loc} = \{\mathsf{start},\mathsf{select}\}$;
      \item $\mathrm{Var} = \{\mathtt{ns}, \mathtt{nb}\}$;
      \item $\mathrm{Act} = \{\mathtt{ic}, \mathtt{nt}, \mathtt{gs}, \mathtt{gb}, \mathtt{refill}\}$;
      \item $\mathrm{Loc}_0 = \{\mathsf{start}\}$;
      \item $g_0 = \mathtt{ns} \in  \{\max\} \land \mathtt{nb} \in \{\max\}$ 
      \item 
        \begin{align*}
          \mathrm{Effect}: \mathrm{Act} \times \mathrm{Eval}(TV) &\longrightarrow \mathrm{Eval}(TV) \\
          (\mathtt{refill}, \eta) &\longmapsto[\mathtt{ns} \mapsto {\max}, \mathtt{nb} \mapsto {\max}]\\
          (\mathtt{gs}, \eta) &\longmapsto \eta[\mathtt{ns} \mapsto \eta(\mathtt{ns}) - 1]\\
          (\mathtt{gb}, \eta) &\longmapsto \eta[\mathtt{nb} \mapsto \eta(\mathtt{nb}) - 1]\\
        \end{align*}
    \end{itemize}
  \end{exm}

  \section{Transition system of a program graph.}

  \begin{figure}[p]
    \centering
    \begin{tikzpicture}
      \node[state,initial] (start11) {$\mathsf{start}, \mathtt{ns} = 1, \mathtt{nb} = 1$};
      \node[state,below of=start11] (select11) {$\mathsf{select}, \mathtt{ns} = 1, \mathtt{nb} = 1$};
      \node[state,below left=1cm and 0cm of select11] (start01) {$\mathsf{start}, \mathtt{ns} = 0, \mathtt{nb} = 1$};
      \node[state,below right=1cm and 0cm of select11] (start10) {$\mathsf{start}, \mathtt{ns} = 1, \mathtt{nb} = 0$};
      \node[state,below of=start01] (select01) {$\mathsf{select}, \mathtt{ns} = 0, \mathtt{nb} = 1$};
      \node[state,below of=start10] (select10) {$\mathsf{select}, \mathtt{ns} = 1, \mathtt{nb} = 0$};
      \node[state,below=5cm of select11] (start00) {$\mathsf{start}, \mathtt{ns} = 0, \mathtt{nb} = 0$};
      \draw[->] (start11) to[loop above] node{$\mathtt{refill}$} (start11); 
      \draw[->] (start11) to node{$\mathtt{ic}$} (select11);
      \draw[->] (select11) to node{$\mathtt{gs}$} (start01);
      \draw[->] (select11) to node[swap]{$\mathtt{gb}$} (start10);
      \draw[->] (start01) to node{$\mathtt{ic}$} (select01);
      \draw[->] (start10) to node{$\mathtt{ic}$} (select10);
      \draw[->] (select01) to node[swap]{$\mathtt{gb}$} (start00);
      \draw[->] (select10) to node{$\mathtt{gs}$} (start00);
      \node[left=0.5cm of start01, inner sep=0,outer sep=0] (temp) {};
      \draw[rounded corners] (start00) edge[straight-left] (temp.center)
        (temp.center) edge[straight-right, ->] (start11);
      \node[left=0cm of temp] {$\mathtt{refill}$};
      \draw[->] (start01) to[bend left] node{$\mathtt{refill}$} (start11);
      \draw[->] (start10) to[bend right] node[swap]{$\mathtt{refill}$} (start11);
    \end{tikzpicture}
    \caption{Transition system of the BVM program graph}
    \label{fig:bvm-pg-ts}
  \end{figure}

  \begin{defn}
    Given $TV$ and $PG$ a program graph, we define 
    \[
      TS(PG) := (S, \mathrm{Act}, {\to}, I, \mathrm{AP}, L)
    \]
    where
    \begin{itemize}
      \item $S = \mathrm{Loc} \times \mathrm{Eval}(TV)$;
      \item $\mathrm{AP} = \mathrm{Loc} \cup \mathrm{Conditions}$ ;
      \item $I = \{(\ell_0, \eta)  \mid \ell_0 \in \mathrm{Loc}_0, \eta \models g_0\}$;
      \item $\to$ is defined by:
        \[
        \begin{prooftree}
          \hypo{\ell \trh{g : \alpha} \ell'}
          \hypo{\eta \models g}
          \infer 2{(\ell, \eta) \tr \alpha (\ell', \mathrm{Effect}(\alpha, \eta))}
        \end{prooftree}
        ,\] 
      \item and $L(\ell, \eta) = \{\ell\} \cup \{g  \mid \eta \models g\}$.
    \end{itemize}
  \end{defn}

  \begin{exm}
    The BVM program graph example seen in the previous example can be transformed as a transition system thanks to the previous definition; it is shown in figure~\ref{fig:bvm-pg-ts}.
    To simplify, we assume $\max = 1$.
  \end{exm}
\end{document}
