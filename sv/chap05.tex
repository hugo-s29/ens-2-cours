\documentclass[./main]{subfiles}

\begin{document}
  \selectlanguage{french}
  \setquotestyle{french}

  \chapter{Propriétés "observables".}

  La terminologie "propriété observable" n'est pas utilisée dans la littérature, mais c'est en réalité la compacité.

  \begin{rmk}[Rappel]
    Si $f : X \to Y$ alors  \[
    f_! \dashv f^\bullet: \wp(X) \to \wp(Y)
    ,\]
    où $f_!$ est l'image directe, et $f^\bullet$ est l'image réciproque.

    Ainsi, $f^\bullet : \wp(Y) \to \wp(X)$  préserve les intersections (\textit{i.e.} si $\mathcal{S}\subseteq \wp(Y)$ alors on a que$f^\bullet (\bigcap \mathcal{S}) = \bigcap_{S \in \mathcal{S}}  f^\bullet(S)$).

    De plus, $f^\bullet$ préserve les unions car $f^\bullet \dashv f_\bullet : \wp(Y) \to \wp(X)$ où
     \begin{align*}
      f_\bullet: \wp(X) &\longrightarrow \wp(Y) \\
      A &\longmapsto \bigcup \{B \subseteq Y  \mid f^\bullet(B) \subseteq A\} 
    .\end{align*}
  \end{rmk}

  \begin{defn}
    Soient $(X, \Omega X)$ et $(Y, \Omega Y)$ deux espaces topologiques.
    Une fonction $f : X \to Y$ est \textit{continue} si $f^\bullet : \wp(Y) \to \wp(X)$ se restreint en une fonction $f^\bullet : \Omega Y \to \Omega X$, autrement dit 
    \[
    \forall  V \in \Omega Y, \quad\quad f^\bullet(V) = \{x \in X  \mid f(x) \in V\}  \in \Omega(X)
    .\]

    On définie ainsi une catégorie d'espaces topologiques.

    Un \textit{homéomorphisme} $f : X \to Y$ est une bijection continue telle que 
    \[
    f^{-1} : Y \to X
    \]
    est continue.\footnote{Ce n'est pas évident : par exemple, il y a une bijection $[0, 1] \to \mathds S^1$ (où $\mathds S^1$ est le cercle unité de $\mathds{R}^2$) continue mais la réciproque ne l'est pas.}
  \end{defn}

  \begin{lem}
    Une fonction $f : \Sigma^\omega \to \Gamma^\omega$ est continue si et seulement si
    \begin{gather*}
    \forall \alpha \in \Sigma^\omega, \forall n \in \mathds{N},
    \exists k \in \mathds{N}, \forall \beta \in \Sigma^\omega,\\
    \beta(0)\ldots\beta(k) = \alpha(0)\ldots\alpha(k)\\
    \vertical\implies\\
    f(\beta)(0) \ldots f(\beta)(n) = f(\alpha)(0) \ldots f(\alpha)(n)
    .\end{gather*}

    Autrement dit, $f$ est continue ssi on peut déterminer une partie finie de sa sortie à partir d'une partie finie de son entrée.
  \end{lem}

  Soit $P \subseteq \Sigma^\omega$, et on définit la \textit{fonction caractéristique} de $P$ :
  \begin{align*}
    \chi_P: \Sigma^\omega &\longrightarrow \mathbf{2} = \{0,1\} \\
    \alpha &\longmapsto \begin{cases}
      1 & \text{ si } \alpha \in P\\
      0 & \text{ si } \alpha \not\in P\\
    \end{cases}
  .\end{align*}

  Avec $\Omega \mathbf{2} = \wp(\mathbf{2}) = \{\emptyset, \{0\} , \{1\} , \{0,1\}\}$ (ce qui est cohérent avec l'idée que $\mathbf{2}$ représente les booléens), on a que $\chi_P$ est continue ssi 
  \begin{itemize}
    \item $\chi_P^\bullet \{0\} = \Sigma^\omega \setminus P$ est un ouvert ;
    \item $\chi_P^\bullet \{1\} = P $ est un ouvert.
  \end{itemize}
  On arrive donc à la notion de \textit{clopen}.

  \begin{defn}
    Soit $(X, \Omega X)$ un espace topologique.
    Une partie~$P \subseteq X$ est \textit{clopen} (\textit{ouvert fermé} en français) si $P$ et $X \setminus P$ sont ouverts.
  \end{defn}

  \begin{rmk}
    \begin{enumerate}
      \item On a $\emptyset$ est clopen, et que, si $A$ et $B$ sont clopen alors $A \cup B$ est clopen.
      \item On a $X$ est clopen, et que, si $A$ et $B$ sont clopen alors $A \cap B$ est clopen (dual du point précédent).
      \item Si $A$ est clopen alors $X \setminus A$ est clopen.
    \end{enumerate}
  \end{rmk}

  \begin{exm}
    Soit $u \in \Sigma^\star$, on a que $\mathsf{ext}(u)$ est ouvert.
    Mais, on a aussi que $\Sigma^\omega \setminus \mathsf{ext}(u)$ est ouvert :
    \[
    \Sigma^\omega \setminus \mathsf{ext}(u) = \bigcup \{ \mathsf{ext}(v)  \mid v \neq u \text{ et } \mathsf{length}(v) = \mathsf{length}(u) \}
    .\]
  \end{exm}

  \begin{rmk}
    Sur $(\Sigma^\omega, \Omega\Sigma^\omega)$, tous les $\mathsf{ext}(W)$ où $W \subseteq \Sigma^\star$ est \textit{\textbf{fini}} sont clopen.
    La réciproque est fausse, comme le montre le lemme suivant.
  \end{rmk}

  \begin{lem}
    Si $\Sigma$ est infini et $a \in \Sigma$, alors
    \[
    \Sigma^\omega \setminus \mathsf{ext}(a) = \bigcup_{\Sigma \ni b \neq a}  \mathsf{ext}(b)
    \]
    est clopen mais \textit{\textbf{pas}} de la forme $\mathsf{ext}(W)$ avec $W$ fini.
  \end{lem}

  \section{Compacité.}

  \begin{defn}
    Soit $(X, \Omega X)$ un espace topologique.
     \begin{enumerate}
      \item Une partie $A \subseteq X$ est \textit{compacte} si, pour toute famille $(V_i)_{i \in I} \in \Omega X^I$ telle que $A \subseteq \bigcup_{i \in  I} V_i$, il existe $J \subseteq I$ \textit{\textbf{fini}} tel que $A \subseteq \bigcup_{j \in J} V_j$.
      \item On dit que $(X, \Omega X)$ est \textit{compact} si $X$ est une partie compacte.
    \end{enumerate}
  \end{defn}

  \begin{rmk}[Non-exemple]
    Si $\Sigma$ est infini alors $\Sigma^\omega$ n'est pas compact :
    \[
    \Sigma^\omega = \bigcup_{a \in \Sigma} \mathsf{ext}(a)
    .\]
  \end{rmk}

  \begin{prop}
    Si $\Sigma$ est \textit{\textbf{fini}} alors $\Sigma^\omega$ est compact.
  \end{prop}
  \begin{prv}
    On procède à l'aide du lemme de Kőnig.
    Supposons que $\Sigma^\omega = \bigcup_{i \in  I} U_i$ où $U_i \in \Omega\Sigma^\omega$.
    On a que $U_i = \mathsf{ext}(V_i)$ pour un $V_i \subseteq \Sigma^\star$ (en général, $V_i$ est infini).
    Soit $V = \bigcup_{i \in I} V_i \subseteq \Sigma^\star$, et on vérifie que $\mathsf{ext}(V) = \Sigma^\omega$.
    Pour chaque $n \in \mathds{N}$, on définit $W_n \subseteq \Sigma^n$ par récurrence :
    \begin{itemize}
      \item On pose $W_0 := \{\varepsilon\}$ si  $\varepsilon \in V$ et $W_0 := \emptyset$ sinon.
      \item On pose
        \[
          W_{n+1} := \mleft\{\, u \in V \;\middle|\; u \text{ n'a pas de préfixe dans } {\textstyle \bigcup_{k \le n} W_k} \,\mright\} 
        .\]
    \end{itemize}
    On pose enfin $W = \bigcup_{n \in \mathds{N}} W_n$.
    On a que $\mathsf{ext}(W) = \mathsf{ext}(V)$ (car, pour tout $v \in V$, il existe $w \in W$ tel que $w \subseteq v$), et $W$ est "prefix-free" (c'est-à-dire que, pour $w, w' \in W$, on a $w \not\subseteq w'$ ssi $w \neq w'$).

    Si $W$ est fini alors on s'arrête.

    Par l'absurde, supposons $W$ infini, et posons $T = \mathrm{Pref}(W)$ qui est un arbre par définition.
    L'arbre $T$ est à branchement fini (car $\Sigma$ est fini), et $T$ est infini (car $W$ l'est)
    Par le lemme de Kőnig, il existe un chemin infini $\pi \in \Sigma^\omega$ dans $T$.
    Comme $\Sigma^\omega = \mathsf{ext}(W)$, il existe $u \in W$ tel que $u \subseteq \pi$.
    De plus, il existe $a \in \Sigma$ tel que $ua \subseteq \pi$ et donc $ua \in T = \mathrm{Pref}(W)$.

    On arrive à une contradiction car $u \in W$ et $W$ est prefix-free.
  \end{prv}

  \begin{crlr}
    On a que $\Sigma^\omega$ est compact ssi $\Sigma$ fini.
    \qed
  \end{crlr}

  \begin{lem}
    Si $(X, \Sigma X)$ est compact et $C \subseteq X$ est fermé alors $C$ est compact.
  \end{lem}
  \begin{prv}
    L'idée est que si $C \subseteq \bigcup_{i \in I} V_i$ alors $X \subseteq (X \setminus C) \cup \bigcup_{i \in I} V_i$.
  \end{prv}

  \begin{crlr}
    Si $\Sigma$ est fini alors $A \subseteq \Sigma^\omega$ est clopen ss'il existe $W \subseteq \Sigma^\star$ \textit{\textbf{fini}} tel que $A = \mathsf{ext}(W)$.
    \qed
  \end{crlr}

  \section{Espace Hausdorff.}

  \begin{defn}
    On dit que $(X, \Omega X)$ est \textit{Hausdorff} (ou $\mathrm{T_2}$) lorsque, pour tout $x \neq  y \in X$, alors il existe $U, V \in \Omega X$ tels que 
    \[
    U \cap V = \emptyset \quad\quad x \in U \quad\text{et}\quad y \in V
    .\]
  \end{defn}

  \begin{exm}
    L'espace $(\Sigma^\omega, \Omega\Sigma^\omega)$ est Hausdorff.
    Soient $\alpha \neq \beta$.
    Il existe $u \subseteq \alpha$ et $v \subseteq \beta$ tels que $\mathsf{ext}(u) \cap \mathsf{ext}(v)$.

    (On peut choisir $u = p \alpha(\mathsf{length}(p))$ et $v = p \beta(\mathsf{length}(p))$ où $p $ est le plus long préfixe commun à $\alpha$ et $\beta$.)
  \end{exm}

  \begin{prop}
    Si $(X, \Omega)$ est compact Hausdorff et $C \subseteq X$ est compact alors $C$ est fermé.
  \end{prop}
  \begin{prv}
    Soit $C \subseteq X$ est compact et $x \not\in C$.
    Pour tout $y \in C$, il existe $U_y, V_y$ tels que $U_y \cap V_y = \emptyset$ et $x \in U_y$ et $y \in V_y$.
    Donc, on a $C \subseteq \bigcup_{y \in C} V_y$.
    Comme $C$ est compact, il existe $y_1, \ldots, y_n \in C$ tels que $C \subseteq V_{y_1} \cup V_{y_2} \cup \cdots \cup V_{y_n}$.
    On a que $x \in U_{y_1} \cup \cdots  \cup U_{y_n} =: U \in \Omega X$.
    Et, $U \subseteq X \setminus C$ car, $U \cap (V_{y_1} \cup \cdots \cup V_{y_n}) = \emptyset$.
  \end{prv}

  \begin{crlr}
    Si $(X, \Omega X)$ est compact Hausdorff et $C \subseteq X$, 
    \[
    \text{$C$ compact} \iff \text{$C$ fermé}
    .\] 
  \end{crlr}
\end{document}
