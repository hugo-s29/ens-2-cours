\documentclass[./main]{subfiles}

\begin{document}
  \chapter{Circuits quantiques.}

  On donne une définition informelle des circuits quantiques :
  \begin{itemize}
    \item le temps va de gauche à droite ;
    \item les lignes horizontales sont les quibts ;
    \item les opérations sont les matrices unitaires ou des mesures.
  \end{itemize}
  Voici quelques exemples de circuits quantiques.

  Avec la matrice de Hadamard :
  \begin{center}
    \begin{quantikz}
      \ket 0 & \gate{\mathbf{H}} & \meter{} &
      \ensuremath{
        \begin{cases}
          0 &\text{avec une probabilité } 1/2\\
          1 &\text{avec une probabilité } 1/2
        \end{cases}
      }
    \end{quantikz}
  \end{center}

  La matrice $\mathbf{CNOT} = \begin{pmatrix} 1 & 0 & 0 & 0  \end{pmatrix} $
  \begin{quantikz}
    \ket a & \ctrl{1} & \ket a \\
    \ket b & \targ{} & \ket{a \oplus b}
  \end{quantikz}
\end{document}
