\documentclass[./main]{subfiles}

\begin{document}
  \chapter{Introduction.}

  \section{Calculabilité.}

  Un ordinateur manipule des \textit{bits} (0 ou 1) via des \textit{opérations} (des programmes).
  Un résultat (c'est plus une hypothèse) est la thèse de Church-Turing étendue :
  \begin{quote}
    Tout calcul réaliste peut être efficacement simulée par une machine de Turing.
  \end{quote}
  L'intuition est que les systèmes physiques sont simulables en temps polynomial.
  Le contre-exemple est la physique quantique.

  % \subsection{Un peu d'histoire\ldots}
  % flemme de faire la frise de l'histoire de la quantique

  \subsection{Objectifs de ce cours.}
  Ce qui est dans ce cours sera :
  \begin{itemize}
    \item les bases du formalisme quantique ;
    \item le calcul algorithmes principaux ;
    \item la communication avec les protocoles principaux,
  \end{itemize}
  mais il ne sera pas :
  \begin{itemize}
    \item un cours de physique ;
    \item un cours sur les avancées récentes.
  \end{itemize}
\end{document}
